\subsection{Модальное 	управление непрерывными и дискретными объектами}

Для объекта
\begin{equation}
    \dot x = A x + bu, \quad y = Cx,
\end{equation}
выбирается П-регулятор (он же модальные) вида
\begin{equation}
    u = - K x
\end{equation}

Такой регулятор вводит обратные свзи по переменным состояния объекта.

Объединение регулятора и объекта дает замкнутую систему:
\begin{equation}
    \dot x = F x = (A - bK) x, \quad y = C x
\end{equation}
где $К$~---матрица строка коэффициентов обратной связи, $F$~--- матрица замкнутой системы.

Выбор коэффициетов обратных связей матрица K обеспечивает получение заданных динамических свойст системы: быстродействие и колебательность.

В соответствиии с методом модального управления устойчивость и заданные показатели качетва достигаются назначением корней характеристического полинома замкнутой системы.

Метод основывается на следующем положении. Если система полностью управляема, то существует единственная матрица обратной свзяи K, обеспечивающая получение заданных значений корней характеристического полинома замкнутой системы.

При отсутствии возмущений, регулятор обеспечивает абсолютную точность стабилизации системы.

Расчет маатрица K:
\begin{enumerate}
    \item По заданным показателям качества находятся корни характеристического полинома $p_i$ и его коэффицинты $a_i$ (например, методом стандартных переходных функций)
    \item Рассчитываются коэффициенты обратной связи K соответствующие канонической управляемой форме объекта
    \begin{equation}
        k_i = a_{ci} - a_i        
    \end{equation}
    \item для обратного перехода от канонической формы к исходной используется матрица подобия
    \begin{equation}
        P = U^* \cdot U^{T}
    \end{equation}
    где $U, U^*$~--- матрицы управляемости исходной и канончиской форм.
    
    Преобразование для перехода 
    \begin{equation}
        K = K^* \cdot P
    \end{equation}
\end{enumerate}


\subsection{Алгоритм 	синтеза модального управления непрерывным 	и дискретными ОУ при полной измеримости его вектора состояния, использующий решение матричного уравнения Сильвестра}

Для объекта
\begin{equation}
    \dot x = A x + bu, \quad y = Cx,
\end{equation}
выбирается П-регулятор (он же модальные) вида
\begin{equation}
    u = - K e \quad e = - x
\end{equation}

Решение задачи:
\begin{enumerate}
    \item Вводится модель ошибок замкнутой системы
    \begin{gather}
        \begin{cases}
            \dot e = F e \\
            v = C e
        \end{cases}
    \end{gather}
    \item Задается эталонная модель ошибор
    \begin{gather}
        \begin{cases}
            \dot \xi = \Gamma \xi, \quad \xi(0) = \xi_0 \\
            \nu = H \xi
        \end{cases}
    \end{gather}
    где $\Gamma$~--- матрица, определяющая требуемые свойства СУ.
    
    ХП матрица $\Gamma$ должен совпадать с требуемым (метод стандартных переходных функций).
    $H$ выбирается из условия полной наблюдаемости пары $\Gamma, H$.
    
    Связь между исходной моделью ошибок и элатонной выражается как
    \begin{equation}
        e = M \xi
    \end{equation}
    
    \item решается систему уравнений включающая уравнение Сильвестра
    \begin{equation}
        \begin{cases}
        M \Gamma - AM = bH \\
        K = - H M^{-1}
        \end{cases}
    \end{equation}
    Все!
    Замечание. Уравнение Сильвестра имеет единственное решение относительно M, если:
    \begin{enumerate}
        \item ОУ полностью управляем (нужно перед всем этим этим проверять)
        \item ЭМ полностью наблюдаема
        \item A и $\Gamma$ не имеют одинаковых корней
    \end{enumerate}
\end{enumerate}


