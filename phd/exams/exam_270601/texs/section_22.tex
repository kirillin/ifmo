\subsection{Точность непрерывных систем в типовых режимах}
\subsection{Способы	повышения точностных характеристик 	систем}

Поведение невозмущенной замкнутой системы $y(t) = \Phi(s) y^* (t)$ для различных типов задающего воздействия 
\begin{equation}
    y^* (t) = y_0^* + y_1^* t + y_2^* t^2 + \dots, \quad
    y_i^* (t) = \cfrac{y^* (0)}{i!} = const
\end{equation}
В зависимости от $y_i^*$ различают режимы:
\begin{enumerate}
    \item стационарный (стабилизация) $y^*=y^*_0$
    \item режим движения с постоянной скоростю $y^* = V_0 t$
    \item режим движения с постоянным ускорением $y^* = \cfrac{a_0 t^2}{2}$
\end{enumerate}

Вводятся коэффициенты ошибок для оценки точность
\begin{gather}
    \Phi_0 = \Phi(0) = \cfrac{W(s)}{1 + W(s)}|_{s=0};
    \quad
    \Phi_1 = \Phi'(0) = \cfrac{W'(s)}{(1+ W(s))^2}|_{s=0}
    \quad
    \Phi_2 = \Phi'' (0) = \cfrac{W''(s) (1 + W(s)) - 2 W'(s)}{(1+W(s))^3}|_{s=0}\\
    y_{\text{уст}}(t) = \Phi_0 y^* + \Phi_1 \dot y^* + \cfrac{\Phi_2}{2!} \ddot y^* + \dots\\
    e_{\text{уст}}(t) = (1 - \Phi_0) y^* - \Phi_1 \dot y^* - \cfrac{\Phi_2}{2!} \ddot y^* + \dots
\end{gather}

\begin{enumerate}
    \item В стационарном режиме $y_{\text{уст}} = \Phi_0 y^*$.
Абсолютняа точность и установившаяся ошибка достигается при $\Phi_0 = 1$, что выполняется для астатических систем. Для статическых систем $e_{\text{уст}} = \cfrac{1}{1+k} y^*$. Ошибку можно уменьшить засчет увеличения коэффициента обратной связи.
    
    \item В режиме движения с постоянной скоростью имеет место
    \begin{equation}
        y_{\text{уст}} = \Phi_0 y^* + \Phi_1 V_0 = \Phi_0 V_0 t + \Phi_1 V_0
    \end{equation}
    Абсолютная точность при $\Phi_0 = 1, \Phi_1 = 0$~--- астатизм 2 порядка и выше.
    
    Для астататизма перого порядка $e_{\text{уст}} = \cfrac{1}{k_1} V_0$. Для уменьшения установившейся ошибки увеличивать коэффициенты обратной связи.
    
    Для статического режима~--- с течением времени ошибка неограничено возрастает.
    
    \item В режиме с постоянные ускорением
    \begin{equation}
        y_{\text{уст}} = \Phi_0 y^* + \Phi_1 a_0 V + \cfrac{\Phi_2}{2!} a_0 = \Phi_0 \cfrac{a_0 t^2}{2!} + \Phi_1 a_0 t + \Phi_2 \cfrac{a_0}{2}
    \end{equation}
    Абсолютная точность $\Phi_0=1, \Phi_1 = \Phi_2 = 0$, что выполняется для астатических систем порядка 3 и выше
    
    Для астатизма второго порядка $e_{\text{уст}} = \cfrac{1}{k}a_0$
\end{enumerate}

Повышение точности достигается:
\begin{enumerate}
    \item подключением обратных вязей (замыкание системы).
    \item повышение порядка статизма основного канала с помощбю соответствующих регуляторов (алгоритмов управления)
    \item повышение добротности системы за счет увеличения коэффициентов регуляторов.
\end{enumerate}
