\subsection{Структурные свойства линейных непрерывных и дискретных ОУ: управляемость и 	наблюдаемость}
\subsection{Критерии управляемости и наблюдаемости непрерывных 	и дискретных ОУ}


\textbf{Управляемость}~--- это структурное свойство объекта (модели), означающее что что перезод из любого состояния системы в любое другое систояние может быть достигнуто приложением некоторого  ограниченного управляющего воздействия.

С. полностью управляема, если для любых $t >= 0$ и $x_f \in \mathcal{R}$ существует такое $t_f >= t_0)$ и ограниченное управляющее воздействие $u(t), t \in [t_0, t_f]$ такое, что для $x(t_0) = x_0$ выполняется $x(t_f) = x_f$.

Матрица управляемости:
\begin{equation}
    U = [B, AB, \dots,A^{n-1} B]
\end{equation}

Критерий управляемости:
\begin{enumerate}
    \item Линейная система полностью управляемая тогда и только тогда, когда (если $rank(U) = n$, $n$ порядок системы).
    \begin{equation}
        \det(U) \ne 0.
    \end{equation}

    \item C. полностью управляема тогда и только тогда, когда она может быть преобразована к канонической управляемой форме.
\end{enumerate}
Свойство управляемости не зависит от выходной переменной.

\textbf{Наблюдаемость}~--- свойство системы, при котором ее переменные состояния могут быть однозначно определены по выходной переменной y.

Система нызвается полностью наблюдаемой, если для любых $t_0\ge 0$ существует $t_1 > t_0$ такое, что выходной переменной
\begin{equation}
    y = y(t), t \in [t_0, t_1]
\end{equation}
полученной для входного сигнала $u(t)$, соответствуйте единственное значение $x(t_0) = x_0$.

Матрица наблюдаемости:
\begin{equation}
    Q = 
    \begin{bmatrix}
        C\\
        CA\\
        \cdots\\
        C A^{n-1}
    \end{bmatrix}
\end{equation}

Линейная система полностью наблюдаема, если $rank (Q) = n$, т.е
\begin{equation}
    \det Q \ne 0.
\end{equation}