\subsection{Корневые 	условия устойчивости непрерывных и 	дискретных систем}

\textcolor{blue}{\textbf{Корневые критерии устойчивости}} связывают понятие устойчивости с размещением корней на комплексной плоскости:
\begin{enumerate}
    \item Для непрерывных систем: 
    \begin{enumerate}
        \item устойчива~--- все корни в левой полуплоскости;
        \item устойчива по Ляпунову~--- один корень на границе устойчивости;
        \item неустойчива~--- хотябы один корень в правой полуплоскости.
    \end{enumerate}
    
    \item Для дискретных систем (могут быть эелементарно выведены из теории непрерывных систем, учитывая связь корней Н. и Д. систем $z_i = e^{T p_i}$):
    \begin{enumerate}
        \item устойчива~--- все корни строго внутри единичной окружности;
        \item устойчива по Ляпунову~--- один корень на окружности;
        \item неустойчива~--- хотябы один корень вне единичной окружности.
    \end{enumerate}
\end{enumerate}

Критерии:
\begin{enumerate}
    \item Система асимптотически устойчива тогда и только тогда, когда выполняется условие:
    
    Звенья: апериодической (I и II), колебательное.
    \begin{equation}
        Re p_i < 0, \quad i = \overline{1,n}.
    \end{equation}
    
    \item Система устойчива по Ляпунову, если выполняется одно из условий:
        \begin{enumerate}
            \item Апериодическая граница устойчивости (наличие одного вещественного корная на границе устойчивости (мнимой оси)): 
    
            Звенья: Интегратор.
            \begin{equation}
                \begin{cases}
                    \Re p_i < 0, \quad i = \overline{1, n-1}, \\
                    p_n = 0
                \end{cases}
            \end{equation}
            
            \textcolor{green}{
                \textbf{Дискретная полностью наблюдаемая система асимпототический устойчива тогда и только тогда, когда выполняется условие}:
                \begin{equation}
                    |z_i| = |\lambda_i\{A\}| < 1, \quad i = \overline{1,n}.
                \end{equation}
            }
    
            \item Колебательна граница устойчивости (наличие двух комплексно-сопряженных корней на мнимой оси):
            
            Звенья: консервативное (генератор).
            \begin{equation}
                \begin{cases}
                    \Re p_i < 0, \quad i = \overline{1, n-2}, \\
                    \Re p_{n-1, n} = 0, \quad \Im p_{n-1, n} \ne 0.
                \end{cases}
            \end{equation}
        \end{enumerate}
        
    \item Система неустойчива, если хотябы один полюс такой, что
    \begin{equation}
        \Re p_i > 0.
    \end{equation}
\end{enumerate}