\subsection{Частотные критерии устойчивости непрерывных систем}

\textcolor{blue}{\textbf{Критерий Найквиста}} позволяет определить устойчивость замкнутой системы, построив частотную характеристику разомкнутой системы.

Чтобы система в замкнутом состоянии была устойчивой необходимо и достаточно, чтобы при изменении $\oemga$ от $-\infty$ до $\infty$  годограф разомкнутой системы $W(j \omega)$ (АФХ), поворачиваясь вокруг начала координат по часовой стрелке, охватил точку $(−1,j0)$ столько раз, сколько корней в правой полуплоскости содержит знаменатель $W(j \omega)$.

Примечания:
\begin{enumerate}
    \item Если корней в правой полуплоскости нет, то годограф $W(j\omega)$ не должен охватить точку $(−1,j0)$.
    \item Неустойчивая система в разомкнутом состоянии может быть устойчивой в замкнутом состоянии. И наоборот.
    \item Годограф $W(j\omega)$ всегда начинается на оси "$+1$". Но при порядке астатизма равном r, по причине устремления $W(j\oemga)$ к $\infty$ (при $\omega \rightarrow$), видимая часть годографа появляется только в квадранте r, отсчитанном по часовой стрелке.
\end{enumerate}

Свойства годографа Найквиста
\begin{enumerate}
    \item Годограф Найквиста спиралевиден. При $\oemga \rightarrow \infty$ годограф $W(j\omega)\rightarrow 0$, т.к. нет безынерционных систем.
    \item Годограф статических САР начинается из точки на вещественной оси.
    \item Для положительных и отрицательных частот годографы зеркально симметричны относительно оси "+1". Наличие корней на границе устойчивости приводит к устремлению годографа в $\infty$ и приращению его фазы на $-180^o$.
\end{enumerate}

Для определения устойчивости по критерию Найквиста можно строить не АФХ, а ЛАЧХ и ЛФЧХ разомкнутой системы. Чтобы замкнутая система была устойчива, необходимо и достаточно, чтобы сдвиг фазы на частоте единичного усиления разомкнутой системы $W(j \omega)$ не достигал значения $-180^o$.
Если система условно устойчивая, то при модулях больших единицы, фазовый сдвиг может достигать значения $-180^o$ четное число раз.

\textcolor{blue}{\textbf{Критерий устойчивости Михайлова}}.

Чтобы все корни ХУ имели отрицательные вещественные части, необходимо чтобы после подстановки частоты в соответствующий характеристический полином $D(s)$ полное приращение его фазы при изменении $\oemga$ от $0$ до $\infty$ составляло $n \pi / 2$, где $n$~--- степень полинома $D(s)$. При этом характеристический полином опишет в комплексной плоскости кривую~--- "годограф Михайлова".

\begin{enumerate}
    \item Пусть $s_i=\alpha$~--- вещественный положительный корень. Тогда годограф соответствующего линейного множителя $(j\omega - \alpha)$ при изменении $\omega$ от 0 до $\infty$ повернется на угол $-\pi/2$.

    \item Пусть $s_i=-\alpha$~--- вещественный отрицательный корень. Тогда годограф соответствующего линейного множителя $(j\omega + \alpha)$ при изменении $\omega$ от 0 до $\infty$ повернется на угол $\pi/2$.

    \item Пусть $s_{i; i+1}= \alpha \pm j\beta$~--- сопряженные корни с положительной вещественной частью. Тогда годографы соответствующих линейных множителей $(j\omega−\alpha−j\beta)(j\omega−\alpha+j\beta)$ при изменении $\omega$ от 0 до $\infty$ повернутся на углы $-\pi/2+\gamma$, и $\pi/2-\gamma$. Вектор, соответствующий произведению двух сомножителей, повернется на угол равный $-\pi$.

    \item Пусть $s_{i; i+1}=-\alpha \pmj\beta$~--- сопряженные корни с отрицательной вещественной частью. Тогда годографы соответствующих линейных множителей $(j\omega+\alpha-j\beta)(j\omega+\alpha+j\beta)$ при изменении $\omega$ от 0 до $\infty$ повернутся на углы $\pi/2−\gamma$, и $\pi/2+\gamma$. Вектор, соответствующий произведению двух сомножителей, повернется на угол равный $\pi$.
\end{enumerate}


Резюме: Если ХУ имеет $l$ корней с положительной вещественной частью, то угол поворота годографа $D(j\omega)$ при изменении $\omega$ от 0 до $\infty$ составит:
\begin{equation}
    \psi= - l \pi 2+ (n-l) \pi 2=n \pi 2-l \pi,    
\end{equation}
где: $n$~--- порядок ХУ.

Свойства годографа Михайлова
\begin{enumerate}
    \item Годограф всегда спиралевиден.
    \item При $\omega=0$, будет $\psi=0$. Следовательно, годограф начинается с точки на оси "+1".
    \item Поскольку при $\omega\rightarrow\infty, K(j\omega)\rightarrow 0$ (нет безынерционных систем), годограф уходит в бесконечность.
    \item При четном n, годограф стремится к $\infty$ параллельно оси "+1"; при нечетном n, годограф стремится к $\infty$ параллельно оси "+j".
\end{enumerate}

Определение типа границы устойчивости по виду годографа Михайлова
\begin{enumerate}
    \item Астатизм первого порядка~--- "апериодическая" граница устойчивости.
    \item Астатизм второго порядка~--- "апериодическая" граница устойчивости.
    \item "Колебательная" граница устойчивости.
    \item Граница устойчивости типа "бесконечный корень".
\end{enumerate}
