\subsection{Алгебраические критерии устойчивости непрерывных систем}
\subsection{Методы Ляпунова в исследовании устойчивости непрерывных и дискретных систем}

Критерии устойчивости позволяют осуществить анализ системы, не прибегая к определениям устойчивость (где нужно было бы анализировать все возможные переходные прцессы), т.е. на основании косвенных признаков устойчивости, связанных со свойствами математических моделей.

Для автономной системы, решения для которой представляются в виде:
\begin{gather}\label{solves}
    x(t) e^{At}x_0,
    \quad
    y(t) = C^T e^{At} x_0 = \sum\limits_{i=1}^{n} C_i e^{p_i t},
\end{gather}
где $C_i$~--- коэффициенты, зависящие от НУ, $p_i$~--- полюсы системы, или корни характеристическогоого полинома:
\begin{equation}
    a(p) = \det(pI - A) = p^n + a_1 p^{n-1} + \dots + a_n.
\end{equation}

Выражения~\eqref{solves} плказывают, что свойства преходных процессов в целом определяются свойствами компоненты $e^{At}$, которые зависят от полюсов системы (корней характеристического полинома), или, что тоже самое, от собственных чисел матрицы $A$. 

\begin{enumerate}
    \item \textcolor{blue}{Метод Гурвица}
    
    \textbf{Теорема Стодолы}: если система устойчива, то коэффициенты характеристического полинома (ХП) строго положительны (обратное верно только для систем 1 и 2 порядка (\textcolor{gray}{чем и обусловлены более сложные условия в критерии Гурвица и подобных})).
    
    \textbf{Критерий Гурвица}:
    \begin{enumerate}
        \item Система с характеричтическим уравнением устойчива тогда и только тогда, когда все определители $\Delta_i$ матрицы Гурвица строго положительны: $\Delta_i > 0, i=\overline{1,n}$.
        \item Система нейтрально устойчива, елси выполняется одно из условий:
        \begin{enumerate}
            \item Апериодическая граница устойчивости (итегратор)
            \begin{equation}
                \begin{cases}
                    \Delta_i > 0, \quad i=\overline{1, n-1}, \\
                    \Delta_n = \Delta_{n-1} a_n = 0,
                \end{cases}
            \end{equation}
            \item Колебательная граница устойчивости (консервативное звено)
            \begin{equation}
                \begin{cases}
                    \Delta_i > 0, \quad i=\overline{1, n-2}, \\
                    \Delta_{n-1} = 0, \\
                    a_n > 0.
                \end{cases}
            \end{equation}
        \end{enumerate}
    \end{enumerate}
    
    Матрица гурвица:
    \begin{equation}
        \begin{bmatrix}
            a_1 & a_2 & a_3 &\cdots & 0 & 0 \\
            a_0 & a_2 & a_3 &\cdots & 0 & 0 \\
            0 & a_1 & a_3 &\cdots & 0 & 0 \\
            \cdots & \cdots & \cdots & \cdots & \cdots & \cdots \\
            0 & 0 & 0 &\cdots & a_{n-1} & 0 \\
            0 & 0 & 0 &\cdots & a_{n-2} & a_n
        \end{bmatrix}
    \end{equation}
    
    Определители Гурвица~--- главные диагональные миноры матрица М:
    \begin{equation}
        \Delta_1 = a_1, \quad
        \Delta_2 = 
        \begin{bmatrix}
            a_1 & a_3 \\
            a_0 & a_2
        \end{bmatrix}
        = a_1 a_2 - a_0 a_3, \quad
        \cdots, \quad
        \Delta_n = |M| = a_n \Delta_{n-1}.
    \end{equation}
    
    \item \textcolor{blue}{Корневые критерии} \textcolor{red}{(см. предыдущий вопрос про корневые критерии)}
    \begin{equation}
        \cdots
    \end{equation}
    
    \item \textcolor{blue}{Метод Ляпунова}
    
    Матрица A назвается устойчивой или гурвицевой, если выполняется условие $\Re(\lambda_i\{A\}) < 0, i = \overline{1,n}$.
    
    \textbf{Лемма Ляпунова.} Матрица A устойчива тогда и только тогда, когда для любых матриц $Q = Q^T > 0$ уравнение имеет положительно определенное решение $P>0$:
    \begin{equation}
        A^T P + PA = - Q
    \end{equation}
    где $Q = Q^T >0$.
    
     Т.о. существование решения уравнения Ляпунова обеспеивает устойчивость матрицаы A (асимптотическая устойчивость).
     
     Для оценци качества сходимости ПП устойчивой системы рассматривается следующее выражение:
     \begin{equation}\label{eqL}
         A^T P + PA = -Q - 2 \alpha P
     \end{equation}
    
    \textbf{Свойство.} Матрица A удовлетворяет условию
    \begin{equation}
        \Re(\lambda_i\{A\}) < -\alpha, i = \overline{1,n}
    \end{equation}
    тогда и только тогда, когда уравнение~\eqref{eqL} для любых $Q=Q^T >0$ имеет положительно определеное решение $P>0$.
    
    Это условие при $\alpha > 0$ означает, что полюсы системы, смещены влево от границы устойчивости на велечину, превышающую значение $\alpha$, что предполагает асимптотическую устойчивость с \textit{запасом устойчивости}.
\end{enumerate}
