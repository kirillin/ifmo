\subsection{Преобразование Лапласа и его основные свойства}
\textit{Преобразование Лапласа}~--- интегральное преобразование, связывающее функцию $F(s)$ (изображение) комплексного аргумента с функцией $f(t)$ (оригинал) действительного аргумента.

Преобразование Лапласа справедливо только для оригиналов. Оригинал~--- это комплекснозначная функция $f(t)$ действительного аргумента $t$, которая удовлетворяется свойствам:
\begin{enumerate}
    \item $f(t) = 0$, при $t < 0$;
    \item на любом конечном интервале имеет не более чем конечное число точек разрыва;
    \item f(t) имеет ограниченный рост, т.е. возрастает не быстрее показательной функции: существуют такие постоянные $M > 0$ и $a >= 0$, что $|f(t)| < M e^{at}$, при $t > 0$.
\end{enumerate}

Также, для всякого оригинала $f(t)$ изображение по Лапласу $F(s)$ определено в полуплоскости $Re s > s_0$ и является в этой полуплоскости аналитической функцией.

\textbf{Преобразование Лапласа} $\mathfrak{L}\{f(x)\}$~--- позволяет перейти от функции $f(t)$ действительной переменной к функции $F(s)$ комплексной переменной:
\begin{equation}\label{laplase}
    F(s) = \mathfrak{L}\{f(t)\} = \int\limits_{0}^{\infty} f(t) e^{-st} dt
\end{equation}
где $s = \sigma + j w$~--- комплексная переменная, которая выбирается так, чтобы интеграл~\eqref{laplase} сходился. Правая часть выражения~\eqref{laplase} называется \textit{интегралом Лапласа}.

\textbf{Обратное преобразования Лапласа} $\mathfrak{L}^{-1}  \{F(s)\}$ позволяет все сделать наоборот:
\begin{equation}\label{invlaplase}
    f(t) = \mathfrak{L}^{-1}  \{F(s)\} = \cfrac{1}{2 \pi j} \: \int\limits_{\sigma - j \cdot \infty}^{\sigma + j \cdot \infty} F(s) e^{s \tau} ds
\end{equation}
где $j=\sqrt{-1}$, $\sigma$~--- выбирается так, чтобы интеграл сходился. Интеграл~\eqref{invlaplase} называется \textit{интегралом Бромвича}.

Примеры:
\begin{gather*}
    \mathfrak{L}\{\delta(t)\} = 1, \quad
    \mathfrak{L}\{\mathbf{1}(t)\} = \cfrac{1}{s}, \quad
    \mathfrak{L}\{e^{-at}\} = \cfrac{1}{s + a}.
\end{gather*}

% \textbf{Дискретное преобразование Лапласа}~--- \textcolor{red}{???}.

% Различают два вида дискретных преобразований:
% \begin{itemize}
%     \item $\mathcal{D}$--преобразование
    
%     \item $\mathcal{Z}$--преобразование
%     Если произвести замену $z = e^{sT}$, то можно получить:
%     \begin{equation}
%         \mathcal{Z(x_d(t)} = \sum\limits_{n=0}^{\infty} x(nT) \cdot z^{-n}
%     \end{equation}
% \end{itemize}

\textbf{\textcolor{blue}{Свойства}}

\begin{enumerate}
    \item Линейность
        \begin{equation}
            a f(t) + b g(t) = a F(s) + b G(s)
        \end{equation}
    \item Теорема подобия
        \begin{equation}
            f(at) = \cfrac{1}{a} F\Big(\cfrac{s}{a}\Big)
        \end{equation}
    \item Теорема запаздывания

        для любого $\tau>0$:
        \begin{equation}
            f(t-\tau) = e^{-st} F(s)
        \end{equation}

    \item Теорема смещения

        для любого $s_0$:
        \begin{equation}
            e^{s_0 t} f(t) = F(s - s_0)
        \end{equation}
    
    \item Дифференцирование оригинала
    
        \begin{equation}
            f^{(n)}(t) = s^n F(s) - \sum\limits_{k=0}^{n-1} s^{n-1-k} f^{(k)}(0),
        \end{equation}
        где $f^{(k)}(0)$~--- предел справа в т. $t=0$.
        
    \item Дифференцирование изображения
        
        Дифференцирование изображения сводится к умножению оригинала на $(-t)$:
        \begin{equation}
            F^{n} (s) = (-t)^n f(t)    
        \end{equation}
    
    \item Интегрирование оригинала и изображения
    
        \begin{equation}
            \int\limits_0^{\infty} f(\tau) d\tau = \cfrac{F(s)}{s}
        \end{equation}
        
        Если интеграл с изображением сходится, то справедливо:
        \begin{equation}
            \int\limits_s^{\infty} F(s) ds = \cfrac{f(t)}{t}
        \end{equation}
        
    \item Теорема о свертке
    
    Свертке оригиналов соответствует произведение изобраений
    \begin{equation}
        f(t) \circledast g(t) = F(s) \cdot G(s)
    \end{equation}
    
    \item Теоремы о начальном и конечном значении (предельные теоремы)
    
    Теорема о конечном значении очень полезна, так как описывает поведение оригинала на бесконечности с помощью простого соотношения. Это, например, используется для анализа устойчивости траектории динамической системы.
    \begin{equation}
        f(\infty) = \lim_{s \to 0} s F(s)
    \end{equation}
\end{enumerate}


%С помощью него исследуются свойства динамических систем и решаются дифференциальные и интегральные уравления.

%Одной из особенностей преобразования Лапласа, которые предопределили его широкое распространение в научных и инженерных расчётах, является то, что многим соотношениям и операциям над оригиналами соответствуют более простые соотношения над их изображениями. Так, свёртка двух функций сводится в пространстве изображений к операции умножения, а линейные дифференциальные уравнения становятся алгебраическими.