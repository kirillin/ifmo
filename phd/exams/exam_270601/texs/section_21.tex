\subsection{Коэффициенты ошибок по задающему и возмущающему воздействиям и их вычисление. Статизм и астатизм непрерывных систем}
\begin{enumerate}
    \item \textbf{По задающему воздействию}.
    
    Коэффициенты ошибок позволяют вычислять установившиееся значение ошибки относительно задающего воздействия на основе разложения задающего воздействия в бесконечнй ряд по производным задающего воздействия.
    
    ПФ по ошибке
    \begin{equation}
        \Phi_e(s) = \cfrac{E(s)}{G(s)} = \cfrac{1}{1 + W(s)} \quad
        E(s) = \Phi_e(s) G(s)
    \end{equation}
    
    Разложив ошибку в ряд Тейлора в окресности $s=0$
    \begin{equation}
        E(s) = (c_0 + c_1 s + \cfrac{c_2 s^2}{2!} + \dots + \cfrac{c_k s^k}{k!}) G(s)
    \end{equation}
    где $c_i$~--- коэффициенты разложения в ряд Тейлора (коэффициенты ошибок).
    
    \begin{equation}
        e_{\text{уст}} (t) = \sum\limints_{i=0}^{p} \cfrac{c_i}{i!} \cdot \cfrac{d^i g(i)}{d t^i} 
        \quad
        c_0 = [\Phi_e(s)]_{s=0},
        \quad
        \cfrac{c_i}{i!} = [\cfrac{d^i \Phi_e(s)}{d s^i}]_{s=0}
    \end{equation}
    
    \item Статическая система
    
    Если коэффициент ошибок $c_0 \ne 0$, то система статическая отностельно задающего воздействия.
    \begin{enumerate}
        \item Для 
            \begin{equation}
                g(t) = g_0 = const \quad
                e_{\text{уст}} = c_0 g_0
            \end{equation}
            
            Статическая система отрабатывает постоянное воздействие с постоянной ошибкой.

        \item Для
            \begin{equation}
                g(t) = g_1 t 
                \quad
                e_{\text{уст}} = c_0 g_1 t + c_1 g_1
            \end{equation}
            
            Статическая система воздействия с постоянной скоростью в установившемся режиме отрабатывает с бесконечной ошибкой (причем ошибка растет в бесконечность линейно).
    \end{enumerate}

    \item Система с астатизмом относительно задающего воздействия
    
    I порядок астатизма:
    \begin{equation}
        c_0 = 0, \quad c_1 \ne 0
    \end{equation}

    \begin{enumerate}
        \item Постоянное входное воздействие
        \begin{equation}
            g(t) = g_0 = const
            \quad
            e_{\text{уст}} = \underbrace{c_0}_{=0} g_0 \textcolor{gray}{(= 0)}
        \end{equation}
        \item Нарастающее входное воздействие
        \begin{equation}
            g(t) = g_1 t \quad
            e_{\text{уст}} = \underbrace{c_0}_{=0} g_1 t + c_1 g_1 = c_1 g_1
        \end{equation}
        Добротность по скорости системы с астатизмом I порядка [$c^{-1}$]
        \begin{equation}
            \cfrac{1}{c_1} = \cfrac{g_1}{e_{\text{уст}}}
        \end{equation}
    \end{enumerate}

    II порядок астатизма:
    \begin{equation}
        c_0 = 0, \quad c_1 = 0, \quad c_2 \ne 0
    \end{equation}
    
    \begin{enumerate}
        \item
        \begin{equation}
            g(t) = g_0 = const \quad
            e_{\text{уст}} = c_0 g_0 + c_1 \cfrac{dg_0}{dt} + \cfrac{c_2}{2} \cfrac{d^2 g_0}{dt^2} + \dots = c_0 g_0 = 0
        \end{equation}
        \item 
        \begin{equation}
            g(t) = g_1 t, \quad
            e = 0, \quad
            \cfrac{dg(t)}{dt} = g_1, \quad
            \cfrac{d^2g(t)}{dt^2} = 0
        \end{equation}
        \item 
        \begin{equation}
            g(t) = \cfrac{g_2}{2} t^2, \quad
            e_{\text{уст}} = \cfrac{c_2}{2} g_2 \quad\Rightarrow\quad
        \end{equation}
        Добротность по ускорению
        \begin{equation}
            \Rightarrow\quad
            \cfrac{g_2}{e} = \Big(\cfrac{c_2}{2}\Big)^{-1}
        \end{equation}
    \end{enumerate}


\end{enumerate}
