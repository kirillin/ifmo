\subsection{Понятие передаточной функции и передаточной матрицы непрерывных систем}

Переда́точная функция~--- один из способов математического описания динамической системы. Представляет собой дифференциальный оператор, выражающий связь между входом и выходом линейной стационарной системы. Зная входной сигнал системы и передаточную функцию, можно восстановить выходной сигнал.

Например, для линейной модели ВВ одноканального объекта управления:
\begin{equation}
    a(p) y(t) = b(p) u(t),
\end{equation}
передаточкая фукция будет иметь вид:
\begin{equation}
    y(t) = \cfrac{b(p)}{a(p)} u(t) = W(p) u(t),
\end{equation}
где $W(p)$~--- интегралльно--дфифференциальный опрератор или передаточная функция, $a_i, b_i$~--- параметры модели, $n$~--- порядок модели, $r = n - m \ge 1$~--- относительная степень модели. В физически реализуемых системах относительная степень модели не должна быть меньшей единицы.


В случае $u(t)=0$ система называется автономной. $a(p)=0$~--- характеристический полином, где его корни $p_i$~--- полюсы системы. $b(p)=0$~--- характеристический полином правой части, где его корни $p_i^0$~--- нули системы. В случае $a_0 = 1$~--- характеристический полином называется приведенным.

Преимущество использования таких операторных моделей: краткость записи уравнений; удобство преобразования сложных моделей.

Для MIMO-систем вводится понятие \textbf{матричной передаточной функции}. 
Если для системы
\begin{equation}
    \dot x = A x + b u, \quad y = C x
\end{equation}
для нулевых начальных условий и при $p = d/dt$ записать:
\begin{equation}
    x = (p I - A)^{-1} B \cdot u \quad\Rightarrow\quad y = C(p I - A)^{-1} B \cdot u,
\end{equation}
то получим передаточную матрицу системы 
\begin{equation}
    W(p) = C(p I - A)^{-1}.
\end{equation}
где $(p I - A)^{-1}$~--- резольвента, которую можно переписать в виде:
\begin{equation}
    (p I - A)^{-1} = \cfrac{adj(pI - A)}{det(pI - A)} = \cfrac{p^{n-1} + R_1 p^{n-2} + \dots + R_n}{det(pI - A)}
\end{equation}
где характеристический полином системы:
\begin{equation}
    det(pI - A) = a(p)
\end{equation}
где собственные числа матрица $A$ в точности совпадают с полюсами системы $\lambda_i\{A\} = p_i$.

В матрица $W(p)$ на месте $ij$ элемента стоит обычная передаточная функция от $i$-ого входа к $j$-у выходу.

