\subsection{Понятие дискретных по времени объектов и систем, их математические модели}

Математические модели дискретных систем управления описывают поведение этих систем только в квантованные моменты времени: $t_k, k = 0, 1, 2, \dots$ Дискретным представлением непрерывных сигналов $u(t), y(t), х(t)$ являются последовательности: ${u(t_k)}, {y(t_k)}, {х(t_k)}$.
Математические модели дискретных систем устанавливают взаимосвязь между этими последовательностями.
Практически все объекты и процессы управления имеют непрерывный характер своего состояния и динамики развития во времени. Поэтому дискретные автоматические системы управления содержат в своей структуре как цифровую (дискретную), так и аналоговую (непрерывную) части. Для согласования этих частей в системе используются аналогово-цифровые и цифроаналоговые преобразователи (АЦП и ЦАП). АЦП ставит в соответствие непрерывной функции $f(t), t \ge t_0$ последовательность ${f(t_k)}=f(k \Delta t), \Deltat=const, k = 0, 1, 2,\dots$. В свою очередь, ЦАП осуществляет преобразование последовательности ${f_k, k = 0, 1, 2, \dots}$ в некоторую непрерывную функцию, которая является аппроксимацией исходной функции $f(t), t \ge t_0$. Часто используют кусочно-постоянную аппроксимацию, поэтому такой преобразователь называют экстраполятором, или фиксатором нулевого порядка.

Построение дискретного представления непрерывной системы носит название процесса дискретизации, или квантования, непрерывной системы. Пусть непрерывная система представлена своей внешней моделью:
\begin{equation}
    a_0 y^{(n)} + a_1 y^{(n-1)} + \dots + a_n y = u
\end{equation}
При достаточно малом шаге квантования дискретизацию этой модели можно выполнить с необходимой точностью путем замены дифференциалов конечными разностями:
\begin{equation}
    y' = \cfrac{d y(t_k)}{dt} = \cfrac{\Delta y(t_k)}{\Delta t} = \cfrac{y(t_{k+1}) - y(t_k)}{\Delta t}
    y'' = \dots
\end{equation}
После подстановки в дискретная внешняя модель системы принимает конечно-разностный вид, который после алгебраических преобразований переводится в рекуррентную форму с постоянными коэффициентами модели $a_i$ и обозначением $z y(k) = y(k+1), z$~--- оператор сдвига.
\begin{equation}
    a_0 z^{n} + \dots + a_n z = u
\end{equation}

Дискретно представляемые сигналы описываются функциями дискретной переменной. Для описания дискретных систем используются решетчатые функции и разностные уравнения. Решетчатые функции являются аналогами непрерывных функций, описывающих непрерывные системы, а разностные уравнения являются аналогами дифференциальных уравнений.

Дискретная система~--- это система в которой присутствует хотя бы один элемент, производящий квантование сигналов по уровню, по времени либо одновременно по тому и другому.
Такой элемент в системе называется импульсным элементом или модулятором.

\textit{Решетчатой функцией} называется функция, получающаяся в результате замены непрерывной переменной на дискретную независимую переменную, определенную в дискретные моменты времени $kТ, k = 0, 1, 2, \dots$ Непрерывной функции $x(t)$ соответствует решетчатая функция $х(kТ)$, где $Т$~---  период квантования, при этом непрерывная функция является огибающей решетчатой функции. При заданном значении периода квантования $Т$ непрерывной функции $x(t)$ соответствует однозначная решетчатая функция  $х(kТ)$. Однако обратного однозначного соответствия между решетчатой и непрерывной функцией не существует, так как через ординаты решетчатой функции можно провести множество огибающих.
Отсчеты по шкале времени удобно вести в целочисленных единицах периода квантования $Т$. С этой целью вместо переменной $t$ непрерывной функции введем новую переменную $\tau=t/T$, при этом непрерывной функции $x(\tau)$ будет соответствовать решетчатая функция $х(k) = x_k$.

\textit{Теорема Котельникова-Шеннона}. Процедура преобразования сигнала непрерывного времени $x(t)$ к дискретному виду, квантованному по времени, называется квантованием. Такая процедура отражает как реальные процессы, проходящие в цифровых системах управления, так и математические операции, использующиеся в различных сферах теории информации. В результате квантования получается импульсная последовательность $x(kT)$ (решетчатая функция), которая при $t = kT$ совпадает с исходным сигналом:
\begin{equation}
    x(kT) = x(t)|_{t=kT},
\end{equation}
и не определена между отсчетами $k$. Потери информации при квантовании зависят от величины интервала квантования $Т$ (частоты квантования $2\pi/T$).
Выбор интервала $Т$ обычно осуществляется из соображений теоретической возможности точного восстановления исходного сигнала по данной дискретной выборке. Согласно теореме Котельникова-Шеннона, если спектр сигнала $x(t)$ ограничен максимальной частотой $\omega_{max}$, то точное восстановление функции $x(t)$ теоретически возможно при условии, что на одном периоде максимальной частоты в сигнале имеется минимум два дискретных отсчета, т.е. частота квантования $\omega$ должна быть более чем в 2 раза больше наибольшей частоты $\omega_max$ в сигнале:
\begin{equation}
    \omega \ge 2 \omega_{max}, T < \cfrac{\pi}{\omega_{max}}
\end{equation}

\textit{Разностные уравнения}. Связь между значениями решетчатой функции при разных значениях аргумента определяется с помощью конечных разностей, которые являются аналогами производных в дифференциальных уравнениях. 

Разностью первого порядка (первой разностью) называется разность между последующим дискретным значением решетчатой функции и ее текущим значением:
\begin{equation}
    \Delta x(k) = x(k+1) – x(k).
\end{equation}
Разность первого порядка характеризует скорость изменения решетчатой функции и, следовательно, является аналогом первой производной непрерывной функции.
Разность второго порядка определяется как разность двух соседних разностей первого порядка:
Разности любого m-го порядка вычисляются аналогично.
