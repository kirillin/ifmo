\subsection{Дискретное преобразование Лапласа (Z-преобразование) дискретных процессов}

В теории импульсных систем для решения разностных уравнений используется дискретное преобразование Лапласа и его модификация~--- дискретное z-преобразование.

Преобразование лапласа для непрерывной и дискретной функций:
\begin{equation}
    X(s) = \int\limits_{0}^{\infty} x(t) e^{-st} dt, \quad  X(s) = T \sum\limits_{k=0}^{\infty} x(kT) e^{-s kT}
\end{equation}

Дискретное преобразование Лапласа, получается путем введением замены:
\begin{equation}
    z^-k = e^{sT} 
    \quad\Rightarrow\quad
    X(z) = T \sum\limits_{k=0}^{\infty} x(kT) z^{-k}
\end{equation}
$x(kT$~--- оригинал решетчатой функции, $X(z)$~--- его изображение, $z$~--- комплексная переменная.

Из теоремы Коши-Адамара следует, что ряд сходится абсолютно вне круга $|z|>R$, где $R = \lim\limits_{k \rightarrow \infty} (|x(kT)|)^{\frac{1}{k}}$.

Для полубесконечной последовательности $\{x(k)\}$ ($x(k) = 0, k < 0$):
\begin{equation}
    x(0), x(1), x(2), \dots
\end{equation}

Ее можно рассматривать как функцию, аргумент которой применает дискретные значения $0,1,2, \dots$. Такие функции называю решетчатыми функциями.

Z- преобразованием последоватьности называют сумму ряда
\begin{equation}
    X(z) = Z\{x(kT)\} = \sum\limits_{k=0}^{\infty} x(kT) z^{-k}.
\end{equation}


Оно лежит в основе метода решения разностных уравнений. Дискретное преобразование Лапласа $X(z)$ отличается от z-преобразования наличием нормирующего множителя $Т$. При анализе дискретных систем z-преобразование позволяет перейти от разностных уравнений к алгебраическим и существенно упростить анализ динамики дискретных систем.

Для обратного перехода от изображения к оригиналу (для нахождения исходной решетчатой функции по ее изображению) используется обратное z-преобразование:
\begin{equation}
    x(kT) = \cfrac{1}{2 \pi j} \oint X(z) z^{k-1} dz
\end{equation}

Корни $s_i$ характеристического полинома непрерывной системы связаны с корнями $z_i$ характеристического полинома эквивалентной дискретной системы соотношением
\begin{equation}
    z_i = e^{p_iT}
\end{equation}
Взаимно-однозначное соответствие корней непрерывной и эквивалентной дискретной систем выполняется только при интервале дискретизации, удовлетворяющем теореме Котельникова-Шеннона.

Важно, что все существующие изображения имеют общую область сходимости --- круг некоторого радиуса $R_{min}$ в комплексной плоскости $\xi$ и область вне круга радиуса $1/ R_{min}$ в плоскости $z$.

\textcolor{blue}{Свойства z-преобразования}.

\begin{enumerate}
    \item Линейность. Для любых $a, b$
    \begin{equation}
        Z\{a x(k) + b y(k)\} = a X(z) + b Y(z)
    \end{equation}
    
    \item Начальное значение последовательности может быть вычислено как
    \begin{equation}
        x(0) = \lim\limits_{z \rightarrow \infty} X(z)
    \end{equation}
    
    \item Конечное значение. Если функция $(1-z^{-1}) X(z)$ не имеет полюсов в области $|z| \ge 1$ и конечное значение последовательности $\{x(k)\}$ существует, оно может быть вычислено как
    \begin{equation}
        \lim\limits_{k \rightarrow \infty} x(k) = \lim\limits_{z \rightarrow \infty} (1-z^{-1}) X(z)
    \end{equation}
    
    \item Обратные сдвиг.
    
    Последовательность $\{x(k-m)\}$, запаздывающую на $m > 0$ такток по отношению к исходной последовательности
    \begin{equation}
        Z\{x(k-m)\} = z^{-m} X(z)
    \end{equation}
    
    \item Сдвиг вперед
    \begin{equation}
        Z\{x(k+m)\} = z^{m} \Bigg[ X(z) - \sum\limits_{i=0}^{m-1} x(i) z^{-i} \Bigg]        
    \end{equation}
    
    \item Свертка решетчатых функций $\{x(k)\}, \{y(k)\}$
    \begin{equation}
        g(k) = \sum\limits_{m=0}^{k} x(m) y(k-m) = \sum\limits_{m=0}^{k} x(k-m) y(m)
    \end{equation}
    имеет изображение, равное произведению этих функций
    \begin{equation}
        Z\{g(k)\} = X(z) Y(z)
    \end{equation}
\end{enumerate}