\subsection{ Показатели качества переходных процессов непрерывных и дискретных систем, вводимые по 	переходной функции}

Переходный процесс — в теории систем представляет реакцию динамической системы на приложенное к ней внешнее воздействие с момента приложения этого воздействия до некоторого установившегося состояния. Изучение переходных процессов — важный шаг в процессе анализа динамических свойств и качества рассматриваемой системы. Примерами внешнего воздействия могут быть дельта-импульс, скачок или синусоида.

Прямые показатели качества: время переходного процесса и перерегулирование.

ВПП~--- время, необходимое выходному сигналу системы для того, чтобы приблизиться к своему установившемуся значению. Обычно предел такого приближения составляет $\Delta = 1-10$ процентов.
\begin{equation}
    t_{\text{п}} = min \{ T_{\text{п}}: (h(t) - h_{\text{уст.}}) \le \Delta, t \ge T_{\text{п}} \},
    \quad
    \Delta = 0.05 h_{\text{уст}}
\end{equation}

Перерегулирование (определяется величиной первого выброса) — отношение разности максимального значения переходной характеристики и её установившегося значения к величине установившегося значения. Измеряется обычно в процентах.

\begin{equation}
    \sigma = \cfrac{h_{max} - h_{\text{уст}}}{h_{\text{уст}}} \cdot 100 \%
\end{equation}

Все вышеописанное справедливо и для дискретный систем.