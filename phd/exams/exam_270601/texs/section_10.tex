\subsection{Передаточные функции и матрицы дискретных систем и объектов и их вычисление на основе структурного представления}

ПФ~--- отношение дискретного преобразования Лапласа входной переменной к выходной при нулевых начальных условиях:
\begin{equation}
    W(z) = \cfrac{Y(z)}{X(z)}
\end{equation}

Передаточная матрица:
\begin{gather}
    \text{for} z\{x(k)\} = X(z), \quad z\{u(k)\} = U(z), \quad z\{y(k)\} = Y(z), \\
    \begin{cases}
        x(k+1) = A x(k) + B u(k),\\
        y(k) = C x(k),
    \end{cases}
    \quad\Rightarrow\quad
    \begin{cases}
        z X(z) = A X(z) + B U(z),\\
        Y(z) = C X(z),
    \end{cases}
\end{gather}
Отсюда, получим предаточную матрицу:
\begin{gather}
    Y(z) = C(z I - A)^{-1} B \cdot U(z)
    \quad\Rightarrow\quad
    W(z) = C(z I - A)^{-1} B
\end{gather}
где
\begin{gather}
    W(z) = \Big[ W_{ij}(z) \Big], i=\overline{1,n}, j=\overline{1,n}
\end{gather}
где $W_{ij}$~--- связывает i-ый вход с j-ым выходом.

Пример:
\begin{equation}
    Y(z) = \cfrac{W_1 W_2}{1+W_1 W_2} G(z) + \cfrac{W_2}{1+W_1 W_2} F(z),
\end{equation}
где $G(z)$~--- изображение входа, $F(z)$~--- изображение помехи.