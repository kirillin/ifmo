\subsection{Понятие 	устойчивости. Виды устойчивости: 	устойчивость по Ляпунову, асимптотическая 	устойчивость, экспоненциальная устойчивость, качественная экспоненциальная 	устойчивость непрерывных и дискретных 	систем}


\textcolor{blue}{\textbf{Устойчивость} это}~--- способность динамической системы возвращаться в исходное положение равновесия после окончания действия внешних возмущений.

В классической теории устойчивости исследуется не устойчивость системы как таковой, а устойчивость ее так называемого невозмущенного движения. Для линейных систем с точки зрения устойчивости не имеет значения, какое их движение принимается в качестве невозмущенного.

Некоторые справедливые (Бессекерский, c. 119) замечания:
\begin{enumerate}
    \item Устойчивость невозмущенного движения не зависит от того, какое движение
системы принято в качестве невозмущенного.
    \item Певозмущенное движение системы устойчиво, если устойчиво ее свободное
движение.
    \item Устойчивость невозмущенного движения не зависит от вида и характера из-
менения внешних (задающего и возмущающих) воздействий. Этот вывод ба-
зируется на двух предыдущих.
\end{enumerate}

\textit{Теорема Ляпунова}:
Если для уравнений возмущенного движения можно найти знакоопределенную функцию $V$, производная $\dot V$ которой в силу этих уравнений была бы или знакопостоянной функцией противоположного знака с $V$, или тождественно равной нулю, то невозмущенное движение $x=0$ устойчиво.


\textcolor{blue}{\textbf{Виды устойчивости}} (для дискретных все тоже самое только для дискретного времени):
\begin{enumerate}
    \item \textcolor{blue}{\textit{устойчивость по Ляпунову}}.
    
    Система устойчива по Ляпунову, если для любого $\epsilon > 0$, найдется $\delta(\epsilon) > 0$ такое, что для всех начальных условий удовлетворяющих $|x_0| < \delta$ и любого $t>0$, выполняется $|x(t,x_0)| < \epsilon$.
    
    Если это условие не вполняется, то система называется неустойчивой.
    
    Определение предусматривает, что траектории $x(t,x_0)$ начинающиеся в некторой малой $\delta$-окрестности положения равновесия, не покидают заданную $\epsilon$-окрестность.
    
    \item \textcolor{blue}{\textit{асимптотическая устойчивость}}.
    
    Система называется асимптотически устойчивой, если
    \begin{enumerate}
        \item она устойчива по Ляпунову;
        \item выполняется условие аттрактивности (притяжения) положения равновесия:
        \begin{equation}
            \lim\limits_{t \rightarrow \infty} x(t, x_0) = 0
        \end{equation}
    \end{enumerate}
    Из асимптотической устойчивости следует устойчивость по Ляпунову.
    
    Определение предусматривает, что положение равновесия обладает притягивающими свойствами, т.е. с течением времени приближвается к положению равновесия $x^*=0$. Но определить время схождения нельзя.
    
    \textcolor{green}{
        \textbf{Дискретная система}
        \begin{equation}
            \begin{cases}
                x(k+1) = A x(k)\\
                y(k) = C x(k)
            \end{cases}
        \end{equation}
        асимптотическая устойчива, если $\lim\limits_{k \rightarrow \infty} |x(k)| = 0$.
    }
    
    
    \item \textcolor{blue}{\textit{экспоненциальная устойчивость}}.
    
    Система экспоненциально устойчива, если найдутся таке числа $c>0, \alpha>0$, что для любых $t \ge 0$, вполняется:
    \begin{equation}
        x(t, x_0) \le c e^{-\alpha t} |x_0|.
    \end{equation}
    
    Отсюда видно, что все траектории, начинающиеся в произвольной $\Delta_0$-окрестности, т.е. $|x_0| \le \Delta_0$, экспоненциально затухают~--- находятся в каждый момент времени $t$ в пределах сужающейся области:
    \begin{equation}
        |x| < x_m(t) = c \Delta_o e^{-\alpha t}.
    \end{equation}
    Фунция $x_m(t)$ ограничивающая сверху текущие значения нормы вектора состояния, называется \textit{мажорантой}.
        
    При определенных условиях экспоненциальная и асимптотическая устойчивости эквивалентны.
            
    \item \textcolor{blue}{\textit{качественная экспоненциальная устойчивость}}.
    
    Система КЭ устойчива для любых траеткорий движения системы и начальных условий, найдется такие $r>0, \beta: \beta + r < 0$, что для любых $t \ge 0$ выполняется неравенство:
    \begin{equation}
        || x(t) - e^{\beta t} x(0) || \le \rho (e^{(\beta - r) t} -e^{\beta t}) ||x(0)||,
    \end{equation}
    где $\rho > 0$.
    
    

\end{enumerate}