\documentclass[a4paper,14pt]{extreport}
\usepackage[utf8]{inputenc}
\usepackage[T2A]{fontenc}
\usepackage[russian]{babel}
\usepackage{eufrak}
% поля:
\usepackage[left=2.5cm, right=1.5cm, top=2cm, bottom=2cm]{geometry}
\linespread{1}
\usepackage{indentfirst} % отделять первую строку раздела абзацным отступом
\setlength\parindent{5ex}
\addto{\captionsrussian}{\renewcommand*{\contentsname}{Содержание}}
\usepackage[hidelinks]{hyperref} % гиперссылки в содержании
\usepackage{graphicx}
\usepackage{float}
\usepackage{amsmath}
\renewcommand*{\thesection}{\arabic{section}}

\usepackage{multirow}
\usepackage[normalem]{ulem}
\useunder{\uline}{\ul}{}

\usepackage{cmap}%позволяет копировать кириллицу из скомпилированного файла

% Глубина разделов, попадающих в содержание
\setcounter{tocdepth}{3}

\linespread{1.3} % настройка межстрочного интервала
\tolerance=1000 % настройка чувствительности вставки переносов
\hfuzz=0pt
\sloppy

\begin{document}
	
\section*{Введение}

Область компьютерного зрения может быть охарактеризована как молодая, разнообразная и динамично развивающаяся. И хотя существуют более ранние работы, можно сказать, что только с конца 1970-х годов началось интенсивное изучение этой проблемы, когда производительность компьютером стала достаточной, чтобы управлять обработкой больших наборов данных, таких как изображения. Сейчас, когда мощность компьютеров выросла на порядки, становится возможным использование гораздо более сложных систем компьютерного зрения, в качествве источников информации для  которых могут использоваться одновременно большое количество видеокамер, инфракрасных датчиков различных диапазонов, радары, лазерные датчики и ряд других. 

Так как одной из целей создания автономных роботов является создание подобного человеку механизма, способного работать в условиях не совместимых с жизнью для человека и не приносящих существенного урона механизму, то актуальной проблемой становится реализация органов чувств для получения и предварительной обработки информации поступающей из окружающей среды, необходимой для эффективного выполнения поставленной задачи. Таким образом, системы технического зрения  находят своё применение в космических, авиационных, наземных, надводных и подводных мобильных средствах, т.е. там, где необходим анализ внешней обстановки в режиме реального времени.

Основные цели, которые преследует техничегое зрение:
\begin{itemize}
	\item возможность автоматизировать любой процесс обработки визуальных данных;
	\item исключить человеческий фактор из производственных процессов;
	\item повысить скорость обработки информации.
\end{itemize} 

	
\section{Области применения}
Существует масса методов для решения различных строго определённых задач компьютерного зрения, где методы часто зависят от задач и редко могут быть обобщены для широкого круга применения. Многие из методов и приложений все ещё находятся в стадии фундаментальных исследований, но всё большее число методов находит применение в коммерческих продуктах, где они часто составляют часть большей системы, которая может решать сложные задачи (например, в области медицинских изображений или измерения и контроля качества в процессах изготовления). 

В большинстве практических применений компьютерного зрения компьютеры предварительно запрограммированы для решения отдельных задач, но методы, основанные на знаниях, становятся всё более общими. 

Важную часть в области искусственного интеллекта занимает автоматическое планирование или принятие решений в системах, которые могут выполнять механические действия, такие как перемещение робота через некоторую среду. Этот тип обработки обычно нуждается во входных данных, предоставляемых системами компьютерного зрения, действующими как видеосенсор и предоставляющими высокоуровневую информацию о среде и роботе. Другие области, которые иногда описываются как принадлежащие к искусственному интеллекту и которые используются относительно компьютерного зрения, это распознавание образов и обучающие методы. В результате, компьютерное зрение иногда рассматривается как часть области искусственного интеллекта или области компьютерных наук вообще.

Ещё одной областью, связанной с компьютерным зрением, является обработка сигналов. Многие методы обработки одномерных сигналов, обычно временных сигналов, могут быть естественным путём расширены для обработки двумерных или многомерных сигналов в компьютерном зрении. Однако из-за своеобразной природы изображений существует много методов, разработанных в области компьютерного зрения и не имеющих аналогов в области обработки одномерных сигналов. Особым свойством этих методов является их нелинейность, что, вместе с многомерностью сигнала, делает соответствующую подобласть в обработке сигналов частью области компьютерного зрения.

Помимо упомянутых подходов к проблеме компьютерного зрения, многие из исследуемых вопросов могут быть изучены с чисто математической точки зрения. Например, многие методы основываются на статистике, методах оптимизации или геометрии. Наконец, большие работы ведутся в области практического применения компьютерного зрения — того, как существующие методы могут быть реализованы программно и аппаратно или как они могут быть изменены с тем, чтобы достичь высокой скорости работы без существенного увеличения потребляемых ресурсов.

Одними из новых областей применения являются автономные транспортные средства, включая подводные, наземные (роботы, машины), воздушные. Уровень автономности изменяется от полностью автономных (беспилотных) до транспортных средств, где системы, основанные на компьютерном зрении, поддерживают водителя или пилота в различных ситуациях. Полностью автономные транспортные средства используют компьютерное зрение для навигации, то есть для получения информации о месте своего нахождения, для создания карты окружающей обстановки, для обнаружения препятствий. Они также могут быть использованы для определённых задач, например, для обнаружения лесных пожаров. Примерами таких систем могут быть система предупредительной сигнализации о препятствиях на машинах и системы автономной посадки самолетов. Некоторые производители машин демонстрировали системы автономного управления автомобилем, но эта технология все ещё не достигла того уровня, когда её можно запустить в массовое производство.

\section{Типичные задачи технического зрения}

Каждая из областей применения компьютерного зрения, описанных выше, связана с рядом задач; более или менее хорошо определённые проблемы измерения или обработки могут быть решены с использованием множества методов. Некоторые примеры типичных задач компьютерного зрения представлены ниже.

\begin{itemize}
	\item Распозование
	
	Классическая задача в компьютерном зрении, обработке изображений и машинном зрении это определение содержат ли видеоданные некоторый характерный объект, особенность или активность. Эта задача может быть достоверно и легко решена человеком, но до сих пор не решена удовлетворительно в компьютерном зрении в общем случае: случайные объекты в случайных ситуациях.
	
	Существующие методы решения этой задачи эффективны только для отдельных объектов, таких как простые геометрические объекты (например, многогранники), человеческие лица, печатные или рукописные символы, автомобили и только в определённых условиях, обычно это определённое освещение, фон и положение объекта относительно камеры.
	
	В литературе описано различное множество проблем распознавания:
	\begin{itemize}
		\item Распознавание: один или несколько предварительно заданных или изученных объектов или классов объектов могут быть распознаны, обычно вместе с их двухмерным положением на изображении или трехмерным положением в сцене.
		
		\item Идентификация: распознается индивидуальный экземпляр объекта. Примеры: идентификация определённого человеческого лица или отпечатка пальцев или автомобиля.
		
		\item Обнаружение: видеоданные проверяются на наличие определённого условия. Например, обнаружение возможных неправильных клеток или тканей в медицинских изображениях. 
		
		\item Обнаружение, основанное на относительно простых и быстрых вычислениях иногда используется для нахождения небольших участков в анализируемом изображении, которые затем анализируются с помощью приемов, более требовательных к ресурсам, для получения правильной интерпретации.
	\end{itemize}
	Существует несколько специализированных задач, основанных на распознавании, например:
	\begin{itemize}
	\item Поиск изображений по содержанию: нахождение всех изображений в большом наборе изображений, которые имеют определённое содержание. Содержание может быть определено различными путями, например в терминах схожести с конкретным изображением (найдите мне все изображения похожие на данное изображение), или в терминах высокоуровневых критериев поиска, вводимых как текстовые данные (найдите мне все изображения, на которых изображено много домов, которые сделаны зимой и на которых нет машин).
	\item Оценка положения: определение положения или ориентации определённого объекта относительно камеры. Примером применения этой техники может быть содействие руке робота в извлечении объектов с ленты конвейера на линии сборки.
	\item Оптическое распознавание знаков: распознавание символов на изображениях печатного или рукописного текста, обычно для перевода в текстовый формат, наиболее удобный для редактирования или индексации (например, ASCII).	
	\end{itemize}	
	
	\item Движение
		
	Несколько задач, связанных с оценкой движения, в которых последовательность изображений (видеоданные) обрабатываются для нахождения оценки скорости каждой точки изображения или 3D сцены. Примерами таких задач являются:
	\begin{itemize}
		\item Определение трехмерного движения камеры
		\item Слежение, то есть следование за перемещениями объекта (например, машин или людей)
		
	\end{itemize}
	\item Восстановление сцены
	
	Даны два или больше изображения сцены, или видеоданные. Восстановление сцены имеет задачей воссоздать трехмерную модель сцены. В простейшем случае, моделью может быть набор точек трехмерного пространства. Более сложные методы воспроизводят полную трехмерную модель.
	
	\item Восстановление изображений
	
	Задача восстановления изображений это удаление шума (шум датчика, размытость движущегося объекта и т. д.). Наиболее простым подходом к решению этой задачи являются различные типы фильтров, таких как фильтры нижних или средних частот. Более сложные методы используют представления того, как должны выглядеть те или иные участки изображения, и на основе этого их изменение.
	
	Более высокий уровень удаления шумов достигается в ходе первоначального анализа видеоданных на наличие различных структур, таких как линии или границы, а затем управления процессом фильтрации на основе этих данных.
		
\end{itemize}


\section{Методы технического зрения}

Реализация систем компьютерного зрения сильно зависит от области их применения, аппаратной платформы и требований по производительности. Некоторые системы являются автономными и решают специфические проблемы детектирования и измерения, тогда как другие системы составляют подсистемы более крупных систем, которые уже могут содержать подсистемы контроля механических манипуляторов (роботы), информационные базы данных (поиск похожих изображений), интерфейсы человек-машина (компьютерные игры) и т. д. Однако, существуют функции, типичные для многих систем компьютерного зрения.

\begin{itemize}
	\item Получение изображений: цифровые изображения получаются от одного или нескольких датчиков изображения, которые помимо различных типов светочувствительных камер включают датчики расстояния, радары, ультразвуковые камеры и т. д. В зависимости от типа датчика, получающиеся данные могут быть обычным 2D изображением, 3D изображением или последовательностью изображений. Значения пикселей обычно соответствуют интенсивности света в одной или нескольких спектральных полосах (цветные или изображения в оттенках серого), но могут быть связаны с различными физическими измерениями, такими как глубина, поглощение или отражение звуковых или электромагнитных волн, или ядерным магнитным резонансом.
	\item Предварительная обработка: перед тем, как методы компьютерного зрения могут быть применены к видеоданным с тем, чтобы извлечь определённую долю информации, необходимо обработать видеоданные, с тем чтобы они удовлетворяли некоторым условиям, в зависимости от используемого метода. Примерами являются:
	\begin{itemize}
		\item Повторная выборка с тем, чтобы убедиться, что координатная система изображения верна
		\item Удаление шума с тем, чтобы удалить искажения, вносимые датчиком
		\item Улучшение контрастности, для того, чтобы нужная информация могла быть обнаружена
		\item Масштабирование для лучшего различения структур на изображении
	\end{itemize}
	\item Выделение деталей: детали изображения различного уровня сложности выделяются из видеоданных. Типичными примерами таких деталей являются:
	\begin{itemize}
		\item Линии, границы и кромки
		\item Локализованные точки интереса, такие как углы, капли или точки: более сложные детали могут относиться к структуре, форме или движению.
	\end{itemize}
	\item Детектирование/Сегментация: на определённом этапе обработки принимается решение о том, какие точки или участки изображения являются важными для дальнейшей обработки. Примерами являются:
	\begin{itemize}
		\item Выделение определённого набора интересующих точек
		\item Сегментация одного или нескольких участков изображения, которые содержат характерный объект
	\end{itemize}
	\item Высокоуровневая обработка: на этом шаге входные данные обычно представляют небольшой набор данных, например набор точек или участок изображения, в котором предположительно находится определённый объект. Примерами являются:
	\begin{itemize}
		\item Проверка того, что данные удовлетворяют условиям, зависящим от метода и применения
		\item Оценка характерных параметров, таких как положение или размер объекта
		\item Классификация обнаруженного объекта по различным категориям
	\end{itemize}

	\item Программное обеспечение
	
	\begin{itemize}
		\item OpenCV
		
		Представляет собой библиотекц алгоритмов компьютерного зрения, обработки изображений и численных алгоритмов общего назначения с открытым кодом. Реализована на C/C++, также разрабатывается для Python, Java, Ruby, Matlab, Lua и других языков[2]. Может свободно использоваться в академических и коммерческих целях — распространяется в условиях лицензии BSD.
		
		Включает в себя модули реализующие базовые структуры, вычисления (математические функции, генераторы случайных чисел) и линейную алгебру, обработку изображений (фильтрация, геометрические преобразования, преобразование цветовых пространств и т. д.), ввод/вывод изображений и видео, модели машинного обучения (SVM, деревья решений, обучение со стимулированием и т. д.),  распознавание и описание плоских примитивов, анализ движения и отслеживание объектов (оптический поток, шаблоны движения, устранение фона),  калибровка камеры, поиск стерео-соответствия и элементы обработки трёхмерных данных, обнаружение объектов на изображении и многое другое.
		
		\item PCL
		
		 Всесторонняя свободная библиотека для n-мерных облаков точек и трёхмерной обработки геометрии.
		 
		 
	\end{itemize}

\end{itemize}



\section{Заключение}
Эта работа посвящена проблематике технического зрения. Рассматривались в основном наиболее общие методические и теоретические аспекты. Большинство упомянутых в
данной работе методов могут применяться не только в системах управления, но и при решении более широкого круга задач.
За последние годы было создано значительное число систем
технического зрения в таких областях как бесконтактные измерения и технический контроль, биометрические системы персональной идентификации, автоматизированное управление наземными транспортными средствами, автоматизация учёта и документооборота с использованием цифробуквенной и штрихкодовой машиночитаемой информации, медицинские системы анализа цифровых изображений (радиологических, микроскопических), системы интеллектуального видеонаблюдения и ряд других.

Разработка методов обра­ботки трехмерной зрительной информации в роботизированных и автоматизированных системах в настоящее время задача актуальная, так как такие факторы, как стоимость, скорость, сложность вычислений, трудность реализации алгоритмов делают неприемлемыми многие уже существующие методы.





\section{Литература}
Л. Шапиро, Дж. Стокман. Компьютерное зрение = Computer Vision. — М.: Бином. Лаборатория знаний, 2006. — 752 с. — ISBN 5-94774-384-1.

Дэвид Форсайт, Жан Понс. Компьютерное зрение. Современный подход = Computer Vision: A Modern Approach. — М.: «Вильямс», 2004. — 928 с. — ISBN 5-8459-0542-7.

А.А. Лукьяница ,А.Г. Шишкин. Цифровая обработка видеоизображений. — М.: «Ай-Эс-Эс Пресс», 2009. — 518 с. — ISBN 978-5-9901899-1-1.

Желтов С.Ю. и др. Обработка и анализ изображений в задачах машинного зрения. — М.: Физматкнига, 2010. — 672 с. — ISBN 978-5-89155-201-2.	
\end{document}

