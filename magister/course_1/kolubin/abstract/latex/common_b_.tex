\section{Общие методы решения задач}
\subsection{Динамические модели}

Формирование эффективных уравнений динамики роботов, которые могут быть рассчитаны на ЭВМ за минимальное время, является одной из важнейших задач в робототехнике. Ее решение необходимо для моделирования динамики манипуляторов в масштабе реального времени, для разработки эффективных алгоритмов управления роботами с учетом динамики, для повышения эффективности исследования и разработки манипуляторов.

Одни из первых результатов в этой области принадлежат Кейну и Виттенбургу. Полученные ими уравнения справедливы не только для роботов, но и для более широкого класса систем, состоящих из шарнирно связанных твердых тел. В дальнейшем было разработано большое количество алгоритмов формирования динамических уравнений манипуляторов, в которых использовались различные способы описания кинематики, расчета кинематических и динамических величин, а также различные формы уравнений динамики системы тел.

Описание кинематики – это способ задания систем координат, связанных со звеньями манипулятора, и выбора параметров, которые однозначно определяют взаимное положение звеньев и конфигурацию всего манипулятора. В представлении Денавита-Хартенберга начала систем координат расположены в шарнирах, а их оси формируются по правилам, которые определяются кинематикой манипулятора. В другом методе описания кинематики локальные системы координат привязаны к центрам масс звеньев, а их оси направлены вдоль главных осей инерции. Параметры, определяемые относительно таких систем координат удобны для динамического анализа.

Еще одной характеристикой методов математического моделирования манипуляторов является способ расчета кинематических и динамических величин, определяющих математическую модель манипулятора. Для этого используются однородные координаты и матрицы преобразования координат размерности 4x4, определяющие относительное положение и ориентацию звеньев манипулятора; матрицы поворотов размерности 3x3 и вектора относительных перемещений; формулы Родриго; ортогональные тензоры; кватернионы; метод векторных параметров с использованием групп Ли .

Хотя вычислительная эффективность того или иного метода формирования динамических уравнений зависит в первую очередь от особенностей его реализации (использования рекурсивных преобразований, динамических аналогий и др.), можно отметить и существенную роль выбора подходящего способа расчета модели манипулятора. Например, матрицы преобразования однородных координат размерности 4x4, обладающие универсальностью в кинематическом описании, практически не используются в задачах реального времени из-за больших вычислительных затрат, необходимых для выполнения операций над ними. В то же время, использование матриц поворотов размера 3x3 позволяет получить эффективные алгоритмы расчета кинематики и динамики. 

Эффективно использование кватернионов, ортогональных тензоров, однако в ряде задач (например, при управлении в декартовых координатах) предпочтительнее использовать матричные представления.

Среди самых современных методов моделирования динамики манипуляторов можно отметить подходы, основанные на использовании нейронных  сетей, пространственных операторов, групп Ли, методов нечеткой логики. Для описания динамики сложных структур (параллельных роботов, манипуляторов с большой избыточностью степеней подвижности, роботов-гуманоидов) используются методы расчета динамики в операционном пространстве роботов и другие.

\subsection{Уравнения движения}

Есть две основные формы записи уравнений движения:

\begin{enumerate}
	\item уравнение в конфигурационном пространстве:
\begin{center}
	$M(q) \ddot q + C(q, \dot q)\dot q + G (q) = \tau$	
\end{center}
где $q, \dot q$ и $\ddot q$ -- положение, скорость и ускорение звена робота, $\tau$ -- силы/моменты приложенные к звену, $M(q)$ -- тензор инерции, $C(q, \dot q$ -- матрица Колиолисовых и центробежных сил, $G(q)$ -- вектор гравитации;

	\item уравнение в операционном пространстве:
\begin{center}
	$\Lambda(x) \ddot x + \mu (x, \dot x) + \rho (x) = f$
\end{center}		
где $x$ -- положение схвата, $\dot x$ -- скорость схвата, $f$ -- действующие силы на схват, $\Lambda(x)$ -- матрица инерции в операционном пространстве, $\mu(x, \dot x)$ -- содержит Кориолисовы и центробежные составляющие, $\rho(x)$ -- вектор гравитации.

\end{enumerate}

Эти уравнения показывают функциональную зависимость: $H$ - функция от $q$, $\Lambda$ функция от $x$. Однако, эти зависимости часто отбрасывают, для более простого понимания. 

Строго говоря, коэффициенты уравнений зависят не только от $q, \dot q$ и $f_{ext}$, но также от динамической модели робототехнической системы.
Т.е. описание механизма отдельно для каждой из его частей: звенья, сочленения и их характеризующие их параметры. Динамическая модель состоит из:

\begin{itemize}
\item кинематической модели;
\item множества параметров инерции.
\end{itemize}

Для описание инерции твердого тела нужно десять параметров инерции: масса, расположение центра масс, шесть вращателльных параметров инерции. Однако, когда тела соединяют друг с другом их степени свободы ограничиваются и некоторые из этих параметров могут не оказывать влияния на поведение системы. Поэтому при составлении динамической модели робототехнического механизма от кинематической модели и измерения ее динамических характеристик, процедура сводится к идентификации значений множества основных параметров инерции, которые поддаются наблюдению в полученной системе.


\subsection{Алгоритмы}

Существует четыре главных задачи динамики, которые решаются путем реализации описанных ниже алгоритмов. 
\begin{enumerate}
\item обратная задача динамики, когда требуется вычислить необходимые моменты/силы приводов звеньев для достижения заданной траектории (положение, скорость и ускорение);
\item прямая задача динамики, когда необходимо определить траекторию движения по заданным моментам/силам приводов звеньев;
\item определение матрицы инерции в конфигурационном пространстве;
\item определение матрицы инерции в операционном пространстве.
\end{enumerate}

Обратная задача решается для непосредственного управления роботом. Обратная -- для моделирования. Матрицы инерции находятся для анализа и управления.

Далее рассмотрены алгоритмы для решения этих задач. Сначала представлены алгоритмы основанные на уравнениях Эйлера-Лагранжа, после на методе Ньюона-Эйлера и Аппеля.

\subsubsection{Алгоритм Uiker и Kahn}
Первая версия алгоритма была разработана в 1965 году J.J. Uicker и в основном использовалась для замкнутых кинематических цепей. В 1969 году M. Kahn доработал его таким образом, что стало возможным использовать его и для разомкнутых кинематических цепей. Также в алгоритм были внесены доработки в 1971 и 1981 годах, тогда N. Orleandea и T. Berea написали программную реализацию это метода в форме программного пакета для анализа динамики роботов. В дальнейшем различные модификации численных и символьных вычислений производили S.Mahil, M. Renaud, M. Tomas и Tesar, R. Waters, J. Hollerbach и многие другие.

Используя этот алгоритм можно решать как прямую, так и обратную задачи динамики. Также он позволяет вычислять матрицы динамической модели робота: матрицу инерции, матрицу Кориолисовых и центробежных сил и вектор гравитации. Уравнения динамики для робота с $n$ степенями свободы, соответствующие рассматриваемому алгоритму, представлены ниже.

\begin{align}\label{eq1}
P_i &= \sum_{j=i}^{n}
\left(
\sum_{k=1}^{j}
\left[
tr(\frac{\partial W_j}{\partial q_i} J_j \frac{\partial W_j^T}{\partial q_k})
\right] \ddot q_k 
\right.&\\
&+
\left.
\sum_{k=1}^{j} \sum_{l=1}^{j} 
\left[
tr(\frac{\partial W_j}{\partial q_i} J_j \frac{\partial^2 W_j^T}{\partial q_k \partial q_l})\dot q_k \dot q_l
\right]
-m_j \vec{q^T} \frac{\partial W_j}{\partial q_i} r_{jo}
\right)
\end{align}
где $P_i$ -- сила/момент привода, $W_i$ -- матрица трансформации от базовой к локальной системе координат $i$-ого звена, $J_i$ -- матрица инерции $i$-ого звена относительно локальной системы координат, $m_i$ -- масса звена $i$, $r_jo$ -- вектор от центра масс звена $i$ к началу базовой системы координат, выраженный в локальной системе координат звена $i$, $\vec g$ --вектор гравитации.

Матрица $W_i$ выражается, как:
\begin{equation}
W_i = A_1^0 A_2^1 ... A_i^{i-1}
\end{equation}
где $A_k^{k-1}$ -- матрица трансформации ($4 \times 4$) между системами координат $i$ и $i-1$.

	\begin{figure}[H]
	\center\includegraphics[width=0.8\linewidth]{1.png}
	\caption{Оси координат двух соседних звеньев}
	\label{fig:scr1}
	\end{figure}

Применение матриц однородных преобразований позволяет использовать одно уравнение и для призматических сочленений, и для вращательных. Системы координат в роботе выбираются в соответствии с конвенцией Денавита-Хартенберга. Таким образом, матрица трансформации из $i-1$ системы координат в $i$, представляет собой:
\begin{equation}
A_i^{i-1} = 
\begin{bmatrix}
1 & 0 & 0 & 0\\
a_i cos (q_i) & cos (q_i) & -sin (q_i) cos (a_i) & sin (q_i) sin (a_i)\\
a_i sin (q_i) & sin (q_i) & cos (q_i) cos (a_i) & -cos (q_i) sin (a_i)\\
s_i & 0 & sin (a_i) & cos (a_i)
\end{bmatrix}
\end{equation}

Оси локальной системы координат звена i представляются единичными векторами ($\vec x_i, \vec y_i, \vec z_i$). Эти системы координат зависят только от геометрических параметров системы, следовательно не совпадаются с главными осями инерции. Таким образом, вместо трех моментов инерции, получается тензор инерции. Uicker использовал тезор инерции $J_i$ размерности $4 \times 4$, который включает в себя и моментты инерции и массы звеньев:


\begin{equation}
\begingroup\makeatletter\def\f@size{14}\check@mathfonts
J_i = 
\begin{bmatrix}
m_i & m_i x_{i0} & m_i y_{i0} & m_i z_{i0}\\
m_i x_{i0} & \frac{1}{2} (-I_{xx} + I_{yy} + I_{zz}) & I_{xy} & I_{xz}\\
m_i y_{i0} & I_{xy} & \frac{1}{2} (I_{xx} - I_{yy} + I_{zz}) & I_{yx} \\
m_i z_{i0} & I_{xz} & I_{yz} & \frac{1}{2} (I_{xx} + I_{yy} - I_{zz})
\end{bmatrix}
\endgroup
\end{equation}
где $x_{i0}, y_{i0}, z_{i0}$ -- координаты центра масс звена $i$ выраженные в базовой системе координат, $I_{xx}, I_{xy},...$ -- моменты инерции вокруг соответствующих осей.

Матрица инерции $J_i$ описывает распределение массы в звене $i$, а также зависит от положения центра масс относительно закрепленной системы координат.

Из матрицы трансформации $A_i^{i-1}$ мы видим, что ее частная производная относительно $q_i$ может быть определена аналитически. Тоже самое справедливо и для матрицы $W_i$. В этом случае нет рекурсивных отношений, поэтому метод Uicker принадлежит к классу нерекурсивных методов.

Теперь, когда мы выяснили общую структуру метода, детальнее рассмотрим как решаются прямая и обратная задачи динамики.

Обозначим $\dot q_i$ и $\ddot q_i$, как скорость и ускорение и запишем уравнение в следующей форме:
\begin{equation}\label{eq2}
P_i = \sum_{j=1}^{n}
H_{ij} (q) \ddot q_j
+
\sum_{k=1}^{n} \sum_{l=1}^{n}
C_{kl}^i (q) \dot q_k \dot q_l + g_i (q)
\end{equation}
где $H_{ij}, C_{kl}^i$ и $g_i$ -- скалярные функции зависящие от вектора обобщенных координат $q = [q_1,...,q_n]^T$. Что бы доказать это, заметим, что в соответствии с уравнением ~\ref{eq1}, эти функции зависят от частных производных матриц $W_i$. В свою очередь матрицы $W_i$ это функции обобщенных координат, из чего следует, что  частные производные матриц $W_i$ также зависят только от обобщенных координат. Теперь, представим модель ~\ref{eq2} в матричной форме:
\begin{equation}\label{eq3}
P = H(q) \ddot q + \dot q^T C(q) \dot q + g(q)
\end{equation}
где $P = [P_1,...,P_n]^T$ -- вектор управляющих сил/моментов, $H(q)$ -- матрица $n\times n$ инерции системы, $C(q)$ -- матрица Кориолисовых и центробежных сил, $g(q)$ -- $n$-вектор соответствующий действию гравитации на систему.

В модели ~\ref{eq3} $\dot q^T C(q) \dot q$ это $n$-вектор:
\begin{equation}
\dot q^T C(q) \dot q = 
\begin{bmatrix}
\dot q^T C^1(q) \dot q\\
\vdots\\
\dot q^T C^n(q) \dot q\\
\end{bmatrix}
\end{equation}
где $C^i(q)$ матрица $n \times n$, элементы которой соответствуют $C_{kl}^i (q)$ в уравнении \ref{eq2}. 

Итак, имея уравнение \ref{eq3} мы можем решить обратную задачу динамики, т.е. по заданным ($q, \dot q, \ddot q$) определить управляющие силы/моменты.

Чтобы решить прямую задачу, нужно определить $q(t)$ по известным моментам $P(t)$ для $t > t_0$ и начальными условиями $q(t_0)$ и $\dot q(t_0)$. Для это представим уравнение \ref{eq3} в следующей форме:
\begin{equation}
P = H(q) \ddot q + h(q, \dot q)
\end{equation}
где $h(q, \dot q) = \dot q^T C(q) \dot q + g(q)$.

Отсюда, выражая $\ddot q$ получим:
\begin{equation}
\ddot q = H(q)^{-1} (P - h(q, \dot q))
\end{equation}

Отсюда $q(t)$ находится интегрированием полученного выражения.

Вычислительная сложность рассмотренного метода составляет $O(n^4)$. Такой алгоритм не подходит для расчета динамики современных роботов в реальном времени. Уравнение \ref{eq1} на 16-битном микропроцессоре вычисляется за 500 мс, для манипулятора с 3-мя степенями свободы и более 5 секунд для 6-степенного. Его можно несколько ускорить пренебрегая Кориолисовыми и центробежными скоростями на медленных режимах работы.

\subsubsection{Алгоритмы S. Mahil, S. Megahed и M. Renaud}
Эти алгоритмы представляет собой модификации Uicker-Kahn алгоритма, в которых авторы избавились от частной производной в уравнении и сократили количество арифметических операций.

В 1979 S.Mahil предложил заменить матрицы трансформации Денавита-Хертенберга формулой поворота Родрига. Чтобы описать суть этой формулы, представим вращение $i$ звена относительно $i-1$ на угол $q_i$. Введем вектор $\vec r_i$ прикрепленный к $i$ звену перед началом вращения. После вращения, в соответствии с формулой Родрига, этот вектор можно выразить, как:
\begin{equation}
\vec r_i = \vec r_i cos (q_i) + (1 - cos(q_i))(\vec e_i \vec r_i) \vec e_i + \vec e_i \times \vec r)i sin (q_i)
\end{equation}
где $\vec e_i$ -- единичный вектор на оси прикрепленной к сочленению $i$.
 
Применим эту формулу для выражения матрицы трансформации $A_i^{i-1}$, которая описывает вращение $i$ системы координат относительно $i-1$. Заметим, что вектор $\vec q_{ij}, j=1,2,3$ -- единичные векторы осей $i$ системы. Тогда, матрица $A_i^{i-1}$:
\begin{equation}
A_i^{i-1} = 
\begin{bmatrix}
\vec q_{i1} & \vec q_{i2} & \vec q_{i3}
\end{bmatrix}
\end{equation}

Теперь, выразим вектор $\vec q_{ij}$, через формулу Родрига:
\begin{equation}
\vec q_{ij} = \vec q_{ij}^0 cos(q_i) + (1-cos(q_i)) (\vec e_{i} \times \vec q_{ij}^0) 
+ \vec e_{i} \times \vec q_{ij}^0 sin(q_i)
\end{equation}
где $\vec q_{ij}^0$ -- вектор оси $j$ $i$-ой системы координат перед вращением.

	\begin{figure}[H]
	\center\includegraphics[width=0.8\linewidth]{2.png}
	\caption{Вращение звена $i$}
	\label{fig:scr2}
	\end{figure}

В отличие от кинематики Денавита-Хартенберга, формула Родрига применяется к локальным системам координат, которые прикреплены к звеньям в произвольных местах.
Наиболее удачным расположением обычно является центр масс звена, тогда системы координат вращения и инерции совпадают, что упрощает уравнение динамики, а также тензор инерции состоит только из главных моментов инерции звена.

В этом алгоритме фигурирует только два момента инерции -- продольный и поперечный.
Такое приближение возможно только для цилиндрических звеньев, длина которых много больше их диаметра. Это накладывает ограничения на применение этого алгоритма.

Алгоритм выводится из уравнения Лагранжа в матричной форме:
\begin{equation}\label{eq4}
P = \sum_{i=1}^{n} 
\left(
H^i (q) \ddot q + D^i (q, \dot q) \dot q + g^i (q)
\right)
\end{equation}
где $P$ -- вектор прикладываемых моментов/сил, $H^i (q), D^i (q, \dot q)$ -- $n \times n$ матрицы и $g^i (q)$ -- $n$-вектор.

Элементы матриц в уравнении \ref{eq4} могут быть выражены с использованием осей сочленений, локальных систем координат и векторов переноса, которые получаются применением формулы Родрига. Для вращательных сочленений:
\begin{align*}
H_{jk}^i = 
(Jn_i + m_i (\vec r_{ji} \times \vec r_{ki})) (\vec e_j \times \vec e_k)
&+ Js_i (\vec e_j \times \vec q_{i1}) (\vec e_k \times \vec q_{i1})
\\&- (\vec e_j  \times \vec q_{ki}) (\vec e_k \times \vec q_{ki})
\end{align*}
где $(j \le k \le i)$, $Jn_i$ -- поперечный момент инерции звена $i$, $Js_i$ -- продольный момент инерции звена $i$, $m_i$ -- масса звена $i$, $\vec r_{ij}$ -- вектор от $j$-ого сочленения к центру масс $i$-ого звена, в инерциальной системе.

Элементы матриц $D^i (q, \dot q)$ выражаются в следующей форме:
\begin{equation}
D_{kl}^i = 
\sum_{j=p+1}^{i}
\frac{\partial H_{kl}^i}{\partial q_j} \dot q_j - 
\frac{1}{2} \sum_{j=1}^{q}
\frac{\partial H_{jl}^i}{\partial q_k} \dot q_j
\end{equation}
где $p$ и $q$ зависят от $k$ и $l$. Это выражение включает частную производную от матриц и должны быть преобразованы к более простой для вычисления форме. Одно из уравнения, для вращательных сочленений:
\begin{align*}
\frac{\partial H_{kl}^i}{\partial q_j} &= 
(Jn_i + m_i \vec r_{ki} \times \vec r_{li})(\vec r_{k} \times \vec r_{j} \times \vec r_{l})+\\
& +m_i [\vec r_{k} \times \vec r_{j} (\vec R_{kj} \times \vec e_j \times r_{li})
- \vec e_l \times \vec r_{ki} (\vec R_{kj} \times \vec e_j \times \vec e_l) -\\
&- \vec e_k \times r_{li} (\vec R_{kj} \times \vec e_j \times \vec r_{li})] 
+ Js_i \vec e_l \times \vec q_{ik} (\vec e_k \times \vec e_j \vec q_{i1})
\end{align*}
где $\vec R_{kj}$ -- вектор от центра масс звена $j$ к центру масс звена $i$.

Выражение в матричной форме связывает элементы матриц Кориолиса и центробежных с элементами матрицы инерции. Тем не менее, это выражение получается достаточно сложным. 

Этот алгоритм принадлежит к классу не рекурсивных, основанных на уравнениях Лагранжа. Сравнивая его с алгоритмом Uicker-Kahn, можно увидеть, что количество арифметических операций в этих алгоритмах имеют одинаковый порядок. Из этого следует, что его также практически невозможно реализовать для вычисления динамики в реальном времени. Тем не менее, основное достоинство этого алгоритма заключается в том, что он позволяется понять смысл динамических параметров в элементах матриц модели.

В 1981 году M. Renaud предложил эффективную процедуру вычисления параметром матриц частных производных, которые упоминались в алгоритме Uicker-Kahn. Алгоритм основывается на следующем свойстве матриц Денавита-Хартенберга:
\begin{equation}\label{eq5}
\frac{\partial A_k^{k-1}}{\partial q_k} = Q A_k^{k-1}
\end{equation}
где, если звено вращательное:
\begin{equation}
Q = 
\begin{bmatrix}
0&0&0&0\\
0&0&-1&0\\
0&0&0&0\\
0&0&0&0\\
\end{bmatrix},
\end{equation} 
если звено призматическое:
\begin{equation}
Q = 
\begin{bmatrix}
0&0&0&0\\
0&0&0&0\\
0&0&0&0\\
1&0&0&0\\
\end{bmatrix}
\end{equation} 

Основываясь на отношении \ref{eq5}:
\begin{equation}
\frac{\partial W_j}{\partial q_i} = 
\begin{bmatrix}
0\\
\hline
\omega_i
\end{bmatrix}
W_j, (i \le j)
\end{equation}

\begin{equation}
\frac{\partial W_j}{\partial q_i \partial q_k} = 
\begin{bmatrix}
0\\
\hline
\omega_i
\end{bmatrix}
\begin{bmatrix}
0\\
\hline
\omega_k
\end{bmatrix}
W_j, (i \le k \le j)
\end{equation}
где $\omega_i$ -- матрица $3 \times 3$, элементы которой получены непосредственно из матрицы $W_j$.

Этот алгоритм также, принадлежит к классу нерекурсивных, основанных на уравнениях Лагранжа. Тем не менее, количество численных операций по прежнему очень велико, что делает невозможным его применение для расчета в режиме реального времени.

Также, стоит отметить, что в 1983 году M. Renaud предложил еще одну модификацию предыдущего алгоритма, вводя тензорное исчисление, а также ряд рекурсивных соотношений для вычисления динамической модели матриц. Но и эта модификация не привела к значительному сокращению числа операций. Однако, автор показал, что полученная аналитическая модель путем применения его процедур эффективна при численном вычислении. Но, к сожалению, порядок формирования модели не автоматизирован и должен осуществляться вручную. Это процедура достаточна сложна и не позволяет избежать ошибок. Таким образом, этот алгоритм также не может рассматриваться для вычислений в реальном времени.

\subsubsection{Алгоритм Vukobratovic-Potconjak}
В этом алгоритме, для описания расположения звеньев в пространстве, также как и предыдущем, используется формула Родрига. Такой подход облегчает динамический анализ механизмов, так как используются только главные моменты инерции.

Кинематическая часть этого алгоритма представляется выражениями:
\begin{equation}\label{eq7.1}
\omega_i = N(i) \dot q
\end{equation}
\begin{equation}\label{eq7.2}
\dot v_i = M(i) \dot q
\end{equation}
где $\omega_i$ -- вектор угловых скоростей звена $i$ выраженный в $i$-ой локальной системе координат, $v_i$ -- вектор линейных скоростей центра масс $i$-ого звена относительно $i$-ой локальной системы координат, $N(i), M(i)$ -- $3 \times n$ матрицы, зависящие от обобщенных координат, $\dot q$ -- $n \times 1$ вектор скоростей.

Подставляя уравнения \ref{eq7.1} и \ref{eq7.2} в кинетическую энергию системы, получим:

\begin{equation}
E_k = \sum_{i=1}^{n} (\frac{1}{2} m_i v_i^T v_i + \frac{1}{2} \omega_i^T J_i \omega_i)
\end{equation}

и, используя матричную форму представления уравнения Лагранжа:

\begin{equation}
\frac{d}{dt}
(\frac{\partial E_k}{\partial \dot q})
- \frac{\partial E_k}{\partial q} + \frac{E_p}{\partial q} = P
\end{equation}

получим динамическую модель робота:

\begin{equation}
P = H(q) \ddot q + h(q, \dot q)
\end{equation}

В выражении выше $m_i$ обозначает массу $i$ звена, $J_i$ -- диагональную матрицу $3 \times 3$ с главными моментами инерции, $E_k$ и $E_p$ -- кинетическую и потенциальную энергии всей системы, $P$ -- вектор управляющих моментов/сил.

Матрицы модели задаются следующими выражениями:
\begin{equation}
H(q) = \sum_{i=1}^{n}
(
m_i M(i)^T M(i) + N(i)^T J_i N(i)
)
\end{equation}
\begin{equation}
h(q, \dot q) = \frac{\partial (E_p - E_k)}{\partial q} + \dot H(q) \dot q
\end{equation}

Частные производные в последнем выражении это матрица инерции, $H(q)$, которая вычисляется из рекурсивных отношений, которые не представлены здесь по причине их внушительной сложности.

Количество арифметических операций в это алгоритме в несколько раз меньше, чем в алгоритме Uicker-Kahn. 

\subsubsection{Алгоритмы R. Watets и J. Hollerbach}
Эти алгоритмы разрабатывались для решения обратной задачи динамики, как частного случая алгоритма Uicker-Kahn.
Алгоритм не позволяет вычислять матрицу инерции системы, потому что вторая производная обобщенных координат фигурирует в уравнении не явно. Именно поэтому эти алгоритмы не эквивалентны.

В соответствии с алгоритмом R. Waters, управляющие силы/моменты выражаются следующим уравнением:
\begin{equation}
\label{eq6}
P_i = 
\sum_{j=i}^{n}
\left[
tr(
\frac{\partial W_j}{\partial q_i} J_j \ddot W_j^T
)
- m_j \vec g^T \frac{\partial W_j}{\partial q_i} r_{i0}
\right]
\end{equation} 
со следующими рекурсивными отношениями:

\begin{equation}
W_j = W_{j-1} A_j^{j-1}
\end{equation}

\begin{equation}
\dot W_j = \dot W_{j-1} A_j^{j-1} + W_{j-1} \frac{\partial A_j^{j-1}}{\partial q_j} \dot q_j
\end{equation}

\begin{equation}
\ddot W_j = \ddot W_{j-1} A_j^{j-1} + 2 \dot W_{j-1}
\frac{\partial A_j^{j-1}}{\partial q_j} \dot q_j +
W_{j-1} \frac{\partial^2 A_j^{j-1}}{\partial q_j^2} \dot q_j^2 +
W_{j-1} \frac{\partial A_j^{j-1}}{\partial q_j} \ddot q_j  
\end{equation}

Эти рекурсивные соотношения уменьшают количество арифметических операций, требующихся на вычисление управляющих сил/моментов до $n^2$. Уменьшение количества операций по сравнению с алгоритмом Uicker-Kahn, связано с тем, что в рассматриваемом алгоритма матрица инерции системы не вычисляется явно, что не требует вычислять частные производные $\frac{\partial W_j}{\partial q_k \partial q_l}$. Однако, несмотря на внушительное уменьшение количество вычислений, этот алгоритм также не позволяет применять его для вычисления динамики в реальном времени.

Еще более значительное сокращение количества сложений и умножений в уравнении было предложено J. Hollerbach, который заметил, что частная производная $\frac{\partial W_j}{\partial q_i}$ может быть выражена как $(\frac{\partial W_j}{\partial q_i}) W_j^i$, где
\begin{equation}
W_j^i = A_{i+1}^{i} A_{i+2}^{i+1} ... A_{j-1}^j
\end{equation}
матрица трансформации из $i$-ой в $j$ локальную систему координат. Подставляя полученные выражения в уравнение \ref{eq6}, получим:
\begin{equation}
P_i = tr(\frac{\partial W_i}{\partial q_i}) D_i -
\vec{g}^T \frac{\partial W_i}{\partial q_i} c_i
\end{equation}
где 
\begin{equation}
D_i = J_i \ddot W_i^T + A_{i+1}^{i} D_{i+1},
\end{equation}
\begin{equation}
c_i = m_i r_{i0} + A_{i+1}^{i} c_{i+1}.
\end{equation}

Угловое ускорение $\ddot W_i^T$ вычисляется в прямой рекурсии от базы к схвату как и в алгоритме Waters. $D_i$ и $c_i$ вычисляются в обратной рекурсии от схвата к базе.

Таим образом, полученные соотношения позволяют достичь $30n-592$ операций умножения и $675n - 464$ операции сложения.

Несколько позже, Hollerbach еще более уменьшил количество операций применив вместо матриц трансформации $4 \times 4$ матрицы поворота $3 \times 3$ и векторы переноса. С этими изменениями количество операций, необходимых для вычисления моментов/сил стало: умножений 412n - 277, сложений 320n - 201.

% NEWTON
\subsection{Алгоритмы основанные на уравнениях Ньютона-Эйлера}

В этой части представлены алгоритмы основанные на уравнениях Ньютона-Эйлера. Их чаще используют для управления роботами.

Уравнения Ньютона-Эйлера определяются силы/моменты действующие на звенья. Если принять за $F_i$ силу действующую на центр масс звена $i$, выраженную в локальной системе координат этого звена, а за $m_i$ массу $i$ звена, то второй закон Ньютона запишется, как:
\begin{equation}
F_i = m_i w_i
\end{equation}
где $w_i$ ускорение центра масс звена i. Уравнение Эйлера устанавливает взаимосвязь между моментами/силами и моментами инерции:
\begin{equation}
M_i = I_i \epsilon_i + \omega_i \times (I_i \omega_i)
\end{equation}
где $I_i$ -- матрица $3 \times 3$ с главными моментами инерции на диагонали,
$\eps_i$ -- угловое ускорение звена $i$, $\omega_i$ -- угловая скорость звена $i$.

Все векторы выражены в локальной системе координат $i$-ого звена, прикрепленной к центру масс. Оси локальной системы координат совпадают с осями инерции звена.

Выражая уравнения в неподвижной инерциальной системе координат уравнения имеют ту же форму, а матрица инерции находится по формуле: $J_i = A_i I_i A_i^T$. Матрица трансформации $A_i$ из локальной системы координат $i$-ого звена в зафиксированную систему. $J_i$ матрица инерции $i$ звена относительно базовой системы координат.

Далее рассмотрены алгоритмы Ньютона-Эйлера в неподвижной инерциальной и локальной системах координат.

\subsubsection{Алгоритм Vukobratovic-Stepanenko}

Уравнения динамики Ньютона-Эйлера впервые были применены к моделированию в 1973 году M. Vukobratovic и J. Stepanenko. Тогда была введена концепция рекурсивного решения прямой и обратной задач динамики. Этот алгоритм также упоминается как кинестатический, основанные на принципе Д'Аламбера. 

	\begin{figure}[H]
	\center\includegraphics[width=0.8\linewidth]{3.png}
	\caption{Звено манипулятора}
	\label{fig:scr3}
	\end{figure}

На рисунке \ref{fig:scr3} представлено звено с необходимыми обозначениями для дальнейших выкладок. $\vec q_{ij}, (j=1,2,3)$ -- единичный вектор $j$ оси локальной системы координат звена $i$, $\vec r_{ii}$ -- вектор от $i$ сочленения до центра масс $i$ звена, $\vec r_{i, i+1}$ -- вектор от $i+1$ сочленения до центра масс $i$-ого звена, $\vec e_i$ -- единичный вектор $i$ сочленения, $i = 1,..., n$ 




\subsubsection{Алгоритм Huston-Kane}

\subsubsection{Алгоритм ???}

Аппель
\subsubsection{Алгоритм Попова}



******************************
\subsection{Вычислительаня эффективность алгоритмов}

В таблице ~\ref{alhgorithms} представлены результаты анализа методов описания динамики манипуляторов по форме уравнений, вычислительной эффективности (для манипуляторов с шестью степенями свободы); также отражено, обеспечивает ли данный алгоритм замкнутость уравнений и возможность решения прямой задачи динамики.

Как показывает анализ таблицы, многие эффективные в вычислительном плане методы не обеспечивают замкнутости системы уравнений динамики, что ограничивает их применение в задачах управления, а также при анализе влияния различных динамических коэффициентов на движение манипулятора.

В последние годы для повышения эффективности уравнений динамики широко применяются символьные преобразования и алгоритмы распараллеливания вычислений. Применение символьных преобразований, способствует уравниванию различных алгоритмов по вычислительной эффективности. Поэтому важными критериями оценки алгоритма становятся  хорошая  алгоритмизуемость (удобство программирования), замкнутость уравнений, возможность применения символьных преобразований и алгоритмов распараллеливания.

Кроме того, можно отметить, что в работах ряда авторов указывается, что во многих случаях (например в задачах управления роботами) наиболее подходящим способом описания динамики являются уравнения Лагранжа. Это отмечено и в работе Han, где утверждается, что при реализации динамических алгоритмов на параллельных процессорах зависимость данных в методах, основанных на уравнениях Ньютона-Эйлера, намного сильнее, чем при использовании уравнений Лагранжа.

% Please add the following required packages to your document preamble:
% \usepackage{multirow}
% \usepackage{graphicx}
% \usepackage[normalem]{ulem}
% \useunder{\uline}{\ul}{}
\begin{table}[H]
\centering
\caption{Алгоритмы}
\label{alhgorithms}
\begin{tabular}{|l|l|r|r|c|c|}
\hline
\multicolumn{1}{|c|}{\multirow{2}{*}{\begin{tabular}[c]{@{}c@{}}Форма\\ уравнений\end{tabular}}} & \multicolumn{1}{c|}{\multirow{2}{*}{\begin{tabular}[c]{@{}c@{}}Авторы\\ алгоритма\end{tabular}}} & \multicolumn{2}{c|}{Число операций} & \multirow{2}{*}{\begin{tabular}[c]{@{}c@{}}Замкнутость\\ уравнений\end{tabular}} & \multirow{2}{*}{\begin{tabular}[c]{@{}c@{}}Прямая\\ задача\end{tabular}} \\ \cline{3-4}
\multicolumn{1}{|c|}{} & \multicolumn{1}{c|}{} & \multicolumn{1}{c|}{*} & \multicolumn{1}{c|}{+} &  &  \\ \hline
\multirow{5}{*}{Лагранж} & Uicker/Kahn & 66271 & 51548 & + & + \\ \cline{2-6} 
 & \begin{tabular}[c]{@{}l@{}}Vukobratovic/\\ Potconjak\end{tabular} & 37189 & 5652 & - & + \\ \cline{2-6} 
 & Hollerbach 3x3 & 2195 & 1719 & - & - \\ \cline{2-6} 
 & Renaud & 992 & 776 & - & + \\ \cline{2-6} 
 & Li & 951 & 842 & - & + \\ \hline
\multirow{6}{*}{Ньютон-Эйлер} & \begin{tabular}[c]{@{}l@{}}Vukobratovic/\\ Stepanenko\end{tabular} & 2907 & 2068 & + & + \\ \cline{2-6} 
 & Walker/Orin & 1771 & 1345 & - & + \\ \cline{2-6} 
 & Wang/Ravani & 1659 & 1252 & - & + \\ \cline{2-6} 
 & Wang/Ravani & 903 & 654 & - & - \\ \cline{2-6} 
 & \begin{tabular}[c]{@{}l@{}}Luh/Walker/\\ Paul\end{tabular} & 792 & 662 & - & - \\ \cline{2-6} 
 & \begin{tabular}[c]{@{}l@{}}Balafoutis/\\ Patel/Misra\end{tabular} & 489 & 420 & - & - \\ \hline
Д'Аламбер & Lee/Nigam & 2963 & 2209 & + & + \\ \hline
Аппель & Попов & 2929 & 2500 & + & + \\ \hline
Кейн & Ma/Xu & 1020 & 851 & - & - \\ \hline
\end{tabular}
\end{table}

\section{Работы по алгоритмам для расчета динамики}

При выводе уравнений динамики манипуляторов используются различные законы и формулировки общих уравнений динамики систем. Среди них можно выделить методы, основанные на уравнениях Лагранжа, Ньютона-Эйлера, Д'Аламбера, Гаусса, Аппеля, Кейна.

Уравнения динамики в форме Лагранжа впервые были получены в работе Uicker и получили дальнейшее развитие в плане повышения эффективности в работах Kahn, Vukobratovic, Mahil, Renaud, Thomas и Tesar. Все перечисленные методы позволяли решать прямую и обратную задачу динамики, были удобны в алгоритмической реализации (кроме Renaud), но обладали низкой вычислительной эффективностью. Waters и Hollerbach применили рекурсивные преобразования при выводе динамических уравнений, причем при использовании матриц поворотов 3x3 было получено значительное сокращение числа операций, но эти методы позволяли решать лишь обратную задачу динамики, поэтому не были пригодны для моделирования. Рекурсивные преобразования и формулы Родриго использовали Vukobratovic и Potkonjak, причем их метод позволял решать и прямую задачу динамики, хотя его вычислительная эффективность и не столь высока. Значительный прогресс в сокращении числа операций достигнут в работах Renaud и Li, также применивших рекурсивные соотношения.

Среди методов, использующих уравнения Ньютона-Эйлера и позволяющих решать прямую задачу динамики, можно отметить работы Vukobratovic и Stepanenko, Orin и Walker, Armstrong, Wang и Ravani. Во всех работах применяется рекурсия. Отметим также алгоритм Balafoutis, в котором за минимальное число операций решена обратная задача динамики.

Использование принципа Д'Аламбера для уравнений Лагранжа позволяет получить достаточно эффективные динамические соотношения, в которых в явном виде отражены эффекты влияния вращательного и поступательного движения звеньев на динамику манипулятора. В работе Попова получены уравнения динамики в явном виде с использованием уравнений Аппеля, позволяющие решать прямую и обратную задачи динамики. Уравнения Кейна особенно эффективны для расчета обобщенных моментов манипуляторов с замкнутыми кинематическими цепями. В работах Коноплева для описания динамики манипуляционных систем применяются агрегативные модели; метод удобен для применения символьных преобразований. Погореловым разработан пакет программ моделирования динамики широкого класса механических систем, включая роботы-манипуляторы. Интересный подход и язык программирования уравнений движения сложных механических систем, состоящих их твердых тел, предложен Сазоновым.

Полный анализ основных достижений в области моделирования динамики роботов, начиная с первых работ 60-70 годов прошлого века по 2000 гг., дан в работе Featherstone и Orin.

