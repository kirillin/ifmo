\addcontentsline{toc}{section}{Литература}

\label{section_bibliography}

\begin{thebibliography}{99} % Для того, чтобы цифры были одинаково выровнены, говорим, что максимальное число будет двухзначное

\bibitem{besekerskiy_1}
Featherstone, R. (1987). Robot Dynamics Algorithms

\bibitem{besekerskiy_2}
Фу К., Гонсалес Р., Ли К. Робототехника.- М.: Мир, 1989.

\bibitem{besekerskiy_3}
Kane T., Dynamics, New York, Holt,  Rihehart  and  Wiston, 1968.

\bibitem{besekerskiy_4}
Виттенбург Й. Динамика систем твердых тел.- М.: Мир,1980.

\bibitem{besekerskiy_4}
Denavit J, Hartenberg R.S. A kinematic notation for lower­-pair mechanisms based on matrices.,  J. Appl. Mech., 77, 1955, c.215-221.
\bibitem{besekerskiy_4}
Hollerbach J. A recursive Lagrangian formulation of manipu­lator dynamics and comparative study of dynamic complication complexity. IEEE Trans. on SMC, SMC-10, No 11, 1980, c.730-736.
\bibitem{besekerskiy_4}
Vukobratovic M., Stepanenko Y. Mathematical model of gene­ral anthropomorphic systems. Math Biosciences, Vol.17, 1973, c.191-242.
\bibitem{besekerskiy_4}
Balafoutis C, Patel R., Misra P. Efficient modeling and computation of manipulator dynamics using orthogonal cartesian tensors. IEEE J. of Rob. and Autom., 4, N 6, c.665-676.
\bibitem{besekerskiy_4}
Dapper, R. Maafl, V. Zahn, R. Eckmiller, Neural Force Control (NFC) Applied to Industrial Manipulators in Interaction with Moving Rigid Objects, Proceedings of the 1998 IEEE International Conference on Robotics  Automation, Leuven, Belgium,  May 1998.
\bibitem{besekerskiy_4}
G. Rodriguez, A. Jain and K. Kreutz-Delgado, "A Spatial Operator Algebra for Manipulator Modelling and Control," Int. J. Robotics Research, vol. 10, no. 4, pp. 371-381, 1991.
\bibitem{besekerskiy_4}

M. Emami, A. Goldenberg, I. Turksen, Fuzzy-Logic Dynamics Modeling of Robot Manipulators, Proceedings of the 1998 IEEE International Conference on Robotics  Automation, Leuven, Belgium,  May 1998.
\bibitem{besekerskiy_4}
R. Featherstone, D. Orin, Robot Dynamics: Equations and Algorithms, Proceedings of the 2000 IEEE International Conference on Robotics  Automation, San Francisco, CA, April 2000.

\bibitem{besekerskiy_2}
Siciliano, B., and Khatib, O. (eds.) (2008). Springer Handbook of Robotics

\bibitem{besekerskiy_2}
Вукобратович М., Стокич Д., Кирчански Н. Неадаптивное и адаптивное управление манипуляционными роботами. - М.: Мир, 1989.
\bibitem{besekerskiy_2}
Шахинпур М. Курс робототехники пер. с англ. — М, Мир, 1990)
\bibitem{besekerskiy_2}
Megahed S., Renaud M., "Minimization of the Computation Time Nec-
essary for the Dynamic Control of Robot Manipulators", 12th ISIR,
Paris, 1982
\bibitem{besekerskiy_2}
Renaud N., "An Efficient Iterative Analytical Procedure for Ob-
taining a Robot Manipulator Dynamic Model", Proc. of First Inter-
national Symp. of Robotics Research, Bretton Woods, New Hampshire,
USA, 1983.
\bibitem{besekerskiy_2}
Han J.-Y. Fault-tolerant computing for robot kinematics using linear arithmetic code. IEEE Int. Conf. Robotics and Auto­mation, Cincinnati, May May 13-18, 1990, vol. 1, c.285-290.

Uicer J.J. Dynamic force analysis of spatial linkages, ASME J. of appl. mech., June, 1967, c.418-424.

\end{thebibliography}
