\subsection{Работы по алгоритмам для расчета динамики}

В таблице ~\ref{alhgorithms} представлены результаты анализа методов описания динамики манипуляторов по форме уравнений, вычислительной эффективности (для манипуляторов с шестью степенями свободы); также отражено, обеспечивает ли данный алгоритм замкнутость уравнений и возможность решения прямой задачи динамики.

Как показывает анализ таблицы, многие эффективные в вычислительном плане методы не обеспечивают замкнутости системы уравнений динамики, что ограничивает их применение в задачах управления, а также при анализе влияния различных динамических коэффициентов на движение манипулятора.

В последние годы для повышения эффективности уравнений динамики широко применяются символьные преобразования и алгоритмы распараллеливания вычислений. Применение символьных преобразований, способствует уравниванию различных алгоритмов по вычислительной эффективности. Поэтому важными критериями оценки алгоритма становятся  хорошая  алгоритмизуемость (удобство программирования), замкнутость уравнений, возможность применения символьных преобразований и алгоритмов распараллеливания.

Кроме того, можно отметить, что в работах ряда авторов указывается, что во многих случаях (например в задачах управления роботами) наиболее подходящим способом описания динамики являются уравнения Лагранжа. Это отмечено и в работе Han, где утверждается, что при реализации динамических алгоритмов на параллельных процессорах зависимость данных в методах, основанных на уравнениях Ньютона-Эйлера, намного сильнее, чем при использовании уравнений Лагранжа.

% Please add the following required packages to your document preamble:
% \usepackage{multirow}
% \usepackage{graphicx}
% \usepackage[normalem]{ulem}
% \useunder{\uline}{\ul}{}
\begin{table}[]
\centering
\caption{My caption}
\label{alhgorithms}
\resizebox{\textwidth}{!}{%
\begin{tabular}{clllll}
\multirow{2}{*}{\begin{tabular}[c]{@{}c@{}}Форма\\ уравнений\end{tabular}} & \multicolumn{1}{c}{\multirow{2}{*}{\begin{tabular}[c]{@{}c@{}}Авторы\\ алгоритма\end{tabular}}} & \multicolumn{2}{c}{Число операций} & \multicolumn{1}{c}{\multirow{2}{*}{\begin{tabular}[c]{@{}c@{}}Замкнутость\\ уравнений\end{tabular}}} & \multicolumn{1}{c}{\multirow{2}{*}{\begin{tabular}[c]{@{}c@{}}Прямая\\ задача\end{tabular}}} \\
 & \multicolumn{1}{c}{} & \multicolumn{1}{c}{*} & \multicolumn{1}{c}{+} & \multicolumn{1}{c}{} & \multicolumn{1}{c}{} \\
\multirow{5}{*}{Лагранж} & Uicker/Kahn & 66271 & 51548 & + & + \\
 & Vukobratovic/Potconjak & 37189 & 5652 & - & + \\
 & Hollerbach 3x3 & 2195 & 1719 & - & - \\
 & Renaud & 992 & 776 & - & + \\
 & Li & 951 & 842 & - & + \\
\multirow{6}{*}{Ньютон-Эйлер} & Vukobratovic/Stepanenko & 2907 & 2068 & + & + \\
 & Walker/Orin & 1771 & 1345 & - & + \\
 & Wang/Ravani & 1659 & 1252 & - & + \\
 & Wang/Ravani & 903 & 654 & - & - \\
 & Luh/Walker/Paul & 792 & 662 & - & - \\
 & Balafoutis/Patel/Misra & 489 & 420 & - & - \\
Д'Аламбер & Lee/Nigam & 2963 & 2209 & + & + \\
Аппель & Попов & 2929 & 2500 & + & + \\
Кейн & Ma/Xu & 1020 & 851 & - & -
\end{tabular}%
}
\end{table}

