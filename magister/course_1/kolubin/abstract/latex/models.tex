\subsection{Динамические модели}

Формирование эффективных уравнений динамики роботов, которые могут быть рассчитаны на ЭВМ за минимальное время, является одной из важнейших задач в робототехнике. Ее решение необходимо для моделирования динамики манипуляторов в масштабе реального времени, для разработки эффективных алгоритмов управления роботами с учетом динамики, для повышения эффективности исследования и разработки манипуляторов.

Одни из первых результатов в этой области принадлежат Кейну и Виттенбургу. Полученные ими уравнения справедливы не только для роботов, но и для более широкого класса систем, состоящих из шарнирно связанных твердых тел. В дальнейшем было разработано большое количество алгоритмов формирования динамических уравнений манипуляторов, в которых использовались различные способы описания кинематики, расчета кинематических и динамических величин, а также различные формы уравнений динамики системы тел.

Описание кинематики – это способ задания систем координат, связанных со звеньями манипулятора, и выбора параметров, которые однозначно определяют взаимное положение звеньев и конфигурацию всего манипулятора. В представлении Денавита-Хартенберга начала систем координат расположены в шарнирах, а их оси формируются по правилам, которые определяются кинематикой манипулятора. В другом методе описания кинематики локальные системы координат привязаны к центрам масс звеньев, а их оси направлены вдоль главных осей инерции. Параметры, определяемые относительно таких систем координат, удобны для динамического анализа.

Еще одной характеристикой методов математического моделирования манипуляторов является способ расчета кинематических и динамических величин, определяющих математическую модель манипулятора. Для этого используются однородные координаты и матрицы преобразования координат размерности 4x4, определяющие относительное положение и ориентацию звеньев манипулятора; матрицы поворотов размерности 3x3 и вектора относительных перемещений; формулы Родриго; ортогональные тензоры; кватернионы; метод векторных параметров с использованием групп Ли .

Хотя вычислительная эффективность того или иного метода формирования динамических уравнений зависит в первую очередь от особенностей его реализации (использования рекурсивных преобразований, динамических аналогий и др.), можно отметить и существенную роль выбора подходящего способа расчета модели манипулятора. Например, матрицы преобразования однородных координат размерности 4x4, обладающие универсальностью в кинематическом описании, практически не используются в задачах реального времени из-за больших вычислительных затрат, необходимых для выполнения операций над ними. В то же время, использование матриц поворотов размера 3x3 позволяет получить эффективные алгоритмы расчета кинематики и динамики. 

Эффективно использование кватернионов, ортогональных тензоров, однако в ряде задач (например, при управлении в декартовых координатах) предпочтительнее использовать матричные представления.

Среди самых современных методов моделирования динамики манипуляторов отметим подходы, основанные на использовании нейронных  сетей, пространственных операторов, групп Ли, методов нечеткой логики. Для описания динамики сложных структур (параллельных роботов, манипуляторов с большой избыточностью степеней подвижности, роботов-гуманоидов) используются методы расчета динамики в операционном пространстве роботов и другие.