\subsection{Работы по алгоритмам для расчета динамики}

При выводе уравнений динамики манипуляторов используются различные законы и формулировки общих уравнений динамики систем. Среди них можно выделить методы, основанные на уравнениях Лагранжа, Ньютона-Эйлера, Д'Аламбера, Гаусса, Аппеля, Кейна.

Уравнения динамики в форме Лагранжа впервые были получены в работе Uicker и получили дальнейшее развитие в плане повышения эффективности в работах Kahn, Vukobratovic, Mahil, Renaud, Thomas и Tesar. Все перечисленные методы позволяли решать прямую и обратную задачу динамики, были удобны в алгоритмической реализации (кроме Renaud), но обладали низкой вычислительной эффективностью. Waters и Hollerbach применили рекурсивные преобразования при выводе динамических уравнений, причем при использовании матриц поворотов 3x3 было получено значительное сокращение числа операций, но эти методы позволяли решать лишь обратную задачу динамики, поэтому не были пригодны для моделирования. Рекурсивные преобразования и формулы Родриго использовали Vukobratovic и Potkonjak, причем их метод позволял решать и прямую задачу динамики, хотя его вычислительная эффективность и не столь высока. Значительный прогресс в сокращении числа операций достигнут в работах Renaud и Li, также применивших рекурсивные соотношения.

Среди методов, использующих уравнения Ньютона-Эйлера и позволяющих решать прямую задачу динамики, отметим работы Vukobratovic и Stepanenko, Orin и Walker, Armstrong, Wang и Ravani. Во всех работах применяется рекурсия. Отметим также алгоритм Balafoutis, в котором за минимальное число операций решена обратная задача динамики.

Использование принципа Д'Аламбера для уравнений Лагранжа позволяет получить достаточно эффективные динамические соотношения, в которых в явном виде отражены эффекты влияния вращательного и поступательного движения звеньев на динамику манипулятора. В работе Попова получены уравнения динамики в явном виде с использованием уравнений Аппеля, позволяющие решать прямую и обратную задачи динамики. Уравнения Кейна особенно эффективны для расчета обобщенных моментов манипуляторов с замкнутыми кинематическими цепями. В работах Коноплева для описания динамики манипуляционных систем применяются агрегативные модели; метод удобен для применения символьных преобразований. Погореловым разработан пакет программ моделирования динамики широкого класса механических систем, включая роботы-манипуляторы. Интересный подход и язык программирования уравнений движения сложных механических систем, состоящих их твердых тел, предложен Сазоновым.

Полный анализ основных достижений в области моделирования динамики роботов, начиная с первых работ 60-70 годов прошлого века по 2000 гг., дан в работе Featherstone и Orin.

