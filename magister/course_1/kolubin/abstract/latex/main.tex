\documentclass[a4paper,14pt]{extreport}
\usepackage[utf8]{inputenc}
\usepackage[T2A]{fontenc}
\usepackage[russian]{babel}
\usepackage{eufrak}
% поля:
\usepackage[left=2.5cm, right=2cm, top=2cm, bottom=2cm]{geometry}
\linespread{1}
\usepackage{indentfirst} % отделять первую строку раздела абзацным отступом
\setlength\parindent{5ex}
\addto{\captionsrussian}{\renewcommand*{\contentsname}{Содержание}}
\usepackage[hidelinks]{hyperref} % гиперссылки в содержании
\usepackage{graphicx}
\usepackage{float}
\usepackage{amsmath}
\renewcommand*{\thesection}{\arabic{section}}

\usepackage{multirow}
\usepackage[normalem]{ulem}
\useunder{\uline}{\ul}{}

\usepackage{cmap}%позволяет копировать кириллицу из скомпилированного файла

% Глубина разделов, попадающих в содержание
\setcounter{tocdepth}{2}

\linespread{1.3} % настройка межстрочного интервала
\tolerance=1000 % настройка чувствительности вставки переносов
\hfuzz=0pt
\sloppy

\begin{document}
% Переменование "Список литературы" в "Литература"
\renewcommand{\refname}{Литература}
\begin{titlepage}
	\begin{center}
		\large
		МИНИСТЕРСТВО ОБРАЗОВАНИЯ И НАУКИ\\ РОССИЙСКОЙ ФЕДЕРАЦИИ
		
		\textbf{Федеральное агентство по образованию}
		\vspace{0.5cm}
		
		УНИВЕРСИТЕТ ИТМО
		\vspace{0.25cm}
		
		Факультет компьютерных технологий и управления
		
		Кафедра систем управления и информатики
		\vfill
		
		
		Студент: Артемов Кирилл\\
		группа P4135\\
		
		\textsc{\LARGE Реферат}\\[5mm]
		
		{\LARGE Вычислительно эффективные алгоритмы решения прямой и обратной задач динамики}	\bigskip
		
	\end{center}
	\vfill
	
	\newlength{\ML}
	\settowidth{\ML}{«\underline{\hspace{0.7cm}}» \underline{\hspace{1cm}}}
	\hfill\begin{minipage}{0.4\textwidth}
		Преподаватель\\
		\underline{\hspace{\ML}} С.\,А.~Колюбин\\
		«\underline{\hspace{0.7cm}}» \underline{\hspace{2cm}} 2016 г.
	\end{minipage}%
	\bigskip
	
	\vfill
	
	\begin{center}
		Санкт-Петербург, 2016 г.
	\end{center}
\end{titlepage}

\tableofcontents
\newpage

\section{Введение}
Первые работы в области робототехнических систем были написаны более двух десятков лет тому назад. Тем не менее, как в прошлом веке, так и сейчас сообщество робототехники сосредоточено на проблеме вычислительной эффективности. Многие из наиболее эффективных алгоритмов использующиеся сейчас в динамике, которые применимы к широкому классу механизмов, были разработаны учеными очень тесно связанными с роботами. 

Наряду с тем, что эффективность алгоритмов имеет несомненно важное значение для моделирования и управления все более сложными механизмами, работающими на более высоких скоростях, другие аспекты динамики также играют важную роль. 
Алгоритмы формулируются в соответствии с уравнениями движения, что способствует более простой разработке и реализации. Кроме того, важно, чтобы алгоритмы могли решать задачи для как можно более общего случая устройств с разной геометрией и структурой.

В этой работе рассмотрены общие методы решения прямой и обратной задач динамики: кратко описаны основные положения математических моделей динамики роботов, уравнения динамики, наиболее эффективные на сегодняшней день алгоритмы расчета ее, сделан обзор литературе по теме, а также существующие нерешенные задачи в области алгоритмов расчета динамики.

\newpage
\section{Мотивация}
В принципе, решение задач динамики для роботов не представляет особых трудностей - робототехническое устройство это, чаще всего, система твердых тел и как получить уравнения движения таких систем известно достаточно давно. Но на практике не все так просто и встает вопрос о поиске способа получения динамики робота в реальном времени, что приводит к необходимости использовать вычислительно эффективные алгоритмы.

В простейшей форме, уравнение движение для робототехнического механизма может быть представлено вектором дифференциальных уравнений[1]:

\begin{center}
$\ddot q = \phi(q, \dot q, \tau)$	
\end{center}
где q - вектор обобщенных координат, $\dot q$ и $\ddot q$ производные по времени, и $\tau$ - вектор сил. Решая такое уравнение относительно $q$ необходимо рассчитать значение функции $\phi$ и выполнить операции интегрирование. Для систем твердых тел сложных робототехнических механизмов основной проблемой является поиск значения функции $\phi$.

Для не сложной системы твердых тел, наиболее простым будет выразить уравнения функции для каждого из параметров, результатом чего станет набор простых функций для каждой из степеней свободы. Далее, вычисляя их, получим числовые значения.

К сожалению, такой подход не работает для более сложных систем, в которых число параметров стремительно растет с увеличением количество тел в системе. Что требует применения специальных методов и алгоритмов.
Все динамические алгоритмы для сложных многозвенных тел оценивают уравнения движения поэтапно. На каждом из этапов получаются значения, которые используются на следующем этапе вычислений и так далее.

Динамика робототехнических систем представляет собой отношения между действующими на механизм силами и ускорениями, которые они  производят. Как правило, робототехнические механизмы моделируются как система твердых тел, в таком случае динамика такой системы также рассматривается как динамика твердого тела. 
Существуют две основные задачи динамики:
\begin{enumerate}
	\item прямая задача -- даны силы/моменты, найти ускорения;
	\item обратная задача -- даны ускорения, найти силы/моменты.
\end{enumerate}

Прямая задача в основном используется для моделирования. Обратная задача более распространена, например для: управления движением и силой робота в режиме реального времени; планирования траекторий и оптимизации; при разработке роботов; а также, как один из элементов алгоритмов прямой задачи.

Также задачами динамики являются:
\begin{enumerate}
	\item вычисление коэффициентов управлений движением;
	\item определение параметров инерции -- оценка параметров инерции механизмов робота посредством измерения их динамического поведения;
	\item гибридная динамика: даны силы в некоторых сочленениях и ускорения в других, найти неизвестные силы и ускорения.
\end{enumerate}

Для определение всех необходимых значений параметров в режиме реального времени стоит задача вычисления их, тут-то и пригождаются вычислительно эффективные алгоритмы.

% общие методы решения
\include{common}


\section{Открытые проблемы}
Отметим, что, несмотря на постоянный рост мощности компьютеров, требование высокой вычислительной эффективности уравнений динамики остается критичным. Это объясняется тем, что, во-первых, в системах управления роботов используются как правило относительно медленные процессоры, и для решения уравнений динамики в реальном времени необходимы эффективные алгоритмы расчета. А, во-вторых, сложность механических структур современных роботов (параллельных, с избыточными степенями подвижности, так называемых роботов-гуманоидов) требует эффективных методов расчета динамики для задач их моделирования и управления.

Алгоритмы для расчета динамика роботов с гибкими звеньями и сочленениями, проблемы идентификации параметров динамических моделей, динамика физического контакта робота с окружающей средой, динамика роботов с подвижной базой -- все эти задачи требуют усовершенствования.


\addcontentsline{toc}{section}{Литература}

\label{section_bibliography}

\begin{thebibliography}{99} % Для того, чтобы цифры были одинаково выровнены, говорим, что максимальное число будет двухзначное

\bibitem{besekerskiy_1}
Featherstone, R. (1987). Robot Dynamics Algorithms

\bibitem{besekerskiy_2}
Фу К., Гонсалес Р., Ли К. Робототехника.- М.: Мир, 1989.

\bibitem{besekerskiy_3}
Kane T., Dynamics, New York, Holt,  Rihehart  and  Wiston, 1968.

\bibitem{besekerskiy_4}
Виттенбург Й. Динамика систем твердых тел.- М.: Мир,1980.

\bibitem{besekerskiy_4}
Denavit J, Hartenberg R.S. A kinematic notation for lower­-pair mechanisms based on matrices.,  J. Appl. Mech., 77, 1955, c.215-221.
\bibitem{besekerskiy_4}
Hollerbach J. A recursive Lagrangian formulation of manipu­lator dynamics and comparative study of dynamic complication complexity. IEEE Trans. on SMC, SMC-10, No 11, 1980, c.730-736.
\bibitem{besekerskiy_4}
Vukobratovic M., Stepanenko Y. Mathematical model of gene­ral anthropomorphic systems. Math Biosciences, Vol.17, 1973, c.191-242.
\bibitem{besekerskiy_4}
Balafoutis C, Patel R., Misra P. Efficient modeling and computation of manipulator dynamics using orthogonal cartesian tensors. IEEE J. of Rob. and Autom., 4, N 6, c.665-676.
\bibitem{besekerskiy_4}
Dapper, R. Maafl, V. Zahn, R. Eckmiller, Neural Force Control (NFC) Applied to Industrial Manipulators in Interaction with Moving Rigid Objects, Proceedings of the 1998 IEEE International Conference on Robotics  Automation, Leuven, Belgium,  May 1998.
\bibitem{besekerskiy_4}
G. Rodriguez, A. Jain and K. Kreutz-Delgado, "A Spatial Operator Algebra for Manipulator Modelling and Control," Int. J. Robotics Research, vol. 10, no. 4, pp. 371-381, 1991.
\bibitem{besekerskiy_4}

M. Emami, A. Goldenberg, I. Turksen, Fuzzy-Logic Dynamics Modeling of Robot Manipulators, Proceedings of the 1998 IEEE International Conference on Robotics  Automation, Leuven, Belgium,  May 1998.
\bibitem{besekerskiy_4}
R. Featherstone, D. Orin, Robot Dynamics: Equations and Algorithms, Proceedings of the 2000 IEEE International Conference on Robotics  Automation, San Francisco, CA, April 2000.

\bibitem{besekerskiy_2}
Siciliano, B., and Khatib, O. (eds.) (2008). Springer Handbook of Robotics

\bibitem{besekerskiy_2}
Вукобратович М., Стокич Д., Кирчански Н. Неадаптивное и адаптивное управление манипуляционными роботами. - М.: Мир, 1989.
\bibitem{besekerskiy_2}
Шахинпур М. Курс робототехники пер. с англ. — М, Мир, 1990)
\bibitem{besekerskiy_2}
Megahed S., Renaud M., "Minimization of the Computation Time Nec-
essary for the Dynamic Control of Robot Manipulators", 12th ISIR,
Paris, 1982
\bibitem{besekerskiy_2}
Renaud N., "An Efficient Iterative Analytical Procedure for Ob-
taining a Robot Manipulator Dynamic Model", Proc. of First Inter-
national Symp. of Robotics Research, Bretton Woods, New Hampshire,
USA, 1983.
\bibitem{besekerskiy_2}
Han J.-Y. Fault-tolerant computing for robot kinematics using linear arithmetic code. IEEE Int. Conf. Robotics and Auto­mation, Cincinnati, May May 13-18, 1990, vol. 1, c.285-290.

Uicer J.J. Dynamic force analysis of spatial linkages, ASME J. of appl. mech., June, 1967, c.418-424.

\end{thebibliography}

\end{document}


