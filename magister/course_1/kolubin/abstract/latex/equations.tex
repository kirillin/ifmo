\subsection{Уравнения движения}

Есть две основные формы записи уравнени

\begin{enumerate}
	\item уравнение в конфигурационном пространстве
\begin{center}
	$H(q) \ddot q + C(q, \dot q)\dot q + \tau_g (g) = \tau$	
\end{center}
	\item уравнение в операционном пространстве
\begin{center}
	$\Lambda(x) v + \mu (x, v) + \rho (x) = f$
\end{center}		
\end{enumerate}



где x - положеине схвата, v - пространственный вектор скорости схвата, f - пространственнй вектор действующей силы на схват и $\Lambda$ и $\rho$ - коэффициенты уравнения движения. $\Lambda$ это матрица инерции в опереционном пространстве, $\mu$ - содержит кориолисовы и центробежные составляющие; и $\rho$ - вектор гравитации, f - сумма внешних сил, действующих на схват.
Эти уравнения показывают функциональную зависимость: H - функция от q. $\Lambda$ функиця от x. Однако, эти зависимости часто отбрасывают, для более простого понимания. x - это вектор координат в операционном пространстве,v и f пространственные вектора, обозначающие скорости схвата и действие внешних сил на него.

Строго говоря, коэффициенты уравнений зависят не только от $q, \dot q$ и $f_{ext}$, но также от динамической модели робототехнической системы.
Т.е. описание механизма отдельно для каждой из его части: звенья, сочленения и их характеризующие их параметры. Динамическая модель состоит из:
кинематической модели
множества параметров инерции
Для описание инерции твердого тела нужно десять параметров инерции: масса, расположение центра масс, шесть вращателльных параметров инерции. Однако, когда тела соединяют друг с другом их степени свободы ограничиваются и некоторые из этих параметров могут не оказывать влияния на поведение системы. Поэтому при составлении динамической модели робототехнического механизма от кинематической модели и измерения ее динамических характеристик, процедура сводится к идентификации значений множества основных параметров инерции, которые поддаются наблюдению в полученной системе.


Наиболее эффективными алгоритмами для расчет динамических параметром робототехнических систем используются следующие три:

\begin{enumerate}
	\item рекурсивный алгоритм Ньтона-Эйлера (RNEA). Алгоритм представляет собой решение обратной задачи динамики и имеет вычислительную сложжность O(n). В зависисмости от параметров запуска может быть использован для расчета: вектора гравитации, кориолиовых/центробежных сил, матрицы инерции и обобщенного момента.
	\item articulated-body algorithm (ABA). Решает прямую задачу динамики. Вычислительная сложность O(n).
	\item composite-rigid-body algorithm (CRBA). Предназначен для расчета матрицы инерции в для пространства обобщенных координат. Вычислительная сложность $O(n^2)$. При комбинации его с алгоритмом RNEA, решается прямая задача динамики с вычислительной сложность $O(n^3)$. Хотя эта комбинация получается гораздо медленнее, чем алгоритм ABA для больших значений, он получается такимже эффективным при значениях n порядка шести.
	
\end{enumerate}


По приблизительным данным: RNEA требует около 200 вычислений с плавающей точкой для каждого тела, ABA требует около 450 и CRBA - около $16n^2$ вычислений для матрицы инерции размером $nxn$.