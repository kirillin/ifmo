\documentclass[utf8, russian, hpadding=5mm, vpadding=15mm, floatsection, columnxxvi, columnxxxi, columnxxxii, equationsection, pointsection, footnoteasterisk]{eskdtext}

%доп. пакеты
\usepackage{amsfonts, amsmath, amssymb}	%Для математических штуковин
\usepackage{wallpaper}					%Для вставки сторонних страниц		

%доп. настройки
\bibliographystyle{ugost2008}					%Стиль списка литературы
\graphicspath{{./images/}{./extra_pdf_pages/}}	%Папки с картинками

%на случай, если команда не определена
\newcommand{\No}{\textnumero}

%для определения форматирования нумерации элементов списка литературы
\makeatletter
\renewcommand{\@biblabel}[1]{#1}
\makeatother

%для штампа	
\ESKDtitle{{\small РИРМ “Интеллектуальное управление в условиях неопределенности”}\\ \small Пояснительная записка}
\ESKDsignature{КСУИ.06.4135.001 ПЗ}
\ESKDgroup{\footnotesize Университет ИТМО\\Кафедра СУиИ\\гр.~P4135}
\ESKDauthor{\resizebox{2.22cm}{\height}{Артемов К.}}
\ESKDchecker{\resizebox{2.22cm}{\height}{Ушаков А.В.}}
\ESKDnormContr{\resizebox{2.22cm}{\height}{ }}
\ESKDapprovedBy{\resizebox{2.22cm}{\height}{ }}

%оступы от заголовков разделов и подразделов
\ESKDsectSkip{section}{7mm}{7mm}
\ESKDsectSkip{subsection}{5mm}{5mm}
\ESKDsectSkip{subsubsection}{3mm}{3mm}
\renewcommand{\theenumi}{\arabic{enumi}}

%тело документа
\begin{document}
%\addtocounter{page}{1} %если в документ будут вставлены иные страницы (ТЗ и проч.)
\ESKDthisStyle{empty}
\mbox{}
\ThisLRCornerWallPaper{1}{title_page_1.pdf}
\newpage
%\ESKDthisStyle{empty}
%\mbox{}
%\ThisLRCornerWallPaper{1}{title_page2.pdf}
%\newpage

\ESKDthisStyle{formII}
\tableofcontents
\newpage
\topmargin = 0 mm
%4.1 Введение.Постановка задачи								
%4.2.Построение МТЧ НОУи результаты ее исследования	________________________
%4.3.Построение МТЧ ДОУи и результаты ее исследования	________________________
%4.4.Построение МТЧ СНС и результаты ее исследования	________________________
%4.5.Построение МФМЧ и результаты ее исследования__________________		
%4.6.Построение медианного МУ НОУ и оценка его результатов______________________
%4.7.Синтез неадаптивного и адаптивного управления, обеспечивающего параметрическую инвариантность выхода СНС относительно неопределенности НОУ

\section*{Введение. Постановка задачи}
\addcontentsline{toc}{section}{Введение.Постановка задачи}

Задан непрерывный объект управления (НОУ) с помощью передаточной функции (ПФ) «вход-выход (ВВ)»
\begin{equation}\label{eq_pf0}
	\Phi (s, q) = \cfrac{b_0 (1 + q_1) s + b_1 (1 + q_2)}{(a_0 (1+q_3)s + a_1 (1+q_4))(a_2 (1+q_5) s^2 + a_3 (1 + q_6) s + a_4 (1 + q_7))}
\end{equation}
где $q_{10}=q_{20}=q_{30}=q_{40}=q_{50}=q_{60}=q_{70}=0$~--- номинальные значения параметров $q_{j0}, j = \overline{1,7}$.

Необходимо проделать работу в соответствии с заданием на расчетно-исследовательскую  работу магистранта (РИРМ). Исходные данные для варианта~№6 ААББАААА указаны в таблице~\ref{problem_data}.

\begin{table}[h!]
	\caption{Исходные данные}
	\begin{tabular}{|p{0.5\linewidth}|p{0.4\linewidth}|}
%\hline
%Параметр & Значения \\
\hline
1.1. Значения параметров ПФ & 
$b_0 = 3; b_1 = 0.4; a_0=2; a_1 = 0.6; a_2 = 0; a_3 = 6; a_4 = 10$
\\
\hline
1.2. Базис описания НОУ & канонический управляемый
\\
\hline
2.1. Интервал дискретности & $\Delta{t} = 0.03$с\\
\hline
2.2. Метод перехода к ДОУ & с помощью интегральной модели ВСВ НОУ
\\
\hline
3. Характеристическая частота  & $\omega_0 = 3 c^{-1}$\\
\hline
5. Граничные (угловые) значения параметра $q_j$  & $\underline{q_j} = -0.2; \overline{q_j} = 0.2$ \\ 
\hline
6. Относительная интервальность матрицы состояния системы & $\delta_{IR} F = 0.02$\\
\hline
7. Величина параметрической неопределенности  & $\underline{q_j} = -0.2; \overline{q_j} = 0.2$\\
\hline
	\end{tabular}
	\label{problem_data}
\end{table}


\newpage
\section{Построение МТЧ НОУ и результаты ее исследования}\label{problem_1}

\begin{enumerate}
	\item Записать непрерывный ОУ (НОУ) в форме <<вход-состояние-выход (ВСВ)>> в требуемом базисе;
	\item Построить модель траекторной чувствительности (МТЧ) НОУ;
	\item Произвести ранжирование параметров  по потенциальной чувствительности к ним выхода ОУ с использованием матрицы управляемости агрегированной системы; 
	
	Оценить, какое из дополнительных движений, вызванных вариацией, потребует максимальных затрат управления при обеспечении его асимптотической сходимости к нулю.
\end{enumerate}


\subsection{Непрерывный ОУ в форме ВСВ}

Заданный ОУ описывается ПФ
\begin{equation}\label{eq_OU}
\Phi (s, q) = \cfrac{3 (1 + q_1) s + 0.4 (1 + q_2)}{(2 (1+q_3)s + 0.6 (1+q_4))(6 (1 + q_6) s + 10 (1 + q_7))}
\end{equation}

Для составления векторно-матричного описания ОУ запишем ПФ в форме
\begin{equation*}\label{eq_pf_coc}
\Phi (s, q) = \cfrac{\cfrac{(1 + q_1)}{4(1+q_3)(1+q_6)} s + \cfrac{(1 + q_2)}{30(1+q_3)(1+q_6)}}{s^2 + \cfrac{20(1+q_3)(1+q_7)+3.6(1+q_4)(1+q_6)}{12(1+q_3)(1+q_6)} s + \cfrac{(1+q_4)(1+q_7)}{2(1+q_3)(1+q_6)}}
\end{equation*}

В каноническом управляемом базисе векторно-матричное представление ОУ принимает вид:
\begin{equation}\label{eq_iso_coc}
\begin{cases}
\dot x(t,q) &= A(q) x(t,q) + B u(t)\\
y(t,q) &= C(q) x(t,q)
\end{cases}
\end{equation}
в котором
\begin{equation}
	A(q) =
	\begin{bmatrix}
		0 & 1 \\
		- \cfrac{(1+q_4)(1+q_7)}{2(1+q_3)(1+q_6)} & - \cfrac{20(1+q_3)(1+q_7)+3.6(1+q_4)(1+q_6)}{12(1+q_3)(1+q_6)}
	\end{bmatrix}
\end{equation}
\begin{equation}
	B =
	\begin{bmatrix}
		0\\
		1
	\end{bmatrix}
\end{equation}

\begin{equation}
	C(q) =
	\begin{bmatrix}
	\cfrac{(1 + q_2)}{30(1+q_3)(1+q_6)} & \cfrac{(1 + q_1)}{4(1+q_3)(1+q_6)} 
	\end{bmatrix}
\end{equation}

\subsection{Модель траекторной чувствительности НОУ}

ПФ номинального ОУ, когда параметры $q_{j} = 0, j = \overline{1,7}$, представляет собой
\begin{equation}
	\Phi(s, 0) = \cfrac{\cfrac{1}{4} s + \cfrac{1}{30}}{s^2 + \cfrac{236}{120} s + \cfrac{1}{2}} 
\end{equation}

Матрицы модели ВСВ номинального ОУ имеют реализации
\begin{align*}
A =
\begin{bmatrix}
0 & 1 \\
- \cfrac{1}{2} & - \cfrac{236}{120}
\end{bmatrix};
B =
\begin{bmatrix}
0\\
1
\end{bmatrix};
C =
\begin{bmatrix}
\cfrac{1}{30} & \cfrac{1}{4}
\end{bmatrix}
\end{align*}

Введем обозначения
\begin{align*}
	&A_{q_j} = \cfrac{\partial{A(q)}}{\partial{q_j}} \bigg|_{q=q_0};
	B_{q_j} = \cfrac{\partial{B(q)}}{\partial{q_j}} \bigg|_{q=q_0};
	C_{q_j} = \cfrac{\partial{C(q)}}{\partial{q_j}} \bigg|_{q=q_0};\\
	&A(q)|_{q=q_0} = A;
	B(q)|_{q=q_0} = B;
	C(q)|_{q=q_0} = C;\\
	&x(t,q)|_{q=q_0} = x(t);
	y(t,q)|_{q=q_0} = y(t);\\
	&\cfrac{\partial{x(t,q)}}{\partial{q_j}} \bigg|_{q=q_0} = \sigma_j(t);
	\cfrac{\partial{y(t,q)}}{\partial{q_j}} \bigg|_{q=q_0} = \eta_j(t);
\end{align*}

Теперь для $j$-й модели траекторной чувствительности получим представление МТЧ
\begin{equation}\label{eq_mts}
	\begin{cases}
		\dot \sigma_j(t) &= A \sigma_j(t) + A_{q_j} x(t) + B_{q_j} u(t); 
		\sigma_j (0) = 0\\
		\eta_j (t) &= C \sigma_j (t) + C_{q_j} x(t)
	\end{cases}
\end{equation}

МТЧ будет генерировать функции траекторной чувствительности $\sigma_j (t)$ по состоянию и $\eta_j (t)$ по выходу, если ее дополнить моделью номинального ОУ~\ref{eq_iso_coc}.

На состояние заданного ОУ влияют $p = 6$ (далее, под записью $j = \overline{1, p}$ будет подразумеваться, что $j = 1,2,3,4,6,7$) параметров: $q_1, q_2, q_3, q_4, q_6, q_7$. Вычислим матрицы моделей траекторной чувствительности
\begin{align}
	&A_{q_1} = 
	\begin{bmatrix}
		0 & 1\\
		0 & 0
	\end{bmatrix};
	B_{q_1} = 
	\begin{bmatrix}
		0\\
		1
	\end{bmatrix};
	C_{q_1} = 
	\begin{bmatrix}
		0 & \cfrac{1}{4}
	\end{bmatrix};\\
	&A_{q_2} = 
	\begin{bmatrix}
	0 & 1\\
	0 & 0
	\end{bmatrix};
	B_{q_2} = 
	\begin{bmatrix}
	0\\
	1
	\end{bmatrix};
	C_{q_2} = 
	\begin{bmatrix}
	\cfrac{1}{30} & 0
	\end{bmatrix};\\
	&A_{q_3} = 
	\begin{bmatrix}
	0 & 1\\
	\cfrac{1}{2} & \cfrac{36}{120}
	\end{bmatrix};
	B_{q_3} = 
	\begin{bmatrix}
	0\\
	1
	\end{bmatrix};
	C_{q_3} = 
	\begin{bmatrix}
	- \cfrac{1}{30} & - \cfrac{1}{4}
	\end{bmatrix};\\
	&A_{q_4} = 
	\begin{bmatrix}
	0 & 1\\
	- \cfrac{1}{2} & - 3.6
	\end{bmatrix};
	B_{q_4} = 
	\begin{bmatrix}
	0\\
	1
	\end{bmatrix};
	C_{q_4} = 
	\begin{bmatrix}
	0 & 0
	\end{bmatrix};\\
	&A_{q_6} = 
	\begin{bmatrix}
	0 & 1\\
	\cfrac{1}{2} & \cfrac{20}{12}
	\end{bmatrix};
	B_{q_6} = 
	\begin{bmatrix}
	0\\
	1
	\end{bmatrix};
	C_{q_6} = 
	\begin{bmatrix}
	- \cfrac{1}{30} & - \cfrac{1}{4}
	\end{bmatrix};\\
	&A_{q_7} = 
	\begin{bmatrix}
	0 & 1\\
	- \cfrac{1}{2} & - \cfrac{20}{12}
	\end{bmatrix};
	B_{q_7} = 
	\begin{bmatrix}
	0\\
	1
	\end{bmatrix};
	C_{q_7} = 
	\begin{bmatrix}
	0 & 0
	\end{bmatrix};
\end{align}

\subsection{Ранжирование параметров}

Сконструируем агрегированную систему с составным вектором $\tilde{x}_j = col\{x, \sigma_j\}$ размерности $ \dim \tilde{x} = 2n$, которая объединением \ref{eq_mts} и \ref{eq_iso_coc}, получает представление
\begin{align}\label{rang_system}
	\dot{\tilde{x}}_j (t) &= \tilde{A}_j \tilde{x}_j (t) + \tilde{B}_j u(t); 
	\tilde{x}_j (0) = col \{x(0), 0\}\\
	x(t) &= \tilde{C}_{x_j} \tilde{x}_j;\\
	y(t) &= \tilde{C}_j \tilde{x}_j (t);\\
	\sigma_j (t) &= \tilde{C}_{\sigma_j} \tilde{x}_j (t);\\
	\eta_j (t) &= \tilde{C}_{\eta_j} \tilde{x}_j (t)
\end{align}
где 
\begin{align*}
	&j = \overline{1, p},
	\tilde{A}_j =
	\begin{bmatrix}
	A & 0\\
	A_{q_j} & A
	\end{bmatrix},
	\tilde{B}_j = 
	\begin{bmatrix}
		B\\
		B_{q_j}
	\end{bmatrix},\\
	&\tilde{C}_{x_j} = 
	\begin{bmatrix}
		I_{n \times n} & O_{n \times n}
	\end{bmatrix},
	\tilde{C}_{j} = 
	\begin{bmatrix}
		C & 0_{m \times n}
	\end{bmatrix},
	\tilde{C}_{\sigma_{j}} = 
	\begin{bmatrix}
		0_{n \times n} & I_{n \times n}
	\end{bmatrix},
	\tilde{C}_{\eta_{j}} = 
	\begin{bmatrix}
		C_{q_j} & C
	\end{bmatrix}.	
\end{align*}

Составим необходимые матрицы

\begin{align*}
	&\tilde{A}_{1,2} =
	\begin{bmatrix}
		0 & 1 & 0 & 0\\
		- \cfrac{1}{2} & - \cfrac{236}{120} & 0 & 0\\
		0 & 1 & 0 & 1\\
		0 & 0 &- \cfrac{1}{2} & - \cfrac{236}{120}
	\end{bmatrix};
	\tilde{A}_3 =
	\begin{bmatrix}
		0 & 1 & 0 & 0\\
		- \cfrac{1}{2} & - \cfrac{236}{120} & 0 & 0\\
		0 & 1 & 0 & 1\\
		\cfrac{1}{2} & \cfrac{36}{120} &- \cfrac{1}{2} & - \cfrac{236}{120}
	\end{bmatrix};\\
	&\tilde{A}_4 =
	\begin{bmatrix}
		0 & 1 & 0 & 0\\
		- \cfrac{1}{2} & - \cfrac{236}{120} & 0 & 0\\
		0 & 1 & 0 & 1\\
		- \cfrac{1}{2} & - 3.6 &- \cfrac{1}{2} & - \cfrac{236}{120}
	\end{bmatrix};
	\tilde{A}_6 =
	\begin{bmatrix}
		0 & 1 & 0 & 0\\
		- \cfrac{1}{2} & - \cfrac{236}{120} & 0 & 0\\
		0 & 1 & 0 & 1\\
		\cfrac{1}{2} & \cfrac{20}{12} &- \cfrac{1}{2} & - \cfrac{236}{120}
	\end{bmatrix};\\
	&\tilde{A}_7 =
	\begin{bmatrix}
		0 & 1 & 0 & 0\\
		- \cfrac{1}{2} & - \cfrac{236}{120} & 0 & 0\\
		0 & 1 & 0 & 1\\
		- \cfrac{1}{2} & - \cfrac{20}{12} &- \cfrac{1}{2} & - \cfrac{236}{120}
	\end{bmatrix};
	\tilde{B}_{1,2,3,4,6,7} =
	\begin{bmatrix}
		0\\
		1\\
		0\\
		1
	\end{bmatrix};
\end{align*}

\begin{align*}
	&\tilde{C}_{x_{1,2,3,4,6,7}} =
	\begin{bmatrix}
		1 & 0 & 0 & 0\\
		0 & 1 & 0 & 0\\
	\end{bmatrix};
	\tilde{C}_{1,2,3,4,6,7} =
	\begin{bmatrix}
		\cfrac{1}{30} & \cfrac{1}{4} & 0 & 0\\
	\end{bmatrix};\\
	&\tilde{C}_{\sigma_{1,2,3,4,6,7}} =
	\begin{bmatrix}
		0 & 0 & 1 & 0\\
		0 & 0 & 0 & 1\\
	\end{bmatrix};\\
	&\tilde{C}_{\eta_1} =
	\begin{bmatrix}
		0 & \cfrac{1}{4} & \cfrac{1}{30} & \cfrac{1}{4}\\
	\end{bmatrix};
	\tilde{C}_{\eta_2} =
	\begin{bmatrix}
		\cfrac{1}{30} & 0 & \cfrac{1}{30} & \cfrac{1}{4}\\
	\end{bmatrix};\\
	&\tilde{C}_{\eta_3} =
	\begin{bmatrix}
		-\cfrac{1}{30} & - \cfrac{1}{4} & \cfrac{1}{30} & \cfrac{1}{4}\\
	\end{bmatrix};
	\tilde{C}_{\eta_4} =
	\begin{bmatrix}
		0 & 0 & \cfrac{1}{30} & \cfrac{1}{4}\\
	\end{bmatrix};\\
	&\tilde{C}_{\eta_6} =
	\begin{bmatrix}
		- \cfrac{1}{30} & - \cfrac{1}{4} & \cfrac{1}{30} & \cfrac{1}{4}\\
	\end{bmatrix};
	\tilde{C}_{\eta_7} =
	\begin{bmatrix}
		0 & 0 & \cfrac{1}{30} & \cfrac{1}{4}\\
	\end{bmatrix};
\end{align*}

Для ранжирования параметров по возможным затратам ресурсов управления для достижения нечувствительности траектории проектируемой системы к этим вариациям проведем анализ
управляемости системы~\ref{rang_system} по ее выходу~$\eta_j$.

Требования к ресурсам управления заметно снижаются, если изначально ограничиться задачей обеспечения траекторной нечувствительности выхода проектируемой системы. На уровне требований к структурным свойствам агрегированной системы~\ref{rang_system} задача сводится к контролю управляемости тройки матриц $(\tilde{C}_{\eta_j}, \tilde{A}_{j}, \tilde{B}_{j})$ и количественной оценке эффекта управления по переменной $\eta_j$ при приложении управления $u(t)$ фиксированной нормы с помощью сингулярных чисел матрицы управляемости
\begin{equation}
	\tilde{W}_{y \eta_j} =
	\begin{bmatrix}
		\tilde{C}_{\eta_j} \tilde{B}_{j} &
		\tilde{C}_{\eta_j} \tilde{A}_{j} \tilde{B}_{j} &
		\tilde{C}_{\eta_j} \tilde{A}_{j}^2 \tilde{B}_{j} &		
		\cdots &
		\tilde{C}_{\eta_j} \tilde{A}_{j}^{2n-1} \tilde{B}_{j}
	\end{bmatrix}
\end{equation}

С учетом $n = 2$, рассчитаем матрицы управляемости $\tilde{W}_{\eta_j}$
\begin{align*}
	&\tilde{W}_{y \eta_{1}} =
	\begin{bmatrix}
		0.5  & - 0.9166667  &  1.4277778 & - 2.120463   
	\end{bmatrix}, 
	\\
	&\tilde{W}_{y \eta_2} =
	\begin{bmatrix}
		0.25 &  - 0.3916667 &   0.5202778 & - 0.5982130  	
	\end{bmatrix}, 
	\\
	&\tilde{W}_{y \eta_3} =
	\begin{bmatrix}
		0   &   0.1083333 & - 0.3505556 &   0.8681759  
	\end{bmatrix}, 
	\\
	&\tilde{W}_{y \eta_4} =
	\begin{bmatrix}
		0.25 & - 1.325    &    3.8808333 & - 9.3064722  
	\end{bmatrix}, 
	\\
	&\tilde{W}_{y \eta_6} =
	\begin{bmatrix}
		0   &   0.45   &    - 1.6488889  &  4.3117963  
	\end{bmatrix}, 
	\\
	&\tilde{W}_{y \eta_7} =
	\begin{bmatrix}
		0.25 & - 0.8416667 &   2.0441667 & - 4.4350093
	\end{bmatrix}
\end{align*}

Вычислим для полученных матриц управляемости сингулярные числа
\begin{align}\label{singul_nums}
	&\alpha\{\tilde{W}_{y \eta_{1}}\} = 2.7613747,
	\alpha\{\tilde{W}_{y \eta_{2}}\} = 0.9189399,\\	
	&\alpha\{\tilde{W}_{y \eta_{3}}\} = 0.9425257,	
	\alpha\{\tilde{W}_{y \eta_{4}}\} = 10.172975,\\		
	&\alpha\{\tilde{W}_{y \eta_{6}}\} = 4.6382024,
	\alpha\{\tilde{W}_{y \eta_{7}}\} = 4.9617363
	\label{singul_nums_end}
\end{align}


%Ранжирование параметров $q_j$ осуществляется по значению сингулярных чисел матриц управляемости. 
%Чем эти числа меньше, тем большими по норме управлениями достигается асимптотическая траекторная нечувствительность компонента yj(t) вектора выхода y(t). Отсюда следует, что асимптотическая сходимость к нулю дополнительного движения будет требовать все меньшего количества затрат при следующем расположении qj : q6, q7, q2, q4, q5.

Ранги матриц $\tilde{W}_{\eta_j}$ равны $rang(\tilde{W}_{\eta_j}) = 1$, что совпадает с размерностью $m = 1$ вектора выхода. Таким образом, выбором закона
управления можно обеспечить сходимость $\lim_{t  \to \infty} \Delta y (t,q_0,\Delta q_j) = 0; j = \overline{1, p}$ с заданным темпом~\cite{NSUsh}. 
Сингулярные числа матриц $\tilde{W}_{\eta_j}$ принимают значения~\ref{singul_nums}--\ref{singul_nums_end}. Проранжируем параметры $q_j$ в порядке увеличения затрат ресурсов на управление
\begin{enumerate}
	\item $q_4$
	\item $q_7$
	\item $q_6$
	\item $q_1$
	\item $q_3$
	\item $q_2$			
\end{enumerate}

Отсюда следует, что асимптотическая сходимость к нулю дополнительного движения $\Delta y (t,q_0,\Delta q_2)$ будет требовать наибольших затрат на управление, чем сходимость остальных дополнительных движений, с тем же темпом.

\newpage
\section{Построение МТЧ ДОУ и результаты ее исследования}

\begin{enumerate}
	\item Перейти к дискретному описанию ОУ с помощью интегральной модели ВСВ НОУ;
	\item Построить модель траекторной чувствительности (МТЧ) дискретного ОУ (ДОУ) к вариации интервала дискретности;
\end{enumerate}


\subsection{Переход к дискретному описанию ОУ}

ДОУ представляет собой дискретную по времени с интервалом дискретности
длительности $\Delta t$ выборку из непрерывных процессов по вектору
состояния $x(t,q)$ и выходу $y(t,q)$ при фиксированном на интервале $t \in \left[\Delta t k, \Delta t(k+1)\right]$ значении управления $u(t) = u(\Delta t k) = u(k)$. Имеет следующий вид
\begin{equation}\label{eq_iso_doc}
	\begin{cases}
		x(k+1, q) = \overline{A}(q) x(k, q) + \overline{B}(q) u(k)\\
		y(k, q) = \overline{C}(q) x(k, q)
	\end{cases}
\end{equation}
где матрицы непрерывного~\ref{eq_iso_coc} и дискретного~\ref{eq_iso_doc} ОУ связаны следующими функциональными соотношениями
\begin{align}
	\overline{A}(q) = e^{A(q) \Delta t}; \overline{B}(q) = A^{-1}(q)(e^{A(q) \Delta t} - I)B(q); \overline{C}(q) = C(q)
\end{align}


Номинальная модель ДОУ получается из~\ref{eq_iso_doc} при векторе параметров $q=q_0$
\begin{equation}\label{eq_iso_doc_n}
	\begin{cases}
		x(k+1) = \overline{A} x(k) + \overline{B} u(k)\\
		y(k) = \overline{C} x(k)
	\end{cases}
\end{equation}

Общий вид интегральной модели~\cite{MIROSH} ВСВ НОУ имеет вид
\begin{align}
	&x(t) = \Phi (t) x(0) + \int_0^t \Phi (t, \tau) B u(\tau) d \tau \\
	&y(t) = C \Phi (t) x(0) + \int_0^t C \Phi(t, \tau) B u(\tau) d \tau 
\end{align}
где $\Phi(t) = e^{At}, \Phi(t, \tau) = \Phi(t) \Phi^{-1}(\tau) = e^{A(t-\tau)}$.

Используя интегральную запись модели ВСВ непрерывного динамического объекта, нетрудно получить связь между матрицами модели ВСВ дискретного и непрерывного объектов в форме
\begin{align}
	\overline{A} = \Phi (\Delta t) = e^{A \Delta t},
	\overline{B} = \Phi (\Delta t) \int_0^{\Delta t} \Phi^{-1} (\tau) d \tau B,
	\overline{C} = C
\end{align}

И окончательные формулы для перехода
\begin{align}
	\overline{A} = e^{A \Delta t},
	\overline{B} = A^{-1} (e^{A \Delta t} - I)B,
	\overline{C} = C  
\end{align}

При $\Delta t = 0.03$c, рассчитаем матрицы модели ВСВ ДОУ
\begin{align*}
	\overline{A} =
	\begin{bmatrix}
		0.9997794 &   0.0291300\\  
		- 0.0145650  &  0.9424904 
	\end{bmatrix};
	\overline{B} =
	\begin{bmatrix}
		0.0004413 \\
		0.0291300 
	\end{bmatrix};
	\overline{C} =
	\begin{bmatrix}
		 0.0333333  &  0.25 
	\end{bmatrix};
\end{align*}

\subsection{Построение МТЧ ДОУ к вариации интервала дискретности}

Модель траекторной чувствительности, необходимая для генерирования функций траекторной чувствительности $\sigma(k)$ и $\eta(k)$ по состоянию и выходу ДОУ, строится путем дифференцирования компонентов представления~\ref{eq_iso_doc} по компонентам $q_j$ вектора параметров $q$ при его номинальном значении (в нашем случае $q = \Delta t$), в результате чего для МТЧ получаем

\begin{equation}\label{eq_mts_d}
	\begin{cases}
		\sigma(k+1) = \overline{A} \sigma(k) + \overline{A}_{q} x(k) + \overline{B}_{q} u(k)\\
		\eta (k) = \overline{C} \sigma (k) + \overline{C}_{q} x(k)
	\end{cases}
\end{equation}
где 
\begin{align*}
	&\overline{A}_{q} = \cfrac{\partial{\overline{A}(q)}}{\partial{\Delta t}} \bigg|_{q=q_0};
	\overline{B}_{q} = \cfrac{\partial{\overline{B}(q)}}{\partial{\Delta t}} \bigg|_{q=q_0};
	\overline{C}_{q} = \cfrac{\partial{\overline{C}(q)}}{\partial{\Delta t}} \bigg|_{q=q_0};\\
	&\sigma(t) = \cfrac{\partial{x(k,q)}}{\partial{\Delta t}} \bigg|_{q=q_0};
	\eta(t) = \cfrac{\partial{y(k,q)}}{\partial{\Delta t}} \bigg|_{q=q_0};\\
	&\cfrac{\partial{\overline{A}(q)}}{\partial{\Delta t}} = 
	\cfrac{\partial{\left(e^{A(q) \Delta t}\right)}}{\partial{\Delta t}} =
	A(q) e^{A(q) \Delta t} =  e^{A(q) \Delta t} A(q)\ = \overline{A}(q) A(q);\\
	&\cfrac{\partial{\overline{B}(q)}}{\partial{\Delta t}} = 
	\cfrac{\partial{}}{\partial{\Delta t}} \left[A^{-1}(q)(e^{A(q) \Delta t} - I)B(q)\right] = 
	A^{-1}(q) A(q) e^{A(q) \Delta t} B = \overline{A}(q) B(q);\\
	&\cfrac{\partial{\overline{C}(q)}}{\partial{\Delta t}} = 
	\cfrac{\partial{C(q)}}{\partial{\Delta t}} = 0.
\end{align*}

Используя полученные выражения вычислим матрицы МТЧ ДОУ
\begin{align*}
	\overline{A}_q =
	\begin{bmatrix}
		- 0.0145650 &   0.9424904\\  
		- 0.4712452 & - 1.8681295  
	\end{bmatrix};
	\overline{B}_q =
	\begin{bmatrix}
		0.0291300\\
		0.9424904  
	\end{bmatrix};
	\overline{C}_q =
	\begin{bmatrix}
		0  &  0 
	\end{bmatrix};
\end{align*}

Сконструируем агрегированную систему с составным вектором \newline $\tilde{x}=col\{x,\sigma\}$ размерности $ \dim \tilde{x} = 2n$, которая объединением \ref{eq_iso_doc_n} и \ref{eq_mts_d}, получает представление
\begin{align}\label{rang_system_d_3}
	\tilde{x}(k+1) &= \tilde{\overline{A}} \tilde{x} (k) + \tilde{\overline{B}} u(k); 
	\tilde{x} (0) = col \{x(0), 0\}\\
	x(k) &= \tilde{\overline{C}}_{x_j} \tilde{x}(k);\\
	y(k) &= \tilde{\overline{C}} \tilde{x} (k);\\
	\sigma (k) &= \tilde{\overline{C}}_{\sigma} \tilde{x} (k);\\
	\eta (k) &= \tilde{\overline{C}}_{\eta} \tilde{x} (k)
\end{align}
где 
\begin{align*}
	&\tilde{\overline{A}} =
	\begin{bmatrix}
	\overline{A} & 0\\
	\overline{A}_{q} & \overline{A}
	\end{bmatrix},
	\tilde{\overline{B}} = 
	\begin{bmatrix}
	\overline{B}\\
	\overline{B}_{q}
	\end{bmatrix},\\
	&\tilde{\overline{C}}_{x} = 
	\begin{bmatrix}
	I_{n \times n} & O_{n \times n}
	\end{bmatrix},
	\tilde{\overline{C}} = 
	\begin{bmatrix}
	\overline{C} & 0_{m \times n}
	\end{bmatrix},
	\tilde{\overline{C}}_{\sigma} = 
	\begin{bmatrix}
	0_{n \times n} & I_{n \times n}
	\end{bmatrix},
	\tilde{\overline{C}}_{\eta} = 
	\begin{bmatrix}
	\overline{C}_{q} & \overline{C}
	\end{bmatrix}.	
\end{align*}

Составим необходимые матрицы

\begin{align*}
	&\tilde{\overline{A}} =
	\begin{bmatrix}
    0.9997794  &  0.0291300 &0& 0 \\
	- 0.0145650  &  0.9424904  &0& 0\\
	- 0.0145650  &  0.9424904 &  0.9997794  &  0.0291300\\  
	- 0.4712452 & - 1.8681295  & - 0.0145650  &  0.9424904  
	\end{bmatrix};\\
	&\tilde{\overline{B}} =
	\begin{bmatrix}
       0.0004413\\  
		0.0291300 \\ 
		0.0291300  \\
		0.9424904 
	\end{bmatrix};
	\tilde{\overline{C}}_{\eta} =
	\begin{bmatrix}
		0&0&0.0333333 &   0.25\\
	\end{bmatrix}
\end{align*}

Проверим управляемость агрегированной системы по выходу $\eta(k)$ с помощью матрицы управляемости $\tilde{\overline{W}}_{y \eta}$
\begin{equation}
	\tilde{\overline{W}}_{y \eta} =
	\begin{bmatrix}
	\tilde{\overline{C}}_{\eta} \tilde{\overline{B}} &
	\tilde{\overline{C}}_{\eta} \tilde{\overline{A}} \tilde{\overline{B}} &
	\tilde{\overline{C}}_{\eta} \tilde{\overline{A}}^2 \tilde{\overline{B}} &		
	\cdots &
	\tilde{\overline{C}}_{\eta} \tilde{\overline{A}}^{2n-1} \tilde{\overline{B}}
	\end{bmatrix}
\end{equation}
которая с учетом $n=2$ имеет реализацию
\begin{equation*}
	\tilde{\overline{W}}_{y \eta} =
	\begin{bmatrix}
	0.2365936  &  0.2111102  &  0.1875234  &  0.1657095  
	\end{bmatrix}
\end{equation*}

Ранги матриц $\tilde{W}_{\eta}$ равны $rang(\tilde{W}_{\eta}) = 1$, что совпадает с размерностью $m = 1$ вектора выхода. Таким образом, выбором закона
управления можно обеспечить сходимость $\lim_{t  \to \infty} \Delta y (t,q_0,\Delta t) = 0$ с заданным темпом. 



\newpage
%\section*{Заключение}
\addcontentsline{toc}{section}{Заключение}
Текст заключения
\newpage
\renewcommand\refname{Список использованных источников}
\providecommand*{\url}[1]{#1} %нужно для описания некоторых источников
\bibliography{used_books}
%\ESKDappendix{обязательное}{Название приложения}\label{append_app_example}
Текст приложения
\end{document}