\section{Построение МТЧ НОУ}\label{problem_1}

\begin{enumerate}
	\item Записать непрерывный ОУ (НОУ) в форме <<вход-состояние-выход (ВСВ)>> в требуемом базисе;
	\item Построить модель траекторной чувствительности (МТЧ) НОУ;
	\item Произвести ранжирование параметров  по потенциальной чувствительности к ним выхода ОУ с использованием матрицы управляемости агрегированной системы; 
	
	Оценить, какое из дополнительных движений, вызванных вариацией, потребует максимальных затрат управления при обеспечении его асимптотической сходимости к нулю.
\end{enumerate}


\subsection{Непрерывный ОУ в форме ВСВ}

Заданный ОУ описывается ПФ
\begin{equation}\label{eq_OU}
\Phi (s, q) = \cfrac{3 (1 + q_1) s + 0.4 (1 + q_2)}{(2 (1+q_3)s + 0.6 (1+q_4))(6 (1 + q_6) s + 10 (1 + q_7))}
\end{equation}

Для составления векторно-матричного описания ОУ запишем ПФ в форме
\begin{equation*}\label{eq_pf_coc}
\Phi (s, q) = \cfrac{\cfrac{(1 + q_1)}{4(1+q_3)(1+q_6)} s + \cfrac{(1 + q_2)}{30(1+q_3)(1+q_6)}}{s^2 + \cfrac{20(1+q_3)(1+q_7)+3.6(1+q_4)(1+q_6)}{12(1+q_3)(1+q_6)} s + \cfrac{(1+q_4)(1+q_7)}{2(1+q_3)(1+q_6)}}
\end{equation*}

В каноническом управляемом базисе векторно-матричное представление ОУ принимает вид:
\begin{equation}\label{eq_iso_coc}
\begin{cases}
\dot x(t,q) &= A(q) x(t,q) + B u(t)\\
y(t,q) &= C(q) x(t,q)
\end{cases}
\end{equation}
в котором
\begin{equation}
	A(q) =
	\begin{bmatrix}
		0 & 1 \\
		- \cfrac{(1+q_4)(1+q_7)}{2(1+q_3)(1+q_6)} & - \cfrac{20(1+q_3)(1+q_7)+3.6(1+q_4)(1+q_6)}{12(1+q_3)(1+q_6)}
	\end{bmatrix}
\end{equation}
\begin{equation}
	B =
	\begin{bmatrix}
		0\\
		1
	\end{bmatrix}
\end{equation}

\begin{equation}
	C(q) =
	\begin{bmatrix}
	\cfrac{(1 + q_2)}{30(1+q_3)(1+q_6)} & \cfrac{(1 + q_1)}{4(1+q_3)(1+q_6)} 
	\end{bmatrix}
\end{equation}

\subsection{Модель траекторной чувствительности НОУ}

ПФ номинального ОУ, когда параметры $q_{j} = 0, j = \overline{1,7}$, представляет собой
\begin{equation}
	\Phi(s, 0) = \cfrac{\cfrac{1}{4} s + \cfrac{1}{30}}{s^2 + \cfrac{236}{120} s + \cfrac{1}{2}} 
\end{equation}

Матрицы модели ВСВ номинального ОУ имеют реализации
\begin{align*}
A =
\begin{bmatrix}
0 & 1 \\
- \cfrac{1}{2} & - \cfrac{236}{120}
\end{bmatrix};
B =
\begin{bmatrix}
0\\
1
\end{bmatrix};
C =
\begin{bmatrix}
\cfrac{1}{30} & \cfrac{1}{4}
\end{bmatrix}
\end{align*}

Введем обозначения
\begin{align*}
	&A_{q_j} = \cfrac{\partial{A(q)}}{\partial{q_j}} \bigg|_{q=q_0};
	B_{q_j} = \cfrac{\partial{B(q)}}{\partial{q_j}} \bigg|_{q=q_0};
	C_{q_j} = \cfrac{\partial{C(q)}}{\partial{q_j}} \bigg|_{q=q_0};\\
	&A(q)|_{q=q_0} = A;
	B(q)|_{q=q_0} = B;
	C(q)|_{q=q_0} = C;\\
	&x(t,q)|_{q=q_0} = x(t);
	y(t,q)|_{q=q_0} = y(t);\\
	&\cfrac{\partial{x(t,q)}}{\partial{q_j}} \bigg|_{q=q_0} = \sigma_j(t);
	\cfrac{\partial{y(t,q)}}{\partial{q_j}} \bigg|_{q=q_0} = \eta_j(t);
\end{align*}

Теперь для $j$-й модели траекторной чувствительности получим представление МТЧ
\begin{equation}\label{eq_mts}
	\begin{cases}
		\dot \sigma_j(t) &= A \sigma_j(t) + A_{q_j} x(t) + B_{q_j} u(t); 
		\sigma_j (0) = 0\\
		\eta_j (t) &= C \sigma_j (t) + C_{q_j} x(t)
	\end{cases}
\end{equation}

МТЧ будет генерировать функции траекторной чувствительности $\sigma_j (t)$ по состоянию и $\eta_j (t)$ по выходу, если ее дополнить моделью номинального ОУ~\ref{eq_iso_coc}.

На состояние заданного ОУ влияют $p = 6$ (далее, под записью $j = \overline{1, p}$ будет подразумеваться, что $j = 1,2,3,4,6,7$) параметров: $q_1, q_2, q_3, q_4, q_6, q_7$. Вычислим матрицы моделей траекторной чувствительности
\begin{align}
	&A_{q_1} = 
	\begin{bmatrix}
		0 & 1\\
		0 & 0
	\end{bmatrix};
	B_{q_1} = 
	\begin{bmatrix}
		0\\
		1
	\end{bmatrix};
	C_{q_1} = 
	\begin{bmatrix}
		0 & \cfrac{1}{4}
	\end{bmatrix};\\
	&A_{q_2} = 
	\begin{bmatrix}
	0 & 1\\
	0 & 0
	\end{bmatrix};
	B_{q_2} = 
	\begin{bmatrix}
	0\\
	1
	\end{bmatrix};
	C_{q_2} = 
	\begin{bmatrix}
	\cfrac{1}{30} & 0
	\end{bmatrix};\\
	&A_{q_3} = 
	\begin{bmatrix}
	0 & 1\\
	\cfrac{1}{2} & \cfrac{36}{120}
	\end{bmatrix};
	B_{q_3} = 
	\begin{bmatrix}
	0\\
	1
	\end{bmatrix};
	C_{q_3} = 
	\begin{bmatrix}
	- \cfrac{1}{30} & - \cfrac{1}{4}
	\end{bmatrix};\\
	&A_{q_4} = 
	\begin{bmatrix}
	0 & 1\\
	- \cfrac{1}{2} & - 3.6
	\end{bmatrix};
	B_{q_4} = 
	\begin{bmatrix}
	0\\
	1
	\end{bmatrix};
	C_{q_4} = 
	\begin{bmatrix}
	0 & 0
	\end{bmatrix};\\
	&A_{q_6} = 
	\begin{bmatrix}
	0 & 1\\
	\cfrac{1}{2} & \cfrac{20}{12}
	\end{bmatrix};
	B_{q_6} = 
	\begin{bmatrix}
	0\\
	1
	\end{bmatrix};
	C_{q_6} = 
	\begin{bmatrix}
	- \cfrac{1}{30} & - \cfrac{1}{4}
	\end{bmatrix};\\
	&A_{q_7} = 
	\begin{bmatrix}
	0 & 1\\
	- \cfrac{1}{2} & - \cfrac{20}{12}
	\end{bmatrix};
	B_{q_7} = 
	\begin{bmatrix}
	0\\
	1
	\end{bmatrix};
	C_{q_7} = 
	\begin{bmatrix}
	0 & 0
	\end{bmatrix};
\end{align}

\subsection{Ранжирование параметров}

Сконструируем агрегированную систему с составным вектором $\tilde{x}_j = col\{x, \sigma_j\}$ размерности $ \dim \tilde{x} = 2n$, которая объединением \ref{eq_mts} и \ref{eq_iso_coc}, получает представление
\begin{align}\label{rang_system}
	\dot{\tilde{x}}_j (t) &= \tilde{A}_j \tilde{x}_j (t) + \tilde{B}_j u(t); 
	\tilde{x}_j (0) = col \{x(0), 0\}\\
	x(t) &= \tilde{C}_{x_j} \tilde{x}_j;\\
	y(t) &= \tilde{C}_j \tilde{x}_j (t);\\
	\sigma_j (t) &= \tilde{C}_{\sigma_j} \tilde{x}_j (t);\\
	\eta_j (t) &= \tilde{C}_{\eta_j} \tilde{x}_j (t)
\end{align}
где 
\begin{align*}
	&j = \overline{1, p},
	\tilde{A}_j =
	\begin{bmatrix}
	A & 0\\
	A_{q_j} & A
	\end{bmatrix},
	\tilde{B}_j = 
	\begin{bmatrix}
		B\\
		B_{q_j}
	\end{bmatrix},\\
	&\tilde{C}_{x_j} = 
	\begin{bmatrix}
		I_{n \times n} & O_{n \times n}
	\end{bmatrix},
	\tilde{C}_{j} = 
	\begin{bmatrix}
		C & 0_{m \times n}
	\end{bmatrix},
	\tilde{C}_{\sigma_{j}} = 
	\begin{bmatrix}
		0_{n \times n} & I_{n \times n}
	\end{bmatrix},
	\tilde{C}_{\eta_{j}} = 
	\begin{bmatrix}
		C_{q_j} & C
	\end{bmatrix}.	
\end{align*}

Составим необходимые матрицы

\begin{align*}
	&\tilde{A}_{1,2} =
	\begin{bmatrix}
		0 & 1 & 0 & 0\\
		- \cfrac{1}{2} & - \cfrac{236}{120} & 0 & 0\\
		0 & 1 & 0 & 1\\
		0 & 0 &- \cfrac{1}{2} & - \cfrac{236}{120}
	\end{bmatrix};
	\tilde{A}_3 =
	\begin{bmatrix}
		0 & 1 & 0 & 0\\
		- \cfrac{1}{2} & - \cfrac{236}{120} & 0 & 0\\
		0 & 1 & 0 & 1\\
		\cfrac{1}{2} & \cfrac{36}{120} &- \cfrac{1}{2} & - \cfrac{236}{120}
	\end{bmatrix};\\
	&\tilde{A}_4 =
	\begin{bmatrix}
		0 & 1 & 0 & 0\\
		- \cfrac{1}{2} & - \cfrac{236}{120} & 0 & 0\\
		0 & 1 & 0 & 1\\
		- \cfrac{1}{2} & - 3.6 &- \cfrac{1}{2} & - \cfrac{236}{120}
	\end{bmatrix};
	\tilde{A}_6 =
	\begin{bmatrix}
		0 & 1 & 0 & 0\\
		- \cfrac{1}{2} & - \cfrac{236}{120} & 0 & 0\\
		0 & 1 & 0 & 1\\
		\cfrac{1}{2} & \cfrac{20}{12} &- \cfrac{1}{2} & - \cfrac{236}{120}
	\end{bmatrix};\\
	&\tilde{A}_7 =
	\begin{bmatrix}
		0 & 1 & 0 & 0\\
		- \cfrac{1}{2} & - \cfrac{236}{120} & 0 & 0\\
		0 & 1 & 0 & 1\\
		- \cfrac{1}{2} & - \cfrac{20}{12} &- \cfrac{1}{2} & - \cfrac{236}{120}
	\end{bmatrix};
	\tilde{B}_{1,2,3,4,6,7} =
	\begin{bmatrix}
		0\\
		1\\
		0\\
		1
	\end{bmatrix};
\end{align*}

\begin{align*}
	&\tilde{C}_{x_{1,2,3,4,6,7}} =
	\begin{bmatrix}
		1 & 0 & 0 & 0\\
		0 & 1 & 0 & 0\\
	\end{bmatrix};
	\tilde{C}_{1,2,3,4,6,7} =
	\begin{bmatrix}
		\cfrac{1}{30} & \cfrac{1}{4} & 0 & 0\\
	\end{bmatrix};\\
	&\tilde{C}_{\sigma_{1,2,3,4,6,7}} =
	\begin{bmatrix}
		0 & 0 & 1 & 0\\
		0 & 0 & 0 & 1\\
	\end{bmatrix};\\
	&\tilde{C}_{\eta_1} =
	\begin{bmatrix}
		0 & \cfrac{1}{4} & \cfrac{1}{30} & \cfrac{1}{4}\\
	\end{bmatrix};
	\tilde{C}_{\eta_2} =
	\begin{bmatrix}
		\cfrac{1}{30} & 0 & \cfrac{1}{30} & \cfrac{1}{4}\\
	\end{bmatrix};\\
	&\tilde{C}_{\eta_3} =
	\begin{bmatrix}
		-\cfrac{1}{30} & - \cfrac{1}{4} & \cfrac{1}{30} & \cfrac{1}{4}\\
	\end{bmatrix};
	\tilde{C}_{\eta_4} =
	\begin{bmatrix}
		0 & 0 & \cfrac{1}{30} & \cfrac{1}{4}\\
	\end{bmatrix};\\
	&\tilde{C}_{\eta_6} =
	\begin{bmatrix}
		- \cfrac{1}{30} & - \cfrac{1}{4} & \cfrac{1}{30} & \cfrac{1}{4}\\
	\end{bmatrix};
	\tilde{C}_{\eta_7} =
	\begin{bmatrix}
		0 & 0 & \cfrac{1}{30} & \cfrac{1}{4}\\
	\end{bmatrix};
\end{align*}

Для ранжирования параметров по возможным затратам ресурсов управления для достижения нечувствительности траектории проектируемой системы к этим вариациям проведем анализ
управляемости системы~\ref{rang_system} по ее выходу~$\eta_j$.

Требования к ресурсам управления заметно снижаются, если изначально ограничиться задачей обеспечения траекторной нечувствительности выхода проектируемой системы. На уровне требований к структурным свойствам агрегированной системы~\ref{rang_system} задача сводится к контролю управляемости тройки матриц $(\tilde{C}_{\eta_j}, \tilde{A}_{j}, \tilde{B}_{j})$ и количественной оценке эффекта управления по переменной $\eta_j$ при приложении управления $u(t)$ фиксированной нормы с помощью сингулярных чисел матрицы управляемости
\begin{equation}
	\tilde{W}_{y \eta_j} =
	\begin{bmatrix}
		\tilde{C}_{\eta_j} \tilde{B}_{j} &
		\tilde{C}_{\eta_j} \tilde{A}_{j} \tilde{B}_{j} &
		\tilde{C}_{\eta_j} \tilde{A}_{j}^2 \tilde{B}_{j} &		
		\cdots &
		\tilde{C}_{\eta_j} \tilde{A}_{j}^{2n-1} \tilde{B}_{j}
	\end{bmatrix}
\end{equation}

С учетом $n = 2$, рассчитаем матрицы управляемости $\tilde{W}_{\eta_j}$
\begin{align}
	&\tilde{W}_{y \eta_{1}} =
	\begin{bmatrix}
		0.5  & - 0.9166667  &  1.4277778 & - 2.120463   
	\end{bmatrix}, 
	\\
	&\tilde{W}_{y \eta_2} =
	\begin{bmatrix}
		0.25 &  - 0.3916667 &   0.5202778 & - 0.5982130  	
	\end{bmatrix}, 
	\\
	&\tilde{W}_{y \eta_3} =
	\begin{bmatrix}
		0   &   0.1083333 & - 0.3505556 &   0.8681759  
	\end{bmatrix}, 
	\\
	&\tilde{W}_{y \eta_4} =
	\begin{bmatrix}
		0.25 & - 1.325    &    3.8808333 & - 9.3064722  
	\end{bmatrix}, 
	\\
	&\tilde{W}_{y \eta_6} =
	\begin{bmatrix}
		0   &   0.45   &    - 1.6488889  &  4.3117963  
	\end{bmatrix}, 
	\\
	&\tilde{W}_{y \eta_7} =
	\begin{bmatrix}
		0.25 & - 0.8416667 &   2.0441667 & - 4.4350093
	\end{bmatrix}
\end{align}

Вычислим для полученных матриц управляемости сингулярные числа
\begin{align}\label{singul_nums}
	&\alpha\{\tilde{W}_{y \eta_{1}}\} = 2.7613747,
	\alpha\{\tilde{W}_{y \eta_{2}}\} = 0.9189399,\\	
	&\alpha\{\tilde{W}_{y \eta_{3}}\} = 0.9425257,	
	\alpha\{\tilde{W}_{y \eta_{4}}\} = 10.172975,\\		
	&\alpha\{\tilde{W}_{y \eta_{6}}\} = 4.6382024,
	\alpha\{\tilde{W}_{y \eta_{7}}\} = 4.9617363
	\label{singul_nums_end}
\end{align}


%Ранжирование параметров $q_j$ осуществляется по значению сингулярных чисел матриц управляемости. 
%Чем эти числа меньше, тем большими по норме управлениями достигается асимптотическая траекторная нечувствительность компонента yj(t) вектора выхода y(t). Отсюда следует, что асимптотическая сходимость к нулю дополнительного движения будет требовать все меньшего количества затрат при следующем расположении qj : q6, q7, q2, q4, q5.

Ранги матриц $\tilde{W}_{\eta_j}$ равны $rang(\tilde{W}_{\eta_j}) = 1$, что совпадает с размерностью $m = 1$ вектора выхода. Таким образом, выбором закона
управления можно обеспечить сходимость $\lim_{t  \to \infty} \Delta y (t,q_0,\Delta q_j) = 0; j = \overline{1, p}$ с заданным темпом. 
Сингулярные числа матриц $\tilde{W}_{\eta_j}$ принимают значения~\ref{singul_nums}--\ref{singul_nums_end}. Проранжируем параметры $q_j$ в порядке увеличения затрат ресурсов на управление
\begin{enumerate}
	\item $q_4$
	\item $q_7$
	\item $q_6$
	\item $q_1$
	\item $q_3$
	\item $q_2$			
\end{enumerate}

Отсюда следует, что асимптотическая сходимость к нулю дополнительного движения $\Delta y (t,q_0,\Delta q_2)$ будет требовать наибольших затрат на управление, чем сходимость остальных дополнительных движений, с тем же темпом.

\newpage