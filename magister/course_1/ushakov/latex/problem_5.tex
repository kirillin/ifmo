\section{Получение ВМО НОУ с интервальными параметрами}

\begin{enumerate}
	\item Получение векторно-матричное описание (ВМО) НОУ с интервальными параметрами 	c использованием интервальной арифметики на основе интервальной реализации параметров $q_j$, записываемых в форме $[q_j] = [\underline{q_j}, \overline{q_j}]$ при заданных граничных (угловых) значениях $[q_j] = [\underline{-0.2}, \overline{0.2}]$.
	\begin{equation}\label{VMO}
		\begin{cases}
			\dot x (t)  = [A] x(t) + [B] u(t);\\
			y(t) = [C] x(t)			
		\end{cases}
	\end{equation}
	\begin{equation}\label{interv}
		[A] = A_0 + [\Delta A], [B] = B_0 + [\Delta B], [C] = C_0 + [\Delta C]
	\end{equation}
	где $[\Delta A] = [\underline{\Delta A}, \overline{\Delta A}]$~--- интервальный матричный компонент матрицы $[A]$, $A_0 = mid[A]$~--- медиана матрицы $[A]$,  $[\Delta C] = [\underline{\Delta C}, \overline{\Delta C}]$~--- интервальный матричный компонент матрицы $[C]$, $C_0 = mid[C]$~--- медиана матрицы $[C]$, матрица $[B]$~--- в случае НОУ~\ref{eq_iso_coc} не зависит от вектора параметров $q$.

\end{enumerate}

\subsection{Построение векторно-матричное описание НОУ}

Используя~\ref{Aq} и \ref{Cq} и интервальную арифметику, найдем матрицы $A(q)$ и $C(q)$ при угловых значениях параметра $[q_j] = [\underline{q_j}, \overline{q_j}] = [\underline{-0.2}, \overline{0.2}]$
\begin{align}\label{Aqu}
	\underline{A} =
	\begin{bmatrix}
		0 & 1 \\
		-1.115 & -4.425
	\end{bmatrix};
	\overline{A} =
	\begin{bmatrix}
			0 & 1 \\	
			-0.22 & -0.874
	\end{bmatrix};\\
	\underline{C} =
	\begin{bmatrix}
		0.0058 & 0.1389
	\end{bmatrix};
	\overline{C} =
	\begin{bmatrix}
		0.0625 & 0.4688
	\end{bmatrix}.
\end{align}

Теперь, в соответствии с~\ref{interv}, необходимо определить матрицы $A_0, C_0$ и $[\Delta A], [\Delta C]$
\begin{align}\label{midA}
	A_0 =& 0.5 (\underline{A} + \overline{A}) =
	\begin{bmatrix}
		0&1\\
		- 0.6736& -2.6495
	\end{bmatrix}
\end{align}
\begin{align}
	C_0 =& 0.5 (\underline{C} + \overline{C}) = 
	\begin{bmatrix}
		0.0405 & 0.3038
	\end{bmatrix}
\end{align}

\begin{align}\label{widA}
	[\Delta A] = [\underline{\Delta A}, \overline{\Delta A}] = 
	\begin{bmatrix}
		0&0\\
	[- 0.4514, 0.4514] & [-1.7755, 1.7755]
	\end{bmatrix}
\end{align}

\begin{align}
	[\Delta C] = [\underline{\Delta C}, \overline{\Delta C}] = 
	\begin{bmatrix}
			[-0.02199, 0.02199] & [-0.1649, 0.1649]
	\end{bmatrix}
\end{align}

Запишем выражения для матриц $[A]$ и $[C]$
\begin{equation}
	[A] =
	\begin{bmatrix}
		0&1\\
		- 0.6736& -2.6495
	\end{bmatrix} 
	+
	\begin{bmatrix}
		0&0\\
		[- 0.4514, 0.4514] & [-1.7755, 1.7755]
	\end{bmatrix}	
\end{equation}

\begin{equation}
	[C] =
	\begin{bmatrix}
		0.0405 & 0.3038
	\end{bmatrix}
	+
	\begin{bmatrix}
		[-0.02199, 0.02199] & [-0.1649, 0.1649]
	\end{bmatrix}	
\end{equation}




\newpage