\newpage
\section{Заключение}

В этой расчетно-исследовательской работе мною были рассмотрены некоторые вопросы управления в условиях неопределенности с использованием возможностей неадаптивных и адаптивных алгоритмов применительно к непрерывным объектам.

В первой части была построена и исследована модель траекторной чувствительности неопределенного объекта управления.
Во второй части была построена и исследована модель траекторной чувствительности дискретного объекта управления.
В третьей части был синтезирован закон управления для непрерывного объекта управления и произведены исследования с применением аппарата траекторной чувствительности.
В четвертой части была составлена матрица функций модальной чувствительности и выделены неблагоприятные сочетания вариаций параметров.
В пятой части работы с использованием интервальной арифметики было получено интервальное представление непрерывного объекта управления.
В шестой части был синтезирован закон медианного модального управления непрерывным объектом управления, были исследованы свойства робастности полученной системы с помощью метода В.Л. Харитонова.
В седьмой части был синтезирован адаптивный закон управления.

\newpage 