\section{Построение медианного МУ НОУ и оценка его результатов}

\begin{enumerate}
	\item *Представить ОУ~\ref{eq_iso_coc} в базисе, в котором 	неопределенность физических параметров представлена 	неопределенностью значений только матрицы состояния в	форме матричного компонента $\Delta A$;  
	\item Синтезировать закон медианного модального управления, базовый алгоритм которого дополняется контролем нормы  медианной составляющей интервальной матрицы  спроектированной системы с последующим вычислением оценки, вычислить матрицы $K_g$ и $K$. 
	Закон управления (ЗУ) вида $u(t) = K_g g(t)-K x(t)$ должен доставлять системе
	\begin{equation}
		\begin{cases}
			\dot x (t) = [F] x(t) + G g(t);\\
			y(t) = C x(t)
		\end{cases}
	\end{equation}
	образованной объединением НОУ и ЗУ равенство входа $g(t)$ и выхода $y(t)$ в неподвижном состоянии при номинальных значениях параметров с помощью:
	\begin{enumerate}
		\item матрицы $K_g$ прямой связи по входу $g(t)$;
		\item матрыцы $K$ обратной связи по состоянию $x(t)$
	\end{enumerate}
	распределение мод Баттерворта с характеристической частотой $\omega_0 = 3c^{-1}$, которая гарантирует достижение значение оценки относительной интервальности матрицы состояния системы $\delta_I = \cfrac{||\Delta F||}{||F_0||}$ не больше заданной $\delta_{IR}F = 0.02$; 
	\item Исследовать свойство робастности системы, полученной в п.1, с помощью метода В.Л. Харитонова.
\end{enumerate}

\subsection{Получение ВМО НОУ с интервальными параметрами толкьо матрицы состояния}

Модельная параметрическая неопределенность может быть представлена неопределенностью (интервальностью) задания только матрицы состояния объекта управления. Таким образом, объект управления с интервальными параметрами задается векторно-матричной моделью
\begin{equation}\label{eq:mmc}
	\begin{cases}
		\dot x (t) = [A] x(t) + B u(t);\\
		y(t) = C x(t)
	\end{cases}
\end{equation}
где $x \in R^n, u \in R^r, y \in R^m$ -- соответственно векторы состояния, управления и выхода ОУ; $[A], B, C$ -- интервальная матрица состояния, матрица управления и выхода, согласованные по размерности с переменными модели~\ref{eq:mmc}.

Из предыдущего утверждения становится очевидным, что матрицы ОУ полученные в п.5 не соответствуют предъявляемым к ним требованиям и нужно представить ОУ в форме~\ref{eq:mmc}.

Для этого нужно ОУ~\ref{eq_OU} привести к наблюдаемой (фробениусовой) канонической форме, тогда его матрицы примут вид
\begin{equation}\label{Aqn}
	A(q) =
	\begin{bmatrix}
	0 & - \cfrac{(1+q_4)(1+q_7)}{2(1+q_3)(1+q_6)} \\
	1 &-\cfrac{20(1+q_3)(1+q_7)+3.6(1+q_4)(1+q_6)}{12(1+q_3)(1+q_6)}
	\end{bmatrix}
\end{equation}
\begin{equation}\label{Bqn}
	B(q) =
	\begin{bmatrix}
		\cfrac{(1 + q_2)}{30(1+q_3)(1+q_6)}\\
		\cfrac{(1 + q_1)}{4(1+q_3)(1+q_6)} 
	\end{bmatrix}
\end{equation}

\begin{equation}\label{Cqn}
	C =
	\begin{bmatrix}
	0 & 1
	\end{bmatrix}
\end{equation}

Затем, так как полученные ОУ в каноническом наблюдаемом базисе не позволяет параметрическую неопределенность представить только в виде вариации $\Delta A$ матрицы
состояния, то на входе ОУ достаточно включить буферную систему~\cite{NSUsh}
\begin{equation}
	\begin{cases}
		{\dot x_B} (t) = A_B x_B(t) + B_B u_B(t) u(t);\\
		y(t) = C_B x_B(t)
	\end{cases}
\end{equation}
минимальной размерности $\dim x_B = \dim u = r = 1$ и ввести в рассмотрение составной вектор $\tilde{x} = col\{x, x_B\}$ и $\tilde{u} = col\{u, u_B\}$, получим систему
\begin{equation}
	\begin{cases}
		\tilde{\dot x} (t) = \tilde{A} \tilde{x}(t) + \tilde{B} \tilde{u}(t);\\
		y(t) = \tilde{C} \tilde{x}(t)
	\end{cases}
\end{equation}
где
\begin{align}
	\tilde{A} =
	\begin{bmatrix}
	 A(q) & B(q) C_B\\
	 0 & A_B
	\end{bmatrix};
	\tilde{B} = 
	\begin{bmatrix}
		1&0\\
		0&B_B
	\end{bmatrix};
	\tilde{C} = 
	\begin{bmatrix}
		C & 0
	\end{bmatrix}
\end{align}

Составим матрицы, при $A_B = 0, B_B = 1, C_B = 1$
\begin{align}
	\tilde{A} =&
	\begin{bmatrix}
		0 & - \cfrac{(1+q_4)(1+q_7)}{2(1+q_3)(1+q_6)} &	\cfrac{(1 + q_2)}{30(1+q_3)(1+q_6)}\\
		1 &-\cfrac{20(1+q_3)(1+q_7)+3.6(1+q_4)(1+q_6)}{12(1+q_3)(1+q_6)} & \cfrac{(1 + q_1)}{4(1+q_3)(1+q_6)} \\
		0&0&0
	\end{bmatrix};\\
	\tilde{B} =&
	\begin{bmatrix}
	1&0\\
	1&0\\
	0&1
	\end{bmatrix};
	\tilde{C} =
	\begin{bmatrix}
	0 & 1 & 0
	\end{bmatrix}
\end{align}

В соответствии с~\ref{interv},~\ref{midA},~\ref{widA}, запишем интервальную матрицу $[\tilde{A}] = \tilde{A}_0 + [\Delta A]$
\begin{align}
	[\tilde{A}] =
	\begin{bmatrix}
		0&-0.6675  &0.0371 \\
		1& -2.6495 &0.3733\\
		0&0&0	
	\end{bmatrix}
	+
	\begin{bmatrix}
0&	[0.2991, 0.4475]	&[-0.0313, -0.2344] \\
0&	[-1.7755, 1.7755]	&[0.0254, 0.0955]\\
0&0&0
	\end{bmatrix}
\end{align}

Таким образом, ОУ~\ref{eq:mmc} характеризуется параметрической неопределенностью только матрицы состояния.


\subsection{Синтез медианного МУ НОУ}

Порядок полученного в пункте 6.1 ОУ $\dim n = 3$, а ранг матрицы $\tilde{B}$ $rang \tilde{B} = 2$ и $rang \tilde{B} < \dim n$, следовательно возможно решение только неполной задачи обобщенного модального управления (ОМУ).

Агрегирование полученного ОУ и ЗУ
\begin{equation}
	u (t) = K_g g(t) - K x(t)
\end{equation}
образует систему 
\begin{equation}
	\begin{cases}
		\dot x (t) = [F] x(t) + G g(t);\\
		y(t) = C x(t)
	\end{cases}
\end{equation}
где 
\begin{equation}
	[F] = F_0 + \Delta F, \Delta F = \Delta A, G = B K_g.
\end{equation}

Найдем нормы медианной и интервальной составляющих матрицы $\tilde{A}$ (далее будем ее обозначать как $A$)
\begin{align}
    ||A_0|| = 2.92; ||\Delta A|| = 1.8
\end{align}

Сформируем требования к показателям качества проектируемой системы: пусть $t_{n} \le 2c,  \sigma \le 5\%, \delta_{IR} F = 0.02$.

Выберем наблюдаемую пару матрицу модальной модели ($\Lambda$, H). Назначим матрицу $\Lambda$, соответствующую круговому распределению мод с характеристической частотой $\omega_0$ такой, что $||\Lambda|| = \omega_0$ и
\begin{equation}
	\Lambda = \arg\{||\Lambda|| = \cfrac{||\Delta A||}{\delta_{I} F} \& \sigma\{\Lambda\} = \sigma\{F\} \}
\end{equation}

Значение характеристической частоты $\omega_0$ определяется в силу технических требований к проектируемой системе из условия
\begin{equation}
	\omega_0 = max \{\omega_0 \re \cfrac{6}{2} = 3c^{-1}; \omega_0 \re \cfrac{||\Delta A||}{\delta_{IR}} = \cfrac{2.92}{0.02} =  146c^{-1}\} = 146 c^{-1}
\end{equation}

Тогда
\begin{equation}
	\Lambda = \omega_0 
	\begin{bmatrix}
	 - 0.2819&0&0\\
	0&- 2.3675647&0\\  
	0 &0&0
	\end{bmatrix}
	=
	\begin{bmatrix}
  - 0.8458058 &   0        &    0  \\
	0         & - 7.1026942 &   0  \\
	0         &   0         &   0 
	\end{bmatrix}
\end{equation}

 В качестве желаемого распределения мод проектируемой системы распределения мод Баттерворта третьего порядка, которое параметризованное характеристической частотой $\omega_0 = 3c^{-1}$ позволяет для матрицы $\Lambda$ записать


\subsection{Исследование свойства робастности системы}


\newpage