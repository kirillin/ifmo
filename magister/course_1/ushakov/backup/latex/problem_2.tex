\section{Построение МТЧ ДОУ и результаты ее исследования}

\begin{enumerate}
	\item Перейти к дискретному описанию ОУ с помощью интегральной модели ВСВ НОУ;
	\item Построить модель траекторной чувствительности (МТЧ) дискретного ОУ (ДОУ) к вариации интервала дискретности;
\end{enumerate}


\subsection{Переход к дискретному описанию ОУ}

ДОУ представляет собой дискретную по времени с интервалом дискретности
длительности $\Delta t$ выборку из непрерывных процессов по вектору
состояния $x(t,q)$ и выходу $y(t,q)$ при фиксированном на интервале $t \in \left[\Delta t k, \Delta t(k+1)\right]$ значении управления $u(t) = u(\Delta t k) = u(k)$. Имеет следующий вид
\begin{equation}\label{eq_iso_doc}
	\begin{cases}
		x(k+1, q) = \overline{A}(q) x(k, q) + \overline{B}(q) u(k)\\
		y(k, q) = \overline{C}(q) x(k, q)
	\end{cases}
\end{equation}
где матрицы непрерывного~\ref{eq_iso_coc} и дискретного~\ref{eq_iso_doc} ОУ связаны следующими функциональными соотношениями
\begin{align}
	\overline{A}(q) = e^{A(q) \Delta t}; \overline{B}(q) = A^{-1}(q)(e^{A(q) \Delta t} - I)B(q); \overline{C}(q) = C(q)
\end{align}


Номинальная модель ДОУ получается из~\ref{eq_iso_doc} при векторе параметров $q=q_0$
\begin{equation}\label{eq_iso_doc_n}
	\begin{cases}
		x(k+1) = \overline{A} x(k) + \overline{B} u(k)\\
		y(k) = \overline{C} x(k)
	\end{cases}
\end{equation}

Общий вид интегральной модели~\cite{MIROSH} ВСВ НОУ имеет вид
\begin{align}
	&x(t) = \Phi (t) x(0) + \int_0^t \Phi (t, \tau) B u(\tau) d \tau \\
	&y(t) = C \Phi (t) x(0) + \int_0^t C \Phi(t, \tau) B u(\tau) d \tau 
\end{align}
где $\Phi(t) = e^{At}, \Phi(t, \tau) = \Phi(t) \Phi^{-1}(\tau) = e^{A(t-\tau)}$.

Используя интегральную запись модели ВСВ непрерывного динамического объекта, нетрудно получить связь между матрицами модели ВСВ дискретного и непрерывного объектов в форме
\begin{align}
	\overline{A} = \Phi (\Delta t) = e^{A \Delta t},
	\overline{B} = \Phi (\Delta t) \int_0^{\Delta t} \Phi^{-1} (\tau) d \tau B,
	\overline{C} = C
\end{align}

И окончательные формулы для перехода
\begin{align}
	\overline{A} = e^{A \Delta t},
	\overline{B} = A^{-1} (e^{A \Delta t} - I)B,
	\overline{C} = C  
\end{align}

При $\Delta t = 0.03$c, рассчитаем матрицы модели ВСВ ДОУ
\begin{align*}
	\overline{A} =
	\begin{bmatrix}
		0.9997794 &   0.0291300\\  
		- 0.0145650  &  0.9424904 
	\end{bmatrix};
	\overline{B} =
	\begin{bmatrix}
		0.0004413 \\
		0.0291300 
	\end{bmatrix};
	\overline{C} =
	\begin{bmatrix}
		 0.0333333  &  0.25 
	\end{bmatrix};
\end{align*}

\subsection{Построение МТЧ ДОУ к вариации интервала дискретности}

Модель траекторной чувствительности, необходимая для генерирования функций траекторной чувствительности $\sigma(k)$ и $\eta(k)$ по состоянию и выходу ДОУ, строится путем дифференцирования компонентов представления~\ref{eq_iso_doc} по компонентам $q_j$ вектора параметров $q$ при его номинальном значении (в нашем случае $q = \Delta t$), в результате чего для МТЧ получаем

\begin{equation}\label{eq_mts_d}
	\begin{cases}
		\sigma(k+1) = \overline{A} \sigma(k) + \overline{A}_{q} x(k) + \overline{B}_{q} u(k)\\
		\eta (k) = \overline{C} \sigma (k) + \overline{C}_{q} x(k)
	\end{cases}
\end{equation}
где 
\begin{align*}
	&\overline{A}_{q} = \cfrac{\partial{\overline{A}(q)}}{\partial{\Delta t}} \bigg|_{q=q_0};
	\overline{B}_{q} = \cfrac{\partial{\overline{B}(q)}}{\partial{\Delta t}} \bigg|_{q=q_0};
	\overline{C}_{q} = \cfrac{\partial{\overline{C}(q)}}{\partial{\Delta t}} \bigg|_{q=q_0};\\
	&\sigma(t) = \cfrac{\partial{x(k,q)}}{\partial{\Delta t}} \bigg|_{q=q_0};
	\eta(t) = \cfrac{\partial{y(k,q)}}{\partial{\Delta t}} \bigg|_{q=q_0};\\
	&\cfrac{\partial{\overline{A}(q)}}{\partial{\Delta t}} = 
	\cfrac{\partial{\left(e^{A(q) \Delta t}\right)}}{\partial{\Delta t}} =
	A(q) e^{A(q) \Delta t} =  e^{A(q) \Delta t} A(q)\ = \overline{A}(q) A(q);\\
	&\cfrac{\partial{\overline{B}(q)}}{\partial{\Delta t}} = 
	\cfrac{\partial{}}{\partial{\Delta t}} \left[A^{-1}(q)(e^{A(q) \Delta t} - I)B(q)\right] = 
	A^{-1}(q) A(q) e^{A(q) \Delta t} B = \overline{A}(q) B(q);\\
	&\cfrac{\partial{\overline{C}(q)}}{\partial{\Delta t}} = 
	\cfrac{\partial{C(q)}}{\partial{\Delta t}} = 0.
\end{align*}

Используя полученные выражения вычислим матрицы МТЧ ДОУ
\begin{align*}
	\overline{A}_q =
	\begin{bmatrix}
		- 0.0145650 &   0.9424904\\  
		- 0.4712452 & - 1.8681295  
	\end{bmatrix};
	\overline{B}_q =
	\begin{bmatrix}
		0.0291300\\
		0.9424904  
	\end{bmatrix};
	\overline{C}_q =
	\begin{bmatrix}
		0  &  0 
	\end{bmatrix};
\end{align*}

Сконструируем агрегированную систему с составным вектором \newline $\tilde{x}=col\{x,\sigma\}$ размерности $ \dim \tilde{x} = 2n$, которая объединением \ref{eq_iso_doc_n} и \ref{eq_mts_d}, получает представление
\begin{align}\label{rang_system_d_3}
	\tilde{x}(k+1) &= \tilde{\overline{A}} \tilde{x} (k) + \tilde{\overline{B}} u(k); 
	\tilde{x} (0) = col \{x(0), 0\}\\
	x(k) &= \tilde{\overline{C}}_{x_j} \tilde{x}(k);\\
	y(k) &= \tilde{\overline{C}} \tilde{x} (k);\\
	\sigma (k) &= \tilde{\overline{C}}_{\sigma} \tilde{x} (k);\\
	\eta (k) &= \tilde{\overline{C}}_{\eta} \tilde{x} (k)
\end{align}
где 
\begin{align*}
	&\tilde{\overline{A}} =
	\begin{bmatrix}
	\overline{A} & 0\\
	\overline{A}_{q} & \overline{A}
	\end{bmatrix},
	\tilde{\overline{B}} = 
	\begin{bmatrix}
	\overline{B}\\
	\overline{B}_{q}
	\end{bmatrix},\\
	&\tilde{\overline{C}}_{x} = 
	\begin{bmatrix}
	I_{n \times n} & O_{n \times n}
	\end{bmatrix},
	\tilde{\overline{C}} = 
	\begin{bmatrix}
	\overline{C} & 0_{m \times n}
	\end{bmatrix},
	\tilde{\overline{C}}_{\sigma} = 
	\begin{bmatrix}
	0_{n \times n} & I_{n \times n}
	\end{bmatrix},
	\tilde{\overline{C}}_{\eta} = 
	\begin{bmatrix}
	\overline{C}_{q} & \overline{C}
	\end{bmatrix}.	
\end{align*}

Составим необходимые матрицы

\begin{align*}
	&\tilde{\overline{A}} =
	\begin{bmatrix}
    0.9997794  &  0.0291300 &0& 0 \\
	- 0.0145650  &  0.9424904  &0& 0\\
	- 0.0145650  &  0.9424904 &  0.9997794  &  0.0291300\\  
	- 0.4712452 & - 1.8681295  & - 0.0145650  &  0.9424904  
	\end{bmatrix};\\
	&\tilde{\overline{B}} =
	\begin{bmatrix}
       0.0004413\\  
		0.0291300 \\ 
		0.0291300  \\
		0.9424904 
	\end{bmatrix};
	\tilde{\overline{C}}_{\eta} =
	\begin{bmatrix}
		0&0&0.0333333 &   0.25\\
	\end{bmatrix}
\end{align*}

Проверим управляемость агрегированной системы по выходу $\eta(k)$ с помощью матрицы управляемости $\tilde{\overline{W}}_{y \eta}$
\begin{equation}
	\tilde{\overline{W}}_{y \eta} =
	\begin{bmatrix}
	\tilde{\overline{C}}_{\eta} \tilde{\overline{B}} &
	\tilde{\overline{C}}_{\eta} \tilde{\overline{A}} \tilde{\overline{B}} &
	\tilde{\overline{C}}_{\eta} \tilde{\overline{A}}^2 \tilde{\overline{B}} &		
	\cdots &
	\tilde{\overline{C}}_{\eta} \tilde{\overline{A}}^{2n-1} \tilde{\overline{B}}
	\end{bmatrix}
\end{equation}
которая с учетом $n=2$ имеет реализацию
\begin{equation*}
	\tilde{\overline{W}}_{y \eta} =
	\begin{bmatrix}
	0.2365936  &  0.2111102  &  0.1875234  &  0.1657095  
	\end{bmatrix}
\end{equation*}

Ранги матриц $\tilde{W}_{\eta}$ равны $rang(\tilde{W}_{\eta}) = 1$, что совпадает с размерностью $m = 1$ вектора выхода. Таким образом, выбором закона
управления можно обеспечить сходимость $\lim_{t  \to \infty} \Delta y (t,q_0,\Delta t) = 0$ с заданным темпом. 



\newpage