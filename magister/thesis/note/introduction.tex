\section*{Введение}
\addcontentsline{toc}{section}{Введение}

Основная цель работы~--- схватывание объектов с вращающегося стола  при помощи пятизвенного манипулятора KUKA YouBot с использованием системы технического зрения в условиях соревнований RoboCup@Work.


\subsection*{Мотивация}

Одной из задач, которая стоит перед студенческим конструкторским бюро кафедры Систем управления и информатики это превзойти результаты других команд на ежегодных международных соревнованиях RoboCup.

Цель этих соревнований состоит в продвижении робототехники и исследовании искусственного интелекта на популярных и идейно простых, но нетривиальных задачах. В 2017 году на RoboCup проводились соревнования в шести основных категориях: футбол роботов (Soccer), роботы-спасатели (Rescue Leagues Homepage), домашние роботы (@Home), промышленные роботы (@Work), инфраструктура снабжения (RoboCup Logistics League) и младшая лига (RoboCupJunior) для школьников. 

Суть соревнований в категории AtWork~--- создать симуляцию реального производственного цикла~\cite{kraetzschmar2014robocup}. 

На производстве есть детали, которые необходимо доставить из пункта А в пункт Б. Классический способ это сделать~--- использовать конвейер. 

Конвейеры хорошо зарекомендовали себя в горнодобывающей и перерабатывающей промышленности (подъем и транспортирование горной массы), в сельском хозяйстве (сортировка, транспортирование, упаковка сельскохозяйственной продукции, конвейерная разделка туш скота), в производстве продовольственных и непроводольственных товаров (от газировки и пачек с макаронами, до хозяйственных товаров и электроприборов), и в других областях человеческой жизни.%~\cite{convs}.

Однако у конвейеров есть свои недостатки, поэтому многие коллективы исследователей работают над разработкой иных методов. Один из главных минусов конвейера~--- нельзя быстро и дёшево изменить его конфигурацию. Сегодня аналог классическому конвейеру~--- мобильные роботы, с помощью которых можно решать те же самые задачи, более целенаправленно: робота можно перевести в любую точку и быстро переконфигурировать, задав ему новую цель. 

Если мобильную платформу оснастить манипулятором с системой технического зрения, то полученная робототехническая система сможет выполнять более сложные задачи, например, собирать с конвейеров, столов или полок определенные детали и перемещать их в указанный пункт назначения.

%\subsubsection*{Промышленность}
Разрабатываемая робототехническая система в повседневной жизни может быть применена, например, в аэропортах, для погрузки и разгрузки багажа, в библиотеках для автоматизации манипуляций с книгами, на железнодорожных станциях для транспортировки грузов между грузовыми вагонами и колесными грузовиками, в портах для эффективной организации грузов на палубах и в трюмах кораблей и многие другие.

\subsection*{Задачи}

Одно из испытаний, которое предстоит роботу на соревнованиях в лиге AtWork~--- заключается в оценке способности робота манипулировать движущимися объектами, которые перемещаются на конвейерной ленте или вращающемся столе (Conveyor Belt Test / Rotating Table Test). Это испытание требует хорошей точности управления манипулятором и качественной системы технического зрения.

В этой работе ставятся следующие задачи:
\begin{enumerate}
	\item Построить математическую модель манипулятора KUKA Youbot;
	\item Синтезировать систему управления манипулятором;
	\item Спланировать необходимые для захвата движущихся объектов траектории;
	\item Разработать систему технического зрения для оценки положения и скорости движущегося объекта;
	\item Провести моделирование разработанной системы.
\end{enumerate}

\newpage