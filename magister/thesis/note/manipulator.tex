\section{Разработка системы управления манипулятором KUKA YouBot}\label{part_manipulator}

Раздел посвящён теоретическим выкладкам, позволяющим схвату манипулятора перемещаться по заданной траектории. Сначала определяются основные кинематические соотношения, описывающие манипулятор KUKA YouBot (далее манипулятор) без учета действующих на него сил. Они позволяют определить положение манипуляционного механизма в пространстве, а также скорости и ускорения всех звеньев. Синтезируется система управления с использованием кинематических соотношений.

%Во второй части исследуются алгоритмы динамического управления, учитывающие динамику манипулятора при определении управляющих воздействий. Получается динамическая модель манипулятора в форме уравнений Лагранжа, которая представляется в \textcolor{red}{матричной форме для реализации системы управления} и регрессионной для проведения процедуры идентификации динамических параметром манипулятора. Синтезируется система моментного управления.

\subsection{Описание манипулятора}\label{section_description_of_robot}
Основной элемент практически любого робота это~--- манипулятор, в этой работе представленный манипулятором с робота KUKA Youbot и изображённый на рисунке~\ref{img:sizes_of_robot}в. Его механизм имеет пять степеней подвижности. Многозвенная конструкция манипулятора заканчивается двухпальцевым схватом ~--- инструментом, предназначенным для захвата объектов определённой формы.	

Описание технических характеристик дается таблицей~\ref{table_gen_info_of_manipulator} и рисунком~\ref{img:sizes_of_robot}.
%Неуказанные там параметры робота, требуемые для дальнейших расчетов, неизвестны и поэтому подлежат измерению или идентификации, речь о которых пойдет ниже по тексту.

\begin{table}[h!]
	\caption{Общая информация о манипуляторе робота Kuka Youbot.}
	\begin{center}
		\begin{tabularx}{1\textwidth}{|>{\hsize=0.7\textwidth}X|Y|}
			\hline
			\multicolumn{1}{|c|}{Параметр} & \makebox[3cm]{Значение}\\
			\hline
			Количество сочленений & 5\\
			\hline
			Объем рабочей области, $ \text{м}^2 $ & 0.513 \\
			\hline
			Масса, $ \text{кг} $ & 5.3\\
			\hline
			Допустимая нагрузка, $ \text{кг} $ & $0.5$~\\
			\hline
			Точность повторного воспроизведения позиции, $ \text{мм} $ & 1\\
			\hline
			Максимальная скорость в сочленении, $ ^\circ\text{ с}^{-1} $& $90$\\
	    	\hline
			Интерфейс & EtherCAT\\
			\hline
			Напряжение питания, $ \text{В} $ & 24\\
			\hline
		\end{tabularx}
	\end{center}
	\label{table_gen_info_of_manipulator}
\end{table}

\begin{figure}[p]
	\center{\includegraphics[width=0.7\textwidth]{youbot_workspace_1.jpg} \\ а)}
	\vfill
	\begin{minipage}[h]{0.47\linewidth}
		\centering{\includegraphics[width=0.95\linewidth]{youbot_workspace_2.jpg} \\ б)}
	\end{minipage}
	\hfill
	\begin{minipage}[h]{0.47\linewidth}
		\centering{\includegraphics[width=0.95\linewidth]{youbot_length.jpg} \\ в)}
	\end{minipage}
	\caption{Некоторые параметры манипулятора Kuka Youbot: a~--- размеры рабочей области (вид сбоку); б~--- размеры рабочей области (вид сверху); в~--- длины звеньев и предельные значения для углов вращения по каждому из сочленений~\cite{youbot_detailed_specifications}.}
	\label{img:sizes_of_robot}
\end{figure}

%		\hline
%		Поворот 1 сочленения &  $ \pm 169^\circ $\\
%		\hline
%		Поворот 2 сочленения &  $ +90^\circ / -65^\circ$\\
%		\hline
%		Поворот 3 сочленения &  $ +146^\circ / -151^\circ$\\
%		\hline
%		Поворот 4 сочленения &  $\pm 102,5^\circ$\\
%		\hline
%		Поворот 5 сочленения &  $\pm 167,5^\circ$\\
\newpage
\subsection{Кинематическая модель манипулятора}\label{part_kinematic_model}
\input{kinematics/kinematic_model_position}
\vspace{0.5cm}
\subsubsection{Решение задач о скорости звеньев манипулятора}\label{part_kinematic_velocity}

%%\textbf{Приведем некоторые соотношения для скоростей и ускорений.}

Как видно из рассуждений в разделе~\ref{part_kinematic_position}, $i$-ое звено связано с СК, начало которой закрепленно в ($i+1$) сочленении. При этом кинематические параметры этого звена задаются в связанной с ним СК $Ox_{i-1}y_{i-1}z_{i-1}$.

Получим выражения, которые позволят последовательно получать скорости и ускорения звеньев манипулятора, начиная от базы в направлении схвата, выраженные в абсолютной СК $Ox_{0}y_{0}z_{0}$.

\begin{figure}[h!]
	\centering{\includegraphics[width=0.5\textwidth]{ipe/example_of_moving.pdf}}
	\caption{Рисунок, поясняющий приведённые в разделе~\ref{part_kinematic_velocity} выражения}
	\label{img:bodies}
\end{figure}

Заметим, что в соответствии с рисунком~\ref{img:bodies}, на котором изображена упрощенная кинематическая схема для двух соседних звеньев, радиус вектор
\begin{equation}\label{vector_r}
	r_{i-1, i} = r_i - r_{i-1}
\end{equation}
характеризует расположение СК $Ox_{i}y_{i}z_{i}$ относительно СК $Ox_{i-1}y_{i-1}z_{i-1}$.

Согласно~\eqref{vector_r} и теореме механики о сложении скоростей~\cite{yabl19964}, соотношения для  скоростей в подвижной и неподвижной системах координат запишутся:
\begin{equation}
	v_i = v_{i-1} + \omega_{i-1} \times r_{i-1,i} + \cfrac{d^* r_{i-1,i}}{dt},
\end{equation}
где $\frac{d^* r_{i-1,i}}{dt}$~--- скорость $i$-ого звена относительно СК $Ox_{i-1}y_{i-1}z_{i-1}$.

Для угловой скорости:
\begin{equation}\label{omega_i}
	\omega_i = \omega_{i-1} + \omega^{i-1}_{i},
	\quad
	\omega^{i-1}_{i} = z_{i-1} \cdot \dot{q}_i,
\end{equation}
где $\omega^{i-1}_{i}$~--- угловая скорость вращения СК $Ox_{i}y_{i}z_{i}$ относительно $Ox_{i-1}y_{i-1}z_{i-1}$, $\dot{q}_i$~--- угловая скорость вращения $i$-ого звена.

С учетом равенств~\eqref{vector_r}--\eqref{omega_i}, окончательно запишем рекуррентные соотношения для вычисления линейных и угловых скоростей манипулятора:
\begin{equation}\label{linear_velocities}
	v_i = v_{i-1} + \omega_{i} \times r_{i-1,i}, %%% + z_{i-1} \cdot \dot{q}_i,
\end{equation}

\begin{equation}\label{angular_velocities}
	\omega_i = \omega_{i-1} + z_{i-1} \cdot \dot{q}_i.
\end{equation}

Далее, согласно теореме о сложении ускорений, запишем соотношения для линейных и угловых ускорений:
\begin{equation}
	\dot{v}_i = \dot{v}_{i-1} + \omega_{i-1} \times (\omega_{i-1} \times r_{i-1,i}) + 2 \cdot \omega_{i-1} \times \cfrac{d^* r_{i-1,i}}{dt} + \cfrac{d^{*2} r_{i-1,i}}{dt^2},
\end{equation}
\begin{equation}
	\dot{\omega}_{i} = \dot{\omega}_{i-1} + \dot{\omega}^{i-1}_{i},
\end{equation}
где 
\begin{equation}
	\dot{\omega}^{i-1}_{i} = \cfrac{d^{*} \omega^{i-1}_{i}}{dt} + \omega_{i-1} \times \omega^{i-1}_{i},
	\quad
	\cfrac{d^{*} \omega^{i-1}_{i}}{dt} = z_{i-1} \cdot \ddot{q}_i.
\end{equation}

После элементарных преобразований, рекуррентные соотношения для линейных и угловых ускорений запишутся, как:
\begin{equation}\label{linear_accelerations}
	\dot{v}_i = \dot{v}_{i-1} + \omega_{i} \times (\omega_{i} \times r_{i-1,i}) + \dot{\omega}_i \times r_{i-1,i}, %%% + 2 \cdot \omega_{i} \times (z_{i-1} \cdot \dot{q}_i) + \dot{\omega}_{i} \times \dot{p}_{i-1,i} + z_{i-1} \cdot \ddot{q}_i,
\end{equation}
\begin{equation}\label{angular_accelerations}
	\dot{\omega}_i = \dot{\omega}_{i-1} + z_{i-1} \cdot \ddot{q}_i + \omega_{i-1} \times (z_{i-1} \cdot \dot{q}_i).
\end{equation}

Таким образом, соотношения~\eqref{linear_velocities}--\eqref{angular_velocities} и~\eqref{linear_accelerations}--\eqref{angular_accelerations} позволяют найти линейные и угловые скорости и ускорения звеньев, при $i=\overline{1,n}$ и начальных условиях $\omega = \dot{\omega} = v_0 = 0$, $\dot{v}_0 = g$, где $g$~--- вектор ускорения свободного падения.

Отсюда можно без труда получить конечные зависимости линейных и угловых скоростей. Для линейной скорости $k$-ого звена:
\begin{equation}
	v_k = \sum_{i = 1}^{k} z_{i-1} \times r_{i-1, k} \cdot \dot{q}_i.
\end{equation}
Для угловой скорости $k$-ого звена:
\begin{equation}
	\omega_k = \sum_{i = 1}^{k} z_{i-1} \cdot \dot{q}_i.
\end{equation}

Все векторы в рассмотренных соотношениях, описывающие движения манипулятора, заданы в базовой СК $Ox_{0}y_{0}z_{0}$. Руководствуясь соображениями вычислительной эффективности, представим векторы скоростей в СК их собственного звена. Тогда, если матрица перехода из СК $i$-ого звена $Ox_{i}y_{i}z_{i}$ в СК $Ox_{i-1}y_{i-1}z_{i-1}$, то:
\begin{equation}
	{}^{i-1}\!A_{i}^{-1} = 
	\begin{bmatrix}
		{}^{i-1}\!R_{i}^T	& {p}_{i-1,i} \\
		000	& 1
	\end{bmatrix}\!\!\!,
\end{equation}
где ${p}_{i-1,i} = {}^0\!R_{i-1}^T \cdot r_{i-1,i}$.

Можно переписать выражения ~\eqref{linear_velocities}--\eqref{angular_velocities} и~\eqref{linear_accelerations}--\eqref{angular_accelerations} в виде:
\begin{equation}
	v_i^i = {}^{i-1}\!R_{i} \cdot v^{i-1}_{i-1} + \omega^{i}_{i} \times p_{i-1,i},
\end{equation}
\begin{equation}
	\omega^{i}_{i} = {}^{i-1}\!R_{i} \cdot \omega^{i-1}_{i-1} + {}^{i-1}\!R_{i} \cdot z_0 \dot{q}_i,
\end{equation}
\begin{equation}
	\dot{v}^{i}_{i} = {}^{i-1}\!R_{i} \cdot \dot{v}^{i-1}_{i-1} + \omega^i_i \times (\omega^i_i \times p_{i-1,i}) + \dot{\omega}^i_i \times p_{i-1, i},
\end{equation}
\begin{equation}
	\dot{\omega}^i_i = {}^{i-1}\!R_{i} \cdot \dot{\omega}^i_i + {}^{i-1}\!R_{i} \cdot z_0 \ddot{q}_i + {}^{i-1}\!R_{i} \cdot \omega^{i-1}_{i-1} \times (z_0 \dot{q}_i).
\end{equation}

%%%
В задачах о скорости движения схвата манипулятора описывается вектором $\mathbf{s} = [x,y,z,\varphi, \psi, \theta]$, линейной $ \bm{v} $ и угловой $ \bm{\omega} $ скоростями, где со схватом связана СК $ Ox_{n}y_{n}z_{n} $. Вектор $ \mathbf{s} $ определяется конфигурацией манипулятора, задаваемой вектором обобщеннных координат $ \mathbf{q} = [q_1, q_2, \dots, q_n]^T $, что можно записать, как:
\begin{equation}\label{s_fq}
	\mathbf{s} = f (\mathbf{q}).
\end{equation}

Дифференцируя~\eqref{s_fq} по времени, получаем:
\begin{equation}\label{sjq}
	\dot{\mathbf{s}} = J (\mathbf{q}) \cdot \dot{\mathbf{q}},
\end{equation}
где $ \dot{\mathbf{s}} $ вектор скорости схвата:
\begin{equation}\label{fvk_1}
	\dot{\mathbf{s}} =
	\begin{bmatrix}
			\bm{v} \\ 
		\bm{\omega}
	\end{bmatrix}\!\!\!,
\end{equation}
$ J(\mathbf{q}) $~--- матрица Якоби размерности $ 6 \times n $, $\dot{\mathbf{q}} = [\dot{q}_1, \dot{q}_2, \dots, \dot{q}_n]^T $~--- вектор скоростей обобщенных координат.

%\subsubsection{Прямая задача о скорости}
\textbf{Прямая задача о скорости} формулируется, как нахождение линейной~$ \bm{v} $ и угловой~$ \bm{\omega} $ скоростей схвата по известным скоростям обобщенных координат $ \dot{q}_1, \dot{q}_2, \dots, \dot{q}_n $.
Заметим, что из~\eqref{linear_velocities}--\eqref{angular_velocities} следует, что если записать матрицу Якоби в виде:
\begin{equation}\label{Jq}
	J(\mathbf{q}) = 
	\begin{bmatrix}
	\bm{j}_1 & \bm{j}_2 & \cdots & \bm{j}_n
	\end{bmatrix},
\end{equation}
то вектор $ \bm{j}_k $ будет рассчитываться из выражения:
\begin{equation}
	\bm{j}_k = 
	\begin{bmatrix}
	z_{k-1} \\
	z_{z-1} \times p_{k-1,n}
	\end{bmatrix},
\end{equation}

Следовательно, выражение~\eqref{sjq} является решением прямой задачи о скорости, где матрица $ J(\mathbf{q}) $ находится из~\eqref{Jq}.

Матрицу Якоби получим, используя дифференциальные преобразования. Обозначим матрицу Якоби, как:
\begin{equation}\label{J}
	J = 
	\begin{bmatrix}
		J_v \\ 
		J_{\omega}
	\end{bmatrix}\!\!\!,
\end{equation}
тогда перепишем соотношения для скоростей в виде
\begin{align}
	\label{Jv1}
	\bm{v}  = J_{v} \dot{\bm{q}}, \\
	\label{Jw1}
	\bm{\omega} = J_{\omega} \dot{\bm{q}}.
\end{align}

\begin{figure}[h!]
	\begin{minipage}[h]{0.5\linewidth}
		\centering{\includegraphics[width=0.95\linewidth]{ipe/diff_matrix_v2.pdf} \\ а)}
	\end{minipage}
	\hfill
	\begin{minipage}[h]{0.5\linewidth}
		\centering{\includegraphics[width=0.95\linewidth]{ipe/diff_transform_v2.pdf} \\ б)}
	\end{minipage}
	\caption{а~--- дифференциальное перемещение, б~--- изменение матрицы трансфорации при дифференциальном перемещении}
	\label{img:diff_transformations}
\end{figure}

Если представить малое изменение матрицы трансформации $ T $ (рисунок~\ref{img:diff_transformations}a), как
\begin{equation}
	\tilde{T} = T + dT,
\end{equation}
откуда $ dT = \tilde{T} - T = \tilde{T}(d\bm{r}, \delta\bm{\varphi}) - T$, где $ d\bm{r} $ и $ \delta\bm{\varphi} $~--- малые перемещение и поворот соответственно, то для k-ого звена манипулятора,  применяя правило дифференцирования матриц однородных преобразований $ \cfrac{\partial A_k}{\partial q_l} = \theta_k \cdot A_k $, можно получить:
\begin{equation}\label{dkT}
	d_k T = A_1 A_2 \dots A_{k-1} \theta_k dq_k A_{k+1} \dots A_{n-1} A_n,
\end{equation}
где 
\begin{equation}
	\theta_k = 
	\begin{bmatrix}
	0 & -1 & 0 & 0 \\
	1 & 0 & 0 & 0 \\
	0 & 0 & 0 & 0 \\
	0 & 0 & 0 & 0
	\end{bmatrix}
\end{equation}

Выражение~\eqref{dkT} можно переписать в виде:
\begin{equation}
	d_k T = \Delta_{\theta_k} \cdot T,
\end{equation}
где $ \Delta_{\theta_k} $~--- матрица дифференциальных преобразований:
\begin{equation}\label{Delta_matrix}
	\Delta_{\theta_k} = T_{k-1} \theta_k T_{k-1}^{-1} dq_k.
\end{equation}

Обозначим разыскиваемые дифференциальные перемещения  схвата манипулятора, как~$ d \bm{r}_n $ и~$\delta\bm{\varphi}_n $, изображённые на рисунке~\ref{img:diff_transformations}б, тогда матрица дифференциальных преобразований:
\begin{equation}
	\Delta_n =
	\begin{bmatrix}
		0 & - \delta z_n & \delta y_n & d x_n \\
		\delta z_n & 0 & -\delta x_n & d y_n \\
		-\delta y_n & \delta x_n & 0 & d z_n \\
		0 & 0 & 0 & 0
	\end{bmatrix}
	=
	\left[
	\begin{array}{c:c}
	\Omega_{\delta\bm{\varphi}_n} & d \bm{r}_n \\ \hdashline
	000 &  0
	\end{array}
	\right]\!\!\!.
\end{equation}

Раскрыв~\eqref{Delta_matrix}, получим:
\begin{equation}
	\left[
	\begin{array}{c:c}
		\Omega_{\delta\bm{\varphi}_n} & d\bm{r}_n - \Omega_{\delta\bm{\varphi}_n} \cdot {p}_n \\ \hdashline
		000 &  0
	\end{array}
	\right]
	=
	\left[
	\begin{array}{c:c}
		R_{k-1} \Omega_{001} R_{k-1}^T & - R_{k-1} \Omega_{001} R_{k-1}^T {p}_{k-1} \\ \hdashline
		000 &  0
	\end{array}
	\right]
	dq_k,
\end{equation}
где 
\begin{gather}
	\label{dr}
	d\bm{r}_n = \Omega_{\delta\bm{\varphi}_n} \cdot {p}_n - R_{k-1} \cdot \Omega_{001} \cdot R_{k-1}^T \cdot {p}_{k-1} \cdot dq_k, \\
	\label{Omega_phi}
	\Omega_{\delta\bm{\varphi}_n} = R_{k-1} \cdot \Omega_{001} \cdot R_{k-1}^T \cdot dq_k,
\end{gather}
$ \Omega_{001} $~--- матрица, задающая вращение вокруг вектора $ [0,0,1]^T $.

Вращение вокруг оси $ z_{k-1} $ определяется выражением:
\begin{equation}
	z_{k-1} = R_{k-1} \cdot
	\begin{bmatrix}
	0 \\ 0 \\ 1
	\end{bmatrix}\!\!\!,
\end{equation}
тогда, в силу~\eqref{Omega_phi} и \eqref{dr}, имеем
\begin{gather}
	\label{diff_dr}
	d\bm{r}_n = z_{k-1} \times (p_n - p_{k-1}) \cdot dq_k, \\
	\label{diff_delta_phi}
	\delta\bm{\varphi}_n = z_{k-1} \cdot dq_k.
\end{gather}

Из выражения~\eqref{diff_delta_phi} видим, что расчет матрицы Якоби $ J_\omega $ из выражения~\eqref{J} не представляет труда, так как векторы $ z_i $ легко извлекаются из матриц трансформации $ T_i $ (третий столбец).

%\textcolor{red}{Представим альтернативную форму представления матрицы Якоби, позволяющую эффективнее производить ее вычисления...}

Из выражений~\eqref{diff_delta_phi}--\eqref{diff_dr} и~\eqref{J} получим все матрицы Якоби для рассматриваемого манипулятора:

\begin{gather*}
	J_{\omega 1} =
	\begin{bmatrix}
		z^0_0 & \nv & \nv & \nv & \nv
	\end{bmatrix}\!\!,
	\qquad
	J_{\omega 2} =
	\begin{bmatrix}
		z^0_0 & z^0_1 & \nv & \nv & \nv
	\end{bmatrix}\!\!,
	\\
	J_{\omega 3} =
	\begin{bmatrix}
		z^0_0 & z^0_1 & z^0_2 & \nv & \nv
	\end{bmatrix}\!\!,
	\qquad
	J_{\omega 4} =
	\begin{bmatrix}
		z^0_0 & z^0_1 & z^0_2 & z^0_3 & \nv
	\end{bmatrix}\!\!,
	\\
	J_{\omega 5} =
	\begin{bmatrix}
		z^0_0 & z^0_1 & z^0_2 & z^0_3 & z^0_4
	\end{bmatrix}\!\!,
\end{gather*}
где $\nv = [0\;0\;0]^T$~--- нулевой вектор.

\begin{gather*}
J_{v1} =
\begin{bmatrix}
z^0_0 \times \left( r^0_{0,\,1} - r^0_{0,\,0}\right) & \nv & \nv & \nv & \nv
\end{bmatrix}\!\!,
\\
J_{v2} =
\begin{bmatrix}
z^0_0 \times \left( r^0_{0,\,2} - r^0_{0,\,0}\right) & z^0_1 \times \left( r^0_{0,\,2} - r^0_{0,\,1}\right) & \nv & \nv & \nv
\end{bmatrix}\!\!,
\\
J_{v3} =
\begin{bmatrix}
z^0_0 \times \left( r^0_{0,\,3} - r^0_{0,\,0}\right) & z^0_1 \times \left( r^0_{0,\,3} - r^0_{0,\,1}\right) &
z^0_2 \times \left( r^0_{0,\,3} - r^0_{0,\,2}\right) & \nv & \nv
\end{bmatrix}\!\!,
\end{gather*}
\begin{gather*}
J_{v4} =
\begin{bmatrix}
z^0_0 \times \left( r^0_{0,\,4} - r^0_{0,\,0}\right) \\
z^0_1 \times \left( r^0_{0,\,4} - r^0_{0,\,1}\right) \\
z^0_2 \times \left( r^0_{0,\,4} - r^0_{0,\,2}\right) \\
z^0_3 \times \left( r^0_{0,\,4} - r^0_{0,\,3}\right) \\
\nv
\end{bmatrix}^T\!\!\!\!\!,
\qquad
J_{v5} =
\begin{bmatrix}
z^0_0 \times \left( r^0_{0,\,5} - r^0_{0,\,0}\right) \\
z^0_1 \times \left( r^0_{0,\,5} - r^0_{0,\,1}\right) \\
z^0_2 \times \left( r^0_{0,\,5} - r^0_{0,\,2}\right) \\
z^0_3 \times \left( r^0_{0,\,5} - r^0_{0,\,3}\right) \\
z^0_4 \times \left( r^0_{0,\,5} - r^0_{0,\,4}\right)
\end{bmatrix}^T\!\!\!\!\!.
\end{gather*}


%\subsubsection{Обратная задача о скорости}
\textbf{Обратная задача о скорости} формулируется как поиск неизвестных скоростей обобщенных координат при известных линейной и уголовой скоростях схвата. Соотношения записывается как система шести линейных уравнений с $ n $ неизвестными $ \dot{q}_1, \dot{q}_2, \dots, \dot{q}_n $:
\begin{equation}\label{ivp}
	\dot{\mathbf{q}} =  J^{-1}(\mathbf{q})\dot{\mathbf{s}}.
\end{equation}

Решение~\eqref{ivp} существует тогда и только тогда, когда 
\begin{equation}
	rank(J(\mathbf{q})) = rank([J(\mathbf{q}), \dot{\mathbf{s}}]).
\end{equation}

Для пятистепенного манипулятора KUKA Youbot возможны три случая:
\begin{enumerate}
	\item $rank(J) \ne rank([J, \dot{\mathbf{s}}])$~--- не имеет решений;
	\item $rank(J) = rank([J, \dot{\mathbf{s}}])$~--- имеет единственное решение;
	\item $rank(J) = rank([J, \dot{\mathbf{s}}])$~--- бесконечное множество решений.
\end{enumerate}

Таким образом, в сингулярных конфигурациях манипулятора матрица Якоби не позволяется установить однозначную зависимость в~\eqref{ivp}. Поиск решений уравнения~\eqref{ivp} является частью алгоритма управления манипулятором.





\subsection{Система кинематического управления}\label{part_kinematic_control}

Кинематическое управление реализуется в два этапа: первый включает в себя получение траектории движения сочленений манипулятора $ \bm{q}(t) $ или схвата $ \bm{s}(t) $ на интервале времени $ t \in [t_s,\,\,t_f] $; второй этап подразумевает синтез системы управления, способной отработать спланированную траекторию.

Планированию траекторий посвящён следующий раздел. Здесь реализуем систему управления для следования по заданной траектории.

\begin{figure}[h!]
	\centering{\includegraphics[width=0.7\textwidth]{structure_of_actuator_cs.pdf}}
	\vspace{0.5cm}
	\caption{Структурная схема системы управления приводами, встроенная в контроллеры манипулятора}
	\label{img:structure_of_actuator_cs}
\end{figure}

Каждый из приводов манипулятора робота KUKA Youbot имеет собственную систему управления, структура которой иллюстрируется схемой, представленной на рисунке~\ref{img:structure_of_actuator_cs}. Из нее видно, что каждый из приводов робота может управляться заданием значения для угла $ q_{di} $, или скорости $ \dot{q}_{di} $, или момента силы $ \tau_{ed,i} $, который должен быть на нем обеспечен. Это значение подается на вход соответствующего ПИД-регулятора, коэффициенты которого доступны настройке, и далее (уже в виде сигнала напряжения u) на контролируемый двигатель.

Далее в тексте будет рассмотрена система управления, в которой в качестве управляющего сигнала рассматриваются векторы $ \bm{q} $ и $ \dot{\bm{q}} $. Из величин, описывающих состояние робота в данный момент времени, в используемом ПО доступны векторы $\bm{q}(t)$, $\dot{\bm{q}}(t)$ и $\bm\tau_e (t)$.

Цель управления в минимизации ошибки между заданной траекторией и положением схвата в каждый момент времени. Введём обозначения: траекторию обозначим, как:
\begin{equation}
	\bm{s}_d = 
	\begin{bmatrix}
		\bm p_d \\
		\bm\varphi_d
	\end{bmatrix}
	= f(\bm{q}_d),
	\quad
	\dot{\bm{s}}_d = 
	\begin{bmatrix}
		v_d \\
		\omega_d
	\end{bmatrix}
	 = f(\dot{\bm{q}}_d),
\end{equation}
а текущее положение схвата:
\begin{equation}
	\bm{s} = 
	\begin{bmatrix}
		\bm p \\
		\bm\varphi
	\end{bmatrix}
	= f(\bm{q}),
	\quad
	\dot{\bm{s}} = 
	\begin{bmatrix}
		v \\
		\omega
	\end{bmatrix}
	= f(\dot{\bm{q}}),
\end{equation}

%%%\textcolor{red}{Расчёт текущего положения схвата осуществляется методом Ньютона, который касательные и все такое, описанного в~\ref{part_newton}. ДОБАВИТЬ ЧАСТЬ В РЕШЕНИЕ ЗАДАЧ КИНЕМАТИКИ!!1}

%%%%%%%%%%%%%%%
%\textbf{Управление по вектору скорости.}

Для следования по траектории, заданной функциями $ \bm{s}_d(t) $ и  $ \dot{\bm{s}}_d(t) $ реализуем управление по вектору скорости, которое формулируется как минимизация ошибки скорости следования по траектории. Воспользуемся соотношением для обратной задачи о скорости:
\begin{equation}
	\dot{\bm{q}} = J^+(q) \dot{\bm{s}},
\end{equation}
где $ \bm{q} $~--- вектор обобщенных координат, $ \bm{s} $~--- вектор, определяющий положение схвата, $ J^+ $~--- псевдообратная матрица (так как размерность матрицы Якоби для рассматриваемого манипулятора $[6\times 5]$).

Введем ошибку по положению схвата:
\begin{equation}\label{error}
	\bm{e}(t) = \bm{s}_d(t) - \bm{s}(t).
\end{equation}
Затем, дифференцируя~\eqref{error} по времени, получим:
\begin{equation}
	\dot{\bm{e}}(t) = \dot{\bm{s}}_d(t) - \dot{\bm{s}}(t) =
	\dot{\bm{s}}_d(t) - J^+(q) \dot{\bm{q}}.
\end{equation}
Необходимо выбрать вектор $ \dot{\bm{q}}(t) $ таким, чтобы выполнялось условие
\begin{equation}
	\lim_{t\to\infty} \dot{\bm{e}}(t) = 0.
\end{equation}
Тогда, если
\begin{equation}
	\dot{\bm{q}}(t) = J^+(q) \cdot (\dot{\bm{s}}_d(t) - \dot{\bm{e}}(t)),
\end{equation}
то, выбрав 
\begin{equation}
	\dot{\bm{e}}(t) = - K {\bm{e}}(t),
\end{equation}
можно обеспечить асимптотическую устойчивость системы управления.

Схема системы управления изображена на рисунке~\ref{img:velocity_control_system}.
\begin{figure}[h!]
	\centering{\includegraphics[width=1\textwidth]{ipe/velocity_control_system.pdf}}
	\caption{Система управления по вектору скорости}
	\label{img:velocity_control_system}
\end{figure}

В следующем разделе рассмотрим способы задания траектории, которую в последствии будем подавать на вход синтезированной системы управления.



%\subsection{Динамическая модель манипулятора}\label{part_dynamic_model}
\subsubsection{Динамические уравнения}

Для описание динамики манипулятора вводятся в рассмотрение барицентрические СК $Ox_{ci}y_{ci}z_{ci}$\lefteqn,\footnote{Системы координат, чьи начала совпадают с центрами масс соответствующих звеньев.} где $i=\overline{1,5}$, показанные на рисунке~\ref{img_mass_frames}.
Причем, каждая СК $Ox_{ci}y_{ci}z_{ci}$ сонаправлена с~$Ox_iy_iz_i$.

\begin{figure}[h!]
	\centering\includegraphics[height=16.5cm]{kinematics_mass_frames.pdf}
	\caption{Положение барицентрических СК и направление вектора $\vec{g}$.}
	\label{img_mass_frames}
\end{figure}

Для описания положения введенных СК используются следующие векторы:
\begin{equation}
    r^i_{i,\,ci} =
    \begin{bmatrix}
        x_{ci} \\ y_{ci} \\ z_{ci}
    \end{bmatrix}\!\!,\quad i = \overline{1,5},
\end{equation}
где $x_{ci}$, $y_{ci}$ и $z_{ci}$~--- некоторые постоянные величины, $r^i_{j,k}$~--- вектор из начала $Ox_{j}y_{j}z_{j}$ в начало $Ox_{k}y_{k}z_{k}$, выраженный относительно $Ox_{i}y_{i}z_{i}$.

Для компонент тензоров инерции $\mathcal{I}^{i}_i = const$ вводятся обозначения:
\begin{equation}
    \mathcal{I}^{i}_i =
    \begin{bmatrix}
        I_{i,\,xx} & I_{i,\,xy} & I_{i,\,xz} \\
        I_{i,\,xy} & I_{i,\,yy} & I_{i,\,yz} \\
        I_{i,\,xz} & I_{i,\,yz} & I_{i,\,zz}
    \end{bmatrix}\!\!\ldotp
\end{equation}

Вектор гравитации имеет вид:
\begin{equation}
    g_0 =
    \begin{bmatrix}
        0 \\ 0 \\ -g
    \end{bmatrix}\!\!,
\end{equation}
где $g=9.82\text{ м}/\text{с}^2$.

Ниже приводятся формулы для расчета величин, которые потребуются в дальнейшем (далее везде $i = \overline{1,5}$):
\begin{itemize}
    \item для расчета $r^0_{0,\,i}$ и ${}^{0}R_i$:
        \begin{equation}
            {}^0A_i = {}^0A_1 \cdot {}^1A_2 \cdot \ldots \cdot {}^{i-1}A_i;
        \end{equation}
    \item для расчета $r^i_{0,\,i}$:
        \begin{gather}
            r^i_{0,\,i} = {}^{0}R_i^T \cdot r^0_{0,\,i};
        \end{gather}
    \item для расчета $z^0_i$:
        \begin{equation}
            z^0_i = {}^{0}R_i \cdot z^i_i = {}^{0}R_i \cdot
            \begin{bmatrix}
                0 \\ 0 \\ 1
            \end{bmatrix}\!\!;
        \end{equation}
    \item для расчета $g_i$, $v^i_i$ и $\omega^i_i$:
        \begin{equation}\label{eq_transform_of_g_v_omega}
            g_i = {}^{0}R_i^T \cdot g_0,
            \qquad
            v^i_i = {}^{0}R_i^T \cdot v^0_i,
            \qquad
            \omega^i_i = {}^{0}R_i^T \cdot \omega^0_i \ldotp
        \end{equation}
\end{itemize}

%%%%%%%%%%%%%%%%%%%%%%%%%%%%%%%%%%%%%%%%%%%%%%%
%%%%%%%%%%%%%%%%%%%%%%%%%%%%%%%%%%%%%%%%%%%%%%%
%%%%%%%%%%%%%%%%%%%%%%%%%%%%%%%%%%%%%%%%%%%%%%%
\textbf{Вывод уравнений движения}

Для синтеза системы управления, модель манипулятора нужно представить в матричном виде:
\begin{equation}\label{simple_dynamics}
	\tau = D(q) \ddot{q} + C(q,\dot{q}) \dot{q} + G(q),
\end{equation}
где $D(q)$~--- матрица инерции, $C(q,\dot{q})$~--- матрица центробежных и Кориолисовых сил, $G(q)$~--- вектор гравитации, $\tau$~--- вектор моментов.

Выражение для матрицы~$D(q)$ может быть найдено из формулы для кинетической энергии с учетом того, что справедливо
\begin{equation}\label{eq_K_in_form_with_D}
	\left\{
	\begin{aligned}
	\!& K = \frac{1}{2} \, \dot{q}^T D(q) \dot{q}, \\
	\!& D(q) = D^T\!(q),
	\end{aligned}
	\right.
\end{equation}
для матрицы $C(q,\dot{q})$~--- из выражения для $D(q)$ в соответствии с формулами:
\begin{gather}
	C_{ijk} = \cfrac{1}{2} \left( \cfrac{\partial D_{kj}}{\partial q_i} + \cfrac{\partial D_{ki}}{\partial q_j} - \cfrac{\partial D_{ij}}{\partial q_k}\right)\!\!,
	\\
	C_{kj} = \sum_{i = 1}^n C_{ijk} \dot{q}_i,
\end{gather}
где $D_{ij}$, $C_{ij}$~--- элементы матриц $D(q)$ и $C(q,\dot{q})$ соответственно, стоящие на пересечении $i$-ой строки и $j$-го столбца;
а для вектора $G(q)$~--- по формуле
\begin{equation}
	G(q) =
	\begin{bmatrix}
		\cfrac{\partial U}{\partial q_1} &
		\cfrac{\partial U}{\partial q_2} &
		\dots &
		\cfrac{\partial U}{\partial q_5}
	\end{bmatrix}^T\!\!\!\!\!\ldotp
\end{equation}

Потенциальная энергия манипулятора
\begin{equation}
    U =  -\sum_{i=1}^5 \left( m_i g_i^T r^i_{0,\,ci} \right) = -\sum_{i=1}^5 \left( m_i g_i^T r^i_{0,\,i} + g_i^T (m_ir^i_{i,\,ci}) \right)\!.
\end{equation}

Кинетическая энергия манипулятора
\begin{equation}\label{eq_eq_for_K_for_linear_model}
	K = \sum_{i=1}^5 \left( \frac{1}{2} m_i (v^i_i)^T v^i_i + \frac{1}{2} (\omega^i_i)^T \mathcal{I}^{i}_i \omega^i_i + (m_ir^i_{i,\,ci})^T \cdot (v^i_i \times \omega^i_i) \right)  \ldotp
\end{equation}

Якобианы, устанавливающие в соответствии с формулой
\begin{equation}\label{eq_work_of_lin_jacobians}
    v^0_{i} = -J_{vi}\dot{q}, \quad i = \overline{1,5}
\end{equation}
связь между линейными скоростями начал соответствующих СК и вектором~$\dot{q}$:
\begin{gather}
    J_{v1} =
    \begin{bmatrix}
        z^0_0 \times \left( r^0_{0,\,1} - r^0_{0,\,0}\right) & \nv & \nv & \nv & \nv
    \end{bmatrix}\!\!,
    \\
    J_{v2} =
    \begin{bmatrix}
        z^0_0 \times \left( r^0_{0,\,2} - r^0_{0,\,0}\right) & z^0_1 \times \left( r^0_{0,\,2} - r^0_{0,\,1}\right) & \nv & \nv & \nv
    \end{bmatrix}\!\!,
    \\
    J_{v3} =
    \begin{bmatrix}
        z^0_0 \times \left( r^0_{0,\,3} - r^0_{0,\,0}\right) & z^0_1 \times \left( r^0_{0,\,3} - r^0_{0,\,1}\right) &
        z^0_2 \times \left( r^0_{0,\,3} - r^0_{0,\,2}\right) & \nv & \nv
    \end{bmatrix}\!\!,
    \\
    J_{v4} =
    \begin{bmatrix}
        z^0_0 \times \left( r^0_{0,\,4} - r^0_{0,\,0}\right) \\
        z^0_1 \times \left( r^0_{0,\,4} - r^0_{0,\,1}\right) \\
        z^0_2 \times \left( r^0_{0,\,4} - r^0_{0,\,2}\right) \\
        z^0_3 \times \left( r^0_{0,\,4} - r^0_{0,\,3}\right) \\
        \nv
    \end{bmatrix}^T\!\!\!\!\!,
    \qquad
    J_{v5} =
    \begin{bmatrix}
        z^0_0 \times \left( r^0_{0,\,5} - r^0_{0,\,0}\right) \\
        z^0_1 \times \left( r^0_{0,\,5} - r^0_{0,\,1}\right) \\
        z^0_2 \times \left( r^0_{0,\,5} - r^0_{0,\,2}\right) \\
        z^0_3 \times \left( r^0_{0,\,5} - r^0_{0,\,3}\right) \\
        z^0_4 \times \left( r^0_{0,\,5} - r^0_{0,\,4}\right)
    \end{bmatrix}^T\!\!\!\!\!,
\end{gather}
где $\nv = [0\;0\;0]^T$~--- нулевой вектор.

Якобианы, устанавливающие в соответствии с формулой
\begin{equation}\label{eq_work_of_ang_jacobians}
    \omega^0_{i} = -J_{\omega i}\dot{q}, \quad i = \overline{1,5}
\end{equation}
связь между угловыми скоростями звеньев и вектором~$\dot{q}$:
\begin{gather}
    J_{\omega 1} =
    \begin{bmatrix}
        z^0_0 & \nv & \nv & \nv & \nv
    \end{bmatrix}\!\!,
    \qquad
    J_{\omega 2} =
    \begin{bmatrix}
        z^0_0 & z^0_1 & \nv & \nv & \nv
    \end{bmatrix}\!\!,
    \\
    J_{\omega 3} =
    \begin{bmatrix}
         z^0_0 & z^0_1 & z^0_2 & \nv & \nv
    \end{bmatrix}\!\!,
    \qquad
    J_{\omega 4} =
    \begin{bmatrix}
        z^0_0 & z^0_1 & z^0_2 & z^0_3 & \nv
    \end{bmatrix}\!\!,
    \\
    J_{\omega 5} =
    \begin{bmatrix}
        z^0_0 & z^0_1 & z^0_2 & z^0_3 & z^0_4
    \end{bmatrix}\!\!\ldotp
\end{gather}

С учетом полученных выражений, кинетическая энергия может быть переписана в виде:
\begin{gather}
	K = \sum_{i=1}^5 \biggl( \frac{1}{2} m_i \cdot \left( -{}^0R_i^T J_{vi} \dot{q} \right)^T \!\!\cdot \left(-{}^0R_i^T J_{vi} \dot{q}\right) + \frac{1}{2} \left( -{}^0R_i^T J_{\omega i} \dot{q} \right)^T \!\!\cdot \mathcal{I}^{i}_i \cdot \left( -{}^0R_i^T J_{\omega i} \dot{q} \right) + {}\notag\\
	%
	{} + (m_ir^i_{i,\,ci})^T \cdot \Bigl( \left( -{}^0R_i^T J_{vi} \dot{q} \right) \times \left( -{}^0R_i^T J_{\omega i} \dot{q} \right) \Bigr) \biggr) = {}\notag\\
	%
	{} = \sum_{i=1}^5 \biggl(\frac{1}{2} m_i \dot{q}^T J_{vi}^T J_{vi} \dot{q} \!+\! \frac{1}{2} \dot{q}^T J_{\omega i}^T \, {}^0\!R_i \, \mathcal{I}^{i}_i \, {}^0\!R_i^T J_{\omega i} \dot{q} \!+\! (m_i \underbrace{{}^0\!R_i r^i_{i,\,ci}}_{\displaystyle r^0_{i,\,ci}})^T \!\!\cdot \Bigl( \left( J_{vi} \dot{q} \right) \times \left( J_{\omega i} \dot{q} \right) \Bigr) \!\biggr) = {}\notag \\
	%
	{} = \frac{1}{2} \dot{q}^T \Biggl(\sum_{i=1}^5 \Bigl(m_i J_{vi}^T J_{vi} + J_{\omega i}^T \, {}^0\!R_i \, \mathcal{I}^{i}_i \, {}^0\!R_i^T J_{\omega i} + 2 \cdot x\{ m_i r^0_{i,\,ci} \} \!\cdot\! J_{xi}  + {}\notag \\
	%
	{} + 2 \cdot y\{ m_i r^0_{i,\,ci} \} \!\cdot\! J_{yi} + 2 \cdot z\{ m_i r^0_{i,\,ci} \} \!\cdot\! J_{zi}\Bigr) \Biggr) \dot{q}, \label{eq_getting_form_with_D_for_K}
\end{gather}
при преобразованиях учтено то, что
\begin{equation*}
\left( J_{vi} \dot{q} \right) \times \left( J_{\omega i} \dot{q} \right) =
\begin{bmatrix}
J_{vi}^{\{1\}} \dot{q}\\
J_{vi}^{\{2\}} \dot{q}\\
J_{vi}^{\{3\}} \dot{q}
\end{bmatrix}
\times
\begin{bmatrix}
J_{\omega i}^{\{1\}} \dot{q}\\
J_{\omega i}^{\{2\}} \dot{q}\\
J_{\omega i}^{\{3\}} \dot{q}
\end{bmatrix}
=
\begin{bmatrix}
-J_{vi}^{\{3\}} \dot{q} J_{\omega i}^{\{2\}} \dot{q} + J_{vi}^{\{2\}} \dot{q} J_{\omega i}^{\{3\}} \dot{q}\\
J_{vi}^{\{3\}} \dot{q} J_{\omega i}^{\{1\}} \dot{q} - J_{vi}^{\{1\}} \dot{q} J_{\omega i}^{\{3\}} \dot{q}\\
-J_{vi}^{\{2\}} \dot{q} J_{\omega i}^{\{1\}} \dot{q} + J_{vi}^{\{1\}} \dot{q} J_{\omega i}^{\{2\}} \dot{q}
\end{bmatrix}
=
\end{equation*}
\begin{equation}
=
\begin{bmatrix}
-\dot{q}^T \bigl(J_{vi}^{\{3\}} \bigr)^T J_{\omega i}^{\{2\}} \dot{q} +
\dot{q}^T \bigl( J_{vi}^{\{2\}} \bigr)^T J_{\omega i}^{\{3\}} \dot{q}
\\
\dot{q}^T \bigl( J_{vi}^{\{3\}} \bigr)^T J_{\omega i}^{\{1\}} \dot{q} -
\dot{q}^T \bigl( J_{vi}^{\{1\}} \bigr)^T J_{\omega i}^{\{3\}} \dot{q}
\\
-\dot{q}^T \bigl( J_{vi}^{\{2\}} \bigr)^T J_{\omega i}^{\{1\}} \dot{q} +
\dot{q}^T \bigl( J_{vi}^{\{1\}} \bigr)^T J_{\omega i}^{\{2\}} \dot{q}
\end{bmatrix}
=
\begin{bmatrix}
\dot{q}^T \! J_{xi} \dot{q} \\
\dot{q}^T \! J_{yi} \dot{q} \\
\dot{q}^T \! J_{zi} \dot{q}
\end{bmatrix}\!\!,
\end{equation}
где
\begin{align}
&J_{xi} =  - \bigl( J_{vi}^{\{3\}} \bigr)^T J_{\omega i}^{\{2\}} + \bigl( J_{vi}^{\{2\}} \bigr)^T J_{\omega i}^{\{3\}}, \\
&J_{yi} = \phantom{-}\bigl( J_{vi}^{\{3\}} \bigr)^T J_{\omega i}^{\{1\}} - \bigl( J_{vi}^{\{1\}} \bigr)^T J_{\omega i}^{\{3\}}, \\
&J_{zi} =  - \bigl( J_{vi}^{\{2\}} \bigr)^T J_{\omega i}^{\{1\}} + \bigl( J_{vi}^{\{1\}} \bigr)^T J_{\omega i}^{\{2\}} \ldotp
\end{align}

Стоит отметить тот факт, что выражение из~\eqref{eq_getting_form_with_D_for_K}, обозначим которое через $\mathcal{D}(q)$, равное
\begin{gather}
	\mathcal{D}(q) = \sum_{i=1}^5 \Bigl(m_i J_{vi}^T J_{vi} + J_{\omega i}^T \, {}^0\!R_i \, \mathcal{I}^{i}_i \, {}^0\!R_i^T J_{\omega i} + 2 \cdot x\{ m_i r^0_{i,\,ci} \} \!\cdot\! J_{xi} \,\, + {}\notag \\
	%
	{} + 2 \cdot y\{ m_i r^0_{i,\,ci} \} \!\cdot\! J_{yi} + 2 \cdot z\{ m_i r^0_{i,\,ci} \} \!\cdot\! J_{zi}\Bigr),
\end{gather}
в общем случае не равно матрице $D(q)$.
При этом получить последнюю из матрицы $\mathcal{D}(q)$ можно с помощью следующей формулы:
\begin{equation}
	D_{ij} =
	\begin{cases}
		0.5 (\mathcal{D}_{ij} + \mathcal{D}_{ji}), & i \ne j; \\
		\mathcal{D}_{ij}, & i = j;
	\end{cases}
\end{equation}
где $\mathcal{D}_{ij}$~--- элемент матрицы $\mathcal{D}(q)$, стоящий на пересечении $i$-ой строки и $j$-го столбца.


\textbf{Учет динамики приводов}

Уравнения, описывающие динамику приводов, в матричном виде имеют вид
\begin{equation}\label{eq_actuators_dynamic}
	I_a \ddot{q} = \tau_e - \tau,
\end{equation}
где $I_a$~--- диагональная матрица приведенных к выходным валам моментов инерции приводов, $\tau_e$~--- вектор-столбец приведенных к выходным валам приводов моментов силы, развиваемых двигателями, имеющие вид:
\begin{equation}
	I_a =
	\begin{bmatrix}
		I_{a,1} & 0 & \cdots & 0 \\
		0 & I_{a,2} & \cdots & 0 \\
		\vdots & \vdots & \ddots & 0 \\
		0 & 0 & \cdots & I_{a,5}
	\end{bmatrix}\!\!,
	\qquad \qquad
	\tau_e =
	\begin{bmatrix}
		\tau_{e,1} \\ \tau_{e,2} \\ \vdots \\ \tau_{e,5}
	\end{bmatrix}\!\!\ldotp
\end{equation}

Объединяя уравнения~\eqref{eq_dynamic_in_linear} и~\eqref{eq_actuators_dynamic}, имеем 
\begin{equation}
	\tau_e = I_a \ddot{q} + \xi \chi,
\end{equation}
и, добавив в это выражение учет моментов трения, окончательно имеем
\begin{equation}\label{eq_eqs_with_tau_f}
	\tau_e = I_a \ddot{q} + \xi \chi + t_f \ldotp
\end{equation}

Модель трения была выбрана как на поясняемом рисунке~\ref{img_friction_torque} и  описывается следующим уравнением \cite{siciliano2008springer}
\begin{equation}\label{eq_friction_torque}
	\tau_f(\dot{q}) = f_v \dot{q} + f_c \sign(\dot{q}) + f_\text{off},
\end{equation}
где $f_v$, $f_c$~--- диагональные матрицы коэффициентов вязкого и сухого трения соответственно, $f_\text{off}$~--- вектор-столбец сдвигов в моментах силы, имеющие вид
\begin{equation}
	f_v =
	\begin{bmatrix}
		f_{v,1} & 0 & \cdots & 0 \\
		0 & f_{v,2} & \cdots & 0 \\
		\vdots & \vdots & \ddots & 0 \\
		0 & 0 & \cdots & f_{v,5}
	\end{bmatrix}\!\!,
	\quad
	f_c =
	\begin{bmatrix}
		f_{c,1} & 0 & \cdots & 0 \\
		0 & f_{c,2} & \cdots & 0 \\
		\vdots & \vdots & \ddots & 0 \\
		0 & 0 & \cdots & f_{c,5}
	\end{bmatrix}\!\!,
	\quad
	f_\text{off} =
	\begin{bmatrix}
		f_{\text{off},1} \\ f_{\text{off},2} \\ \vdots \\ f_{\text{off},5}
	\end{bmatrix}\!\!\ldotp
\end{equation}

\begin{figure}[h!]
	\centering\includegraphics[width=0.7\textwidth]{friction_torque.pdf}
	\caption{График, поясняющий выбранную модель трения}
	\label{img_friction_torque}
\end{figure}

С~учетом динамики приводов и уравнения~\eqref{simple_dynamics} можно окончательно получить модель манипулятора:
\begin{equation}\label{eq_model_with_standard_matrix}
\tau_e = M(q) \ddot{q} + C(q,\dot{q}) \dot{q} + G(q) + t_f(\dot{q}),
\end{equation}
где $M(q) = I_a + D(q)$.


\subsubsection{Идентификация динамических параметров}
Динамические характеристики робота с необходимой для расчетов точностью не определены, а некоторые, такие как параметры трения, вообще неизвестны. В связи с этим необходимо решить задачу идентификации.

\textbf{Линейная регрессионная модель манипулятора}

Для проведения процедуры идентификации динамических параметров манипулятора необходимо представить динамическую модель в регрессионной форме. Для этого, используя выражение для кинетической и потенциальной энергий, нужно найти функцию Лагранжа:

\begin{gather}
L = K - U = \notag
\\
= \sum_{i=1}^5 \Biggl( m_i \left( \frac{1}{2} (v^i_i)^T v^i_i + g_i^T r^i_{0,\,i} \right) + (m_ir^i_{i,\,ci})^T \cdot \left( v^i_i \times \omega^i_i + g_i \right) + \frac{1}{2} (\omega^i_i)^T \mathcal{I}^{i}_i \omega^i_i \Biggr) = \notag
\\
= \sum_{i=1}^5 \Biggl( m_i \underbrace{\left( \frac{1}{2} (v^i_i)^T v^i_i + g_i^T r^i_{0,\,i} \right)}_{\ds L_{i,1}} + m_i x_{ci} \cdot \underbrace{x\left\{ v^i_i \times \omega^i_i + g_i \right\}}_{\ds L_{i,2}} + \notag
\\
+ m_i y_{ci} \cdot \underbrace{y\left\{ v^i_i \times \omega^i_i + g_i \right\}}_{\ds L_{i,3}} + m_i z_{ci} \cdot \underbrace{z\left\{ v^i_i \times \omega^i_i + g_i \right\}}_{\ds L_{i,4}} + I_{i,\,xx} \cdot \underbrace{\frac{1}{2} \cdot \bigl(x\{\omega^i_i\}\bigr)^2}_{\ds L_{i,5}} +\notag
\\
+ I_{i,\,yy} \cdot \underbrace{\frac{1}{2} \cdot \bigl(y\{\omega^i_i\}\bigr)^2}_{\ds L_{i,6}} + I_{i,\,zz} \cdot \underbrace{\frac{1}{2} \cdot \bigl(z\{\omega^i_i\}\bigr)^2}_{\ds L_{i,7}} + I_{i,\,xy} \cdot \underbrace{x\{\omega^i_i\} \cdot y\{\omega^i_i\}}_{\ds L_{i,8}} +\notag
\\
+ I_{i,\,xz} \cdot \underbrace{x\{\omega^i_i\} \cdot z\{\omega^i_i\}}_{\ds L_{i,9}} + I_{i,\,yz} \cdot \underbrace{y\{\omega^i_i\} \cdot z\{\omega^i_i\}}_{\ds L_{i,10}}\Biggr) \ldotp
\end{gather}

Уравнения движения робота:
\begin{equation}
\frac{d}{dt}\frac{\partial L}{\partial\dot{q_i}} - \frac{\partial L}{\partial q_i} = \tau_i, \quad i = \overline{1,5} \qquad \Rightarrow
\end{equation}
\begin{equation}
\Rightarrow \quad
\left\{
\begin{aligned}
\!&\sum_{i=1}^5 \bigl( m_i \cdot \mathcal{L}_1 \{L_{i,1}\} + m_i x_{ci} \cdot \mathcal{L}_1 \{L_{i,2}\} + \ldots + I_{i,\,yz} \cdot \mathcal{L}_1 \{L_{i,10}\} \bigr) = \tau_1\\
\!&\sum_{i=1}^5 \bigl( m_i \cdot \mathcal{L}_2 \{L_{i,1}\} + m_i x_{ci} \cdot \mathcal{L}_2 \{L_{i,2}\} + \ldots + I_{i,\,yz} \cdot \mathcal{L}_2 \{L_{i,10}\} \bigr) = \tau_2\\
\!&\ldots\\
\!&\sum_{i=1}^5 \bigl( m_i \cdot \mathcal{L}_5 \{L_{i,1}\} + m_i x_{ci} \cdot \mathcal{L}_5 \{L_{i,2}\} + \ldots + I_{i,\,yz} \cdot \mathcal{L}_5 \{L_{i,10}\} \bigr) = \tau_5
\end{aligned}
\right.
\end{equation}
где $\mathcal{L}_j$~--- оператор, работающий в соответствии с формулой:
\begin{equation}
\mathcal{L}_j : \quad \mathcal{L}_j \{f\} = \frac{d}{dt}\frac{\partial f}{\partial\dot{q_j}} - \frac{\partial f}{\partial q_j},
\end{equation}
где в свою очередь $f = f(\dot{q}(t), q(t))$.
Если же заметить, что
\begin{equation}
\mathcal{L}_j \{L_{i,k}\} = 0 \qquad \text{при }j > i, \quad i,j=\overline{1,5}, \quad k=\overline{1,10},
\end{equation}
то выражения для них упрощаются до:
\begin{equation}
\left\{
\begin{aligned}
\!&\sum_{i=1}^5 \bigl( m_i \cdot \mathcal{L}_1 \{L_{i,1}\} + m_i x_{ci} \cdot \mathcal{L}_1 \{L_{i,2}\} + \ldots + I_{i,\,yz} \cdot \mathcal{L}_1 \{L_{i,10}\} \bigr) = \tau_1\\
\!&\sum_{i=2}^5 \bigl( m_i \cdot \mathcal{L}_2 \{L_{i,1}\} + m_i x_{ci} \cdot \mathcal{L}_2 \{L_{i,2}\} + \ldots + I_{i,\,yz} \cdot \mathcal{L}_2 \{L_{i,10}\} \bigr) = \tau_2\\
\!&\ldots\\
\!& m_5 \cdot \mathcal{L}_5 \{L_{5,1}\} + m_5 x_{c5} \cdot \mathcal{L}_5 \{L_{5,2}\} + \ldots + I_{5,\,yz} \cdot \mathcal{L}_5 \{L_{5,10}\} = \tau_5
\end{aligned}
\right.
\end{equation}
или в матричном виде
\begin{equation}\label{eq_dynamic_in_linear}
\tau = \xi \chi,
\end{equation}
где $\tau = [\tau_1, \: \tau_2, \: \ldots, \: \tau_5]^T$~--- вектор обобщенных моментов,\\ $\chi=[\chi_1, \: \chi_2, \: \ldots, \: \chi_5]^T \in \mathbb R^{50}$~--- вектор параметров робота, где в свою очередь
\begin{equation}
\chi_i =
\begin{bmatrix}
m_i & m_i x_{ci} & m_i y_{ci} & m_i z_{ci} & I_{i,\,xx} & I_{i,\,yy} & I_{i,\,zz} & I_{i,\,xy} & I_{i,\,xz} & I_{i,\,yz}
\end{bmatrix}^T\!\!\!\!;
\end{equation}
$\xi$~--- так называемый регрессор, равный
\begin{equation}
\xi =
\begin{bmatrix}
\xi_{1,1} & \xi_{1,2} & \cdots & \xi_{1,5} \\
O_{1 \times 10} & \xi_{2,2} & \cdots & \xi_{2,5} \\
\vdots & \vdots & \ddots & \vdots \\
O_{1 \times 10} & O_{1 \times 10} & O_{1 \times 10} & \xi_{5,5}
\end{bmatrix}\!\!,
\end{equation}
где в свою очередь $O_{1 \times 10}$~--- вектор-строка, состоящая из 10 нулей, а $\xi_{j,i} =$\linebreak $= \xi_{j,i}(\ddot{q}, \dot{q}, q)$~--- вектор-строка, рассчитываемый по формуле
\begin{equation}
\xi_{j,i} =
\begin{bmatrix}
\mathcal{L}_j \{L_{i,1}\} & \mathcal{L}_j \{L_{i,2}\} & \ldots & \mathcal{L}_j \{L_{i,10}\}
\end{bmatrix}\!\!\ldotp
\end{equation}

Подставляя~\eqref{eq_friction_torque} в~\eqref{eq_eqs_with_tau_f}, получим
\begin{equation}
\tau_e = I_a \ddot{q} + \xi \chi + f_v \dot{q} + f_c \sign(\dot{q}) + f_\text{off}\ldotp
\end{equation}
Если ввести в рассмотрение новые матрицы $\bar{\chi}=[\bar{\chi}_1, \: \bar\chi_2, \: \ldots, \: \bar\chi_5]^T$ и\linebreak
\begin{equation}
\bar\xi =
\begin{bmatrix}
\bar\xi_{1,1} & \bar\xi_{1,2} & \cdots & \bar\xi_{1,5} \\
O_{1 \times 10} & \bar\xi_{2,2} & \cdots & \bar\xi_{2,5} \\
\vdots & \vdots & \ddots & \vdots \\
O_{1 \times 10} & O_{1 \times 10} & O_{1 \times 10} & \bar\xi_{5,5}
\end{bmatrix}\!\!,
\end{equation}
определяемые выражениями
\begin{equation}
\bar{\chi}_i =
\begin{bmatrix}
\chi_i & I_{a,i} & f_{v,i} & f_{c,i} & f_{\text{off},i}
\end{bmatrix}^T\!\!\!\!,
\end{equation}
\begin{equation}
\bar{\xi}_{j,i} =
\left\{
\begin{aligned}
\!\begin{bmatrix}\xi_{j,i} & 0 & 0 & 0 & 0\end{bmatrix}&, && i \ne j \\
\!\begin{bmatrix}\xi_{j,i} & \ddot{q_j} & \dot{q_j} & \sign(\dot{q_j}) & 1\end{bmatrix}&, &&i = j
\end{aligned}
\right.
\end{equation}
то данное выражение может быть записано в следующем матричном виде:
\begin{equation}\label{eq_extended_dynamic_in_linear}
\tau_e = \bar\xi \bar\chi \ldotp
\end{equation}





\input{dynamics/dynamic_model_control}


%\subsection{Планирование движения}
%Необходимо выбрать траекторию движения схвата манипулятора такую, чтобы законы изменения положения, скоростей и ускорения, с одной стороны, соответствовали параметрам движения конвейера, а с другой~--- возможностям манипулятора.




\clearpage\newpage