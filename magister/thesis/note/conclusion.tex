\newpage
\section*{Заключение}
\addcontentsline{toc}{section}{Заключение}
\nocite{aldoma2012tutorial}\nocite{papon2013voxel}

В рамках данной выпускной квалификационной работы было произведено моделирование синтезированной системы управления манипулятором на базе манипулятора робота KUKA YouBot  для следования по заданной траектории, планирования траекторий и системы технического зрения для обнаружения целевого объекта на вращающемся столе и оценка траектории его движения.

В ходе работы были достигнуты следующие результаты:

\begin{enumerate}
	\item Построена кинематическая модель манипулятора KUKA Youbot, в рамкох которой были решены прямая и обратная задачи по положению и скорости;
	\item Реализованы алгоритмы планирования траектории в пространстве обобщенных координат для перехода из точки в точку, для обхода заданного набора точек и, в операционном пространстве~--- для следования по заданной траектории;
	\item Реализована система технического зрения, способная обнаруживать объекта элементарной геометрической формы на плоских поверхностях, оценивать положение и ориентацию объекта относительно системы координат камеры и оценивать радиус кривизны траектории движения обнаруженного предмета;
	\item были реализованы два ros-пакета: для системы технического зрения и системы управления и планирования траекторий.
\end{enumerate}

Дальнейшую разработку планируется посвятить моделированию полученной робототехнической системы в симуляторе V-Rep, а затем и на реальном роботе KUKA YouBot.

Хочется выразить слова благодарности моему коллеге Антонову Евгению Сергеевичу, а также всему коллективу студенческого конструкторского бюро кафедры Систем управления и информатики.

\newpage