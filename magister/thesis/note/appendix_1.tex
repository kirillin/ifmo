\ESKDappendix{обязательное}{Программный код в Scilab}\label{append_app_example}

В этом приложении представлен программный код в Scilab, используемый в этой работе.

Программный код функции для расчета коэффициентов кубических сплайнов с фиксированными концами для интерполяции траектории, заданной набором точек $ q_i $ в моменты времени $ t_i $:

\begin{lstlisting}
function [a,b,c,d]=splin_v(t, q, dq0, dqf)
	// Computes a cubic spline interpolation function 
	//  in Hermit form.
	//
	// Params:
	//      t       vector of the time for knots (n >= 3)
	//      q       vector of knots
	//      dq0     initial velocity
	//      dqf     final velocity
	//
	// Returns:
	//      a,b,c,d coeffitions for cubic spline
	n = length(q);  // quantity knots
	a = []; b = []; c = []; d = [];
	delta = [];
	z = []; mu = []; l = [];
	h = [];
	a = q;
	for i = 1:n-1 do
		h(i) = t(i+1) - t(i);
	end
	delta(1) = 3 *(a(2) - a(1)) / h(1) - 3 * dq0;
	delta(n) = 3 * dqf - 3 * (a(n) - a(n-1)) / h(n-1);
	for i = 2:n-1 do
		delta(i) = 3 * (a(i+1) - a(i)) / h(i) - 3 * (a(i) 
		          - a(i-1)) / h(i-1);
	end
	// forward
	mu(1) = 0.5;
	z(1) = delta(1) / (2*h(1)); 
	for i = 2:n-1 do   
		l = 2 * (t(i+1) - t(i-1)) - h(i-1) * mu(i-1);
		mu(i) = h(i) / l;
		z(i) = ( delta(i) - h(i-1) * z(i-1) ) / l;
	end
	// inverse
	l = h(n-1) * (2 - mu(n-1));
	z(n) = ( delta(n) - h(n-1) * z(n-1) ) / l;
	c(n) = z(n);   
	for i = n-1:-1:1 do 
		c(i) = z(i) - mu(i) * c(i+1);
		b(i) = (a(i+1) - a(i)) / h(i) - h(i) * (c(i+1) 
				+ 2 * c(i)) / 3;
		d(i) = (c(i+1) - c(i)) / (3 * h(i));
	end
endfunction
\end{lstlisting}

Программный код функции для расчета коэффициентов кубических сплайнов со свободными концами:

\begin{lstlisting}
function [a,b,c,d]=splin_a(t, q)
	n = length(q);
	a = []; b = []; d = [];
	K = zeros(1,n+1); K(2) = 0;
	c = zeros(1,n+1); c(2) = 0; 
	for i = 3:n do
		j = i - 1;
		m = j - 1;
		A = t(i) - t(j);
		B = t(j) - t(m);
		R = 2 * (A + B) - B * c(j);
		c(i) = A / R;
		K(i) = (3 * ((q(i)-q(j))/A - (q(j)-q(m))/B) 
				- B*K(j)) / R;
	end
	c(n+1) = K(n+1);
	for i = n:-1:3 do
		c(i) = K(i) - c(i) * c(i+1);
	end
	for i = 2:n do
		hi = t(i) - t(i-1)
		a(i-1) = q(i-1);
		b(i-1) = (q(i) - q(i-1)) / hi - (c(i+1) 
					+ 2 * c(i))*hi/3;
		d(i-1) = (c(i+1) - c(i)) / 3 / hi;
	end
	c(1:n) = c(2:n+1); 
endfunction
\end{lstlisting}

Функция для расчета коэффициентов сплайна пятой степени:
\begin{lstlisting}
function [qt,qdt,qddt]=polynomal_trajectory_5(qs, 
		qf, tv, dqs, dqf, v_max)
	// Compute trajectory for 5-th polynom
	// q(t) = a + bt + ct^2 + dt^3 + et^4 +ft^5
	n = length(qs);
	if length(tv)>1 then
		tscal = max(tv);
		t = tv(:)/tscal;
	else
		tscal = 1;
		// normalized time from 0 -> 1
		t = (0:(tv-1))''/(tv-1); 
	end;
	qt = []; qdt = []; qddt = [];
	a = qs;
	b = dqs;
	c = zeros(n,1);
	d = 10 *(qf - qs) - (6*dqs + 4*dqf); 
	e = -15 * (qf - qs) + (8*dqs + 7*dqf);
	f = 6 * (qf - qs) - (3*dqf + 3*dqs);
	times = [ones(length(t),1) t t.^2 t.^3 t.^4 t.^5];
	coefs = [a b c d e f]';
	qt = times * coefs;
	coefs = [ b zeros(n,1) 3*d 4*e 5*f zeros(n,1)]';
	//    qdt = times * coefs;
	qdt = times * coefs / max(times * coefs)
	qdt = qdt * v_max
	coefs = [  zeros(n,1) 6*d 12*e 20*f zeros(n,1) zeros(n,1)]';
	qddt = times * coefs;
endfunction
\end{lstlisting}

Функция рассчитывающая траекторию составленную из трех частей: параболы, прямой, параболы:
\begin{lstlisting}
function [qt,qdt,qddt, t, tb, tf]=plp(qs, 
		qf, step, v_des)
	// Compute trajectrory with velocity 
	//  profile -- parabolic-linear-parabolic
	v = v_des;
	tf = 1.5*(qf - qs) / v; 
	tf = 2
	v = 1.5*(qf - qs) / tf; 
	t = 0:step:tf;
	qt = zeros(length(t), 1); qdt = qt; qddt = qt;
	tb = (qs - qf + v*tf) / v;
	a = v/tb;
	for i = 1:length(t)
		tt = t(i);
		if tt <= tb then
			// initial blend
			qt(i) = qs + a/2*tt^2;
			qdt(i) = a*tt;
			qddt(i) = a;
		elseif tt <= (tf-tb)
			// linear motion
			qt(i) = (qf + qs - v*tf) / 2 + v * tt;
			qdt(i) = v;
			qddt(i) = 0;
		else
			// final blend
			qt(i) = qf - a/2*tf^2 + a*tf*tt - a/2*tt^2;
			qdt(i) = a*tf - a*tt;
			qddt(i) = -a;
		end
	end
endfunction
\end{lstlisting}

Функция возвращающая точку дуги в операционном пространстве в заданный момент времени:
\begin{lstlisting}
function [x,y,alpha]=get_circle_point(t, v, theta_s, alpha0)
	phi = v * t / (%pi * Rc);
	x = Rc * cos(s + phi);
	y = Rc * sin(s + phi);
	alpha = alpha0 + phi;
endfunction
\end{lstlisting}