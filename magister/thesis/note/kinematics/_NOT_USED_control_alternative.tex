\textbf{Управление по положению.}

Первым реализуем алгоритм управления из предположения, что величина отклонения $ \Delta \bm{s} = \bm{s}_d - \bm{s} $ несущественна, тогда запишем следующие рекуррентные соотношения~\cite{zenk2000}:
\begin{gather}
	\label{controller1}
	\Delta q_{k+1} = J^{-1}(q_k) \Delta s_k, \\
	\Delta s_k = q_{k+1} - q_k, \\
	\Delta s_k = 
	\begin{bmatrix}
		\Delta \varphi_k\\
		\Delta p_k
	\end{bmatrix}
	=
	\begin{bmatrix}
		\omega_k \Delta t \\
		v_k \Delta t
	\end{bmatrix}\!\!\!,
\end{gather}
где $ \Delta s_k $~--- вектор ошибки на k-ой итерации между заданным положением схвата и текущим в момент времени t, которую можно минимизировать за время $\Delta t$, путем придания схвату линейной $ v_k $ и угловой $ \omega_k $ скоростей; компоненты вектора $ \Delta s_k $, представлены выражениями:
\begin{gather}
	\Delta p_k = p_d - p_k, \\
	\Delta \varphi_k = \cfrac{1}{2} (x_k \times x_d + y_k \times y_d + z_k \times z_d),
\end{gather}
где $ x_d, y_d, z_d $~--- соответствующие столбцы матрицы, задающей положение и ориентацию схвата:
\begin{equation}\label{Trajectory_r}
	T = T_d(t) = 
	\begin{bmatrix}
		x_d(t) & y_d(t) & z_d(t) & p_d(t)\\
		0&0 &0 &1
	\end{bmatrix}.
\end{equation}

\begin{figure}[h!]
	\centering{\includegraphics[width=0.9\textwidth]{ipe/linear_control_system.pdf}}
	\vspace{0.1cm}
	\caption{Система управления по положению}
	\label{img:controll_system_position}
\end{figure}

Из предположения о сходимости итерационного процесса~\eqref{controller1} за одну итерацию, будем рассматривать~\eqref{controller1}--\eqref{Trajectory_r} как алгоритм слежения схвата манипулятора за заданной траекторией движения~\eqref{Trajectory_r}. Схема системы управления приведена на рисунке~\ref{img:controll_system_position}
